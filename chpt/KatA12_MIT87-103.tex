\section*{\begin{tabular*}{\linewidth}{@{}l @{\extracolsep{\fill}} r@{}}
Nr.~12 & MIT~87/103 \\
\end{tabular*} 
}

\textsf{\textbf{Mitula (\mbox{Sangha}, Fpl.~251)}}

\vspace{1em}

\noindent\begin{tabular}{@{}rl@{}}
	\textbf{Feldarbeit:} & \begin{tabular}[t]{@{}l@{}}\textbf{17.06.1987 (M. K. H. Eggert,}\\ \textbf{H. Holsten, K. Misago)}\end{tabular} \\ 
	\textbf{Taf.:} & \textbf{42.10--14} \\ 
	\textbf{Lit.:} & \textbf{\textsc{Eggert} 1992, 1993} \\ 
\end{tabular} 

\paragraph{Grabung und Befunde}\hspace{-.5em}|\hspace{.5em}%
Neben etwas im Dorf verteilter Keramik, die unter der Kennung MIT~87/101 aufgenommen wurde, und zwei am Ufer angelegten schnellen Sondagen (MIT~87/102), fand sich bei der Prospektion in Mitula am \mbox{Sangha} auch ein Komplex mit Imbonga-Keramik. Dieser wurde unter der Kennung MIT~87/103 aufgenommen. Zwischen einigen Hütten fanden sich die Fragmente eines Gefäßes und einiger Scherben. Eine mit einer potenziellen Grube in Zusammenhang stehende Verfärbung wird nicht berichtet. Es erfolgte keine detaillierte Dokumentation der Situation.

\paragraph{Keramik}\hspace{-.5em}|\hspace{.5em}%
Das Inventar umfasst Teile eines Gefäßes mit geschweifter Wandung und einer aus Bögen bestehendem Rillenzier (Taf.~41.10). Des Weiteren fand sich ein Fragment eines flachen, unverzierten Bodens (Taf.~42.13) sowie einige Ritzverzierung aufweisende Wandungscherben (Taf.~42.11--12, 42.14). Mit Blick auf die technischen Eigenschaften der Scherben lassen sich alle Stücke dem Spektrum der Imbonga-Gruppe zuordnen, wie es auch für das Material des Inneren Kongobeckens von \textcite[59--68]{Wotzka.1995} beschrieben wurde. Ein guter Vergleich für das Gefäßfragment aus Mitula findet sich in Iyonda am Kongo \parencite[Fpl.~8; ][441 Taf. 7,7; Fpl.~8]{Wotzka.1995}.

\paragraph{Datierung}\hspace{-.5em}|\hspace{.5em}%
Eine unter den Scherben entnommene Holzkohleprobe datiert in das 6.--1. Jh. v.~Chr. (Tab.~\ref{tab:MIT87-103_14C-Daten}). Trotz eines vergleichsweise großen Standardfehlers deckt sich diese Datierung mit den veröffentlichten Datierungen für die Imbonga-Gruppe (Abb.~\ref{fig:14C_InnerCongo_Stylegroups}; Tab.~\ref{tab:Wotzka1995-412_14C_Repr}), die von \textcite[66\,f.]{Wotzka.1995} in das 4.--2.~Jh. v.~Chr. datiert wird.