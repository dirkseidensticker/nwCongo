\section*{\begin{tabular*}{\linewidth}{@{}l @{\extracolsep{\fill}} r@{}}
Nr.~5 & MLB~85/103\\
\end{tabular*} 
}

\textsf{\textbf{Maluba (Lua; Fpl.~230)}}

\vspace{1em}

\noindent\begin{tabular}{@{}rl@{}}
\textbf{Feldarbeit:} & \textbf{05.09.1985 (M. K. H. Eggert)} \\ 
\textbf{Abb.:} & \textbf{\ref{fig:MLB85-103_ProfilFoto}} \\
\textbf{Taf.:} & \textbf{27.10--27.13} \\ 
\textbf{Lit.:} & \textbf{--} \\ 
\end{tabular}

\paragraph{Grabung und Befunde}\hspace{-.5em}|\hspace{.5em}%
Etwa 3\,m südwestlich der Grube MLB~85/2 (Kat.-Nr.~4) fand sich eine weitere, zwischen 2,7--3\,m tiefe Grube, mit einem Durchmesser von 1,5--1,6\,m (Abb.~\ref{fig:MLB85-103_ProfilFoto}).\footnote{Die Grube ist potenziell eher noch tiefer gewesen. Ein guter Teil des Befundes war bereits erodiert und zur rezenten Oberfläche des Dorfes fehlen einige Dezimeter.} Die vorhanden Fotos zeigen, dass die Grube eine horizontale Sohle sowie steilschräg überkippte Wandungen aufweist. Nach dem Einsturz des benachbarten Befundes MLB~85/2 wurde das Profil dieser Grube geputzt und einige Scherben abgesammelt.\footnote{Da keine Grabung erfolgte, wurde der Befund MLB~85/103 mit einer Kennung als Oberflächenkomplex aufgenommen.} Wie die Grube MLB~85/2 weist auch MLB~85/103 eine feine, lagige Verfüllung mit hellen, sandigen und dunklen, holzkohlehaltigen Schichten und eine scharfe Außengrenze auf (Abb.~\ref{fig:MLB85-103_ProfilFoto}).

\paragraph{Keramik\vspace{.5em}}\mbox{}\\
\begin{tabular}{@{}lrl@{}}
Bearbeitet:	& 829\,g & (100\,\%) \\ 
\end{tabular} 

\vspace{1em}
\noindent Das Inventar der aus dem Befund MLB~85/103 exemplarisch geborgenen Keramik umfasst unter anderem roulettverzierte Stücke. Insgesamt liegen aus dem Befund 15 GE vor, die nahezu alle am mittleren \mbox{Ubangi} angetroffenen Stilgruppen repräsentieren: eine GE des Batalimo-Maluba-Stils (Kap.~\ref{sec:BTM-Gr}), fünf GE der Dongo-Gruppe (Kap.~\ref{sec:DON-Gr}), eine GE der Bobulu-Gruppe (Kap.~\ref{sec:BBL-Gr}), zwei GE der Motenge-Boma-Gruppe (Kap.~\ref{sec:MTB-Gr}) sowie drei GE des rezenten Dama-Stils (Kap.~\ref{sec:DAM-Gr}). Daneben wurde noch eine mit Schnitzroulette verzierte Scherbe sowie zwei nicht genauer einzuordnende Stücke geborgen. Der Fund rezenter, mit \mbox{Roulette} verzierter Keramik unterstreicht das potenziell junge Alter des Befundes.

