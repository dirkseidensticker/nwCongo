\section{Regionale Sequenzen}\label{sec:Sequenzen}

Die im vorangegangen Kapitel \ref{sec:Keramiksequenz} beschriebenen keramischen Stilgruppen zeichnen die Besiedlung des Arbeitsgebiets vom ersten Auftreten von Keramik bis heute nach. Als Ausgangspunkt für eine systematische Rekonstruktion der Besiedlungsabfolge dienen regional differenzierte Betrachtungen der einzelnen Flussgebiete des \mbox{Ubangi} und Lua, des \mbox{Likwala}-\mbox{aux}-\mbox{Herbes} sowie des \mbox{Sangha} und \mbox{Ngoko}. Die unterschiedlich intensive Forschungsaktivität an den einzelnen Fundstellen führt zu quantitativen und damit auch qualitativen Schwankungen der jeweiligen Spektren keramischer Stilgruppen. Besiedlungsgeschichtlich relevante Aussagen gründen daher auf einer regionalen Betrachtungsweise, welche die genannten Flussabschnitte abdeckt.\footnote{Die tabellarischen Übersichten (Tab.~\ref{tab:UbangiLuaSequenz}--\ref{tab:SanghaNgokoSequenz}) setzen sich aus den von Nord nach Süd aufgelisteten Fundstellen sowie einer von links nach rechts abgetragenen Auflistung der Keramikstile zusammen. Die Auflistung der Fundorte erfolgt dabei absteigend der entgegen der Stromrichtung vergebenen, an die entsprechende Liste von \textcite[542\,f. Karte~1]{Wotzka.1995} anknüpfenden, laufenden Nummern (siehe Anlage~1.B). Sie spiegelt dadurch aber die durch die im Arbeitsgebiet erfolgten Befahrungen erschlossenen, durch den Naturraum laufenden Nord--Süd-Transekte wider. Die Auflistung der keramischen Stilgruppen von links nach rechts folgt der Abfolge ihrer Beschreibungen in Kapitel \ref{sec:Keramiksequenz} und damit einer generalisierten chronologischen Reihenfolge. Das in den jeweiligen tabellarischen Darstellungen gewonnene Bild zeichnet auch die jeweilige Intensität der lokalen Forschungsaktivität nach. Fundstellen, an denen ausgedehnte Surveys sowie Grabungen stattgefunden haben (siehe Tab.~\ref{TabBefundeUntersucht}), weisen häufiger ein breiteres Spektrum keramischer Stilgruppen auf (siehe ebd. 226 Anm.~1). Die gewählte Form der Übersicht liefert daher auch einen Überblick über die Zusammensetzung der stilistisch ansprechbaren Inventare eines jeden Fundplatzes.} Die drei betrachteten regionalen Sequenzen zeichnen dabei jeweils individuelle Entwicklungslinien nach.


\subsection{\mbox{Ubangi}- und Lua-Gebiet}\label{sec:SequenzUbangiLua}

\begin{table*}[!tb]
	\centering
	\resizebox{1.02\textwidth}{!}{%
		{\footnotesize 
	\begin{sftabular}{@{}rlcccccccccccccccc@{}}
\toprule
\textbf{Nr.} &\textbf{Fundort} & \textbf{BTM} & \textbf{NGB} & \textbf{DON} & \textbf{MKL} & \textbf{BBL} & \textbf{BKW} & \textbf{MTB} & \textbf{KPT} & \textbf{DAM} & \textbf{MBN} & \textbf{BAN} & \textbf{MAT} & \textbf{EBA} & \textbf{BDG} & \textbf{MBA} & \textbf{BOT} \\
\midrule
 232 & Ilawa & & & $\circ $ & & & & $\bullet $ & & & $\bullet $ & & & & & & \\
 230 & Maluba & $\bullet $ & $\circ $ & $\bullet $ & & $\bullet $ & & $\bullet $ & & $\bullet $ & $\bullet $ & & $\bullet $ & & & & \\ \hdashline[0.5pt/5pt]
 229 & Kouango & & & & $\circ $ & & & & $\bullet $ & $\circ $ & & $\circ $ & & & & & \\
 228 & Sidi & & & & & & & & & $\bullet $ & & & & & & & \\
 227 & Ndengu & & & & & & & & & $\bullet $ & & $\circ $ & & & & & \\
 226 & Gbandami & & & & & & & & $\bullet $ & $\bullet $ & & $\circ $ & & & & & \\
 225 & Boduna & & & & & & & & $\circ $ & $\bullet $ & & & & & & & \\
 224 & Dokeve 2 & & & & $\bullet $ & & & & & $\bullet $ & & & & & & & \\
 222 & Dama 1 & & & & & & & & & $\bullet $ & & & & & & & \\
 220 & Kpetene & & & & & & & & $\bullet $ & & $\bullet $ & & & & & & \\
 217 & Mboko 1 & & & $\circ $ & $\bullet $ &  & & $\bullet $ & & & & & $\bullet $ & & & & \\
 215 & Bangui & & & & & & & & & & & $\bullet $  & & & & & \\
 214 & Balongoi & & & $\bullet $ & & & & $\bullet $ & $\circ $ & & & $\bullet $ & & & & & \\
 213 & Mokelo & $\bullet $ & & &  $\bullet $ &- & & $\bullet $ & & & & & & & & & \\
 212 & Mondoli & $\circ $ & & & $\bullet $ & & & $\bullet $ & & & & & & & & & \\
 211 & Mboma & & & $\bullet $ & & & & $\bullet $ & & & & & & & & & \\
 210 & Bomboko & $\bullet $ & & & & & & $\circ $ & & & & & & & & & \\
 209 & Batanga & & & $\circ $ & $\bullet $ & & & $\bullet $ & & & $\bullet $ & $\bullet $ & & & & & \\
 208 & Libenge & $\bullet $ & & & & & & $\bullet $ & & & $\bullet $ & $\bullet $ & $\circ $ & & & & \\
 207 & Maoko & $\bullet $ & $\circ $ & & $\circ $ & $\bullet $ & & $\bullet $ & & & $\bullet $ & & & & & & \\
 206 & \begin{tabular}[c]{@{}l@{}}Motenge-\\Boma\end{tabular} & $\circ $ & & $\bullet $ & & $\circ $ & & $\bullet $ & & $\bullet $ & & & & & & & \\
 205 & Nzambi & $\bullet $ & $\bullet $ & $\bullet $ & $\bullet $ & $\circ $ & & $\bullet $ & & $\circ $ & & & & & & & \\
 204 & \begin{tabular}[c]{@{}l@{}}Mbati-\\Ngombe\end{tabular} & $\bullet $ & & $\bullet $ & & & & $\bullet $ & & & $\bullet $ & & & & & & \\
 203 & \begin{tabular}[c]{@{}l@{}}\mbox{Ubangi}\\Fkm 415,5\end{tabular} & $\circ $ & & & & & & & & & & & & & & & \\
 202 & Dongo & $\bullet $ & & $\bullet $ & $\bullet $ & & & $\bullet $ & & & $\bullet $ & & & & & & \\
 201 & Imese & & & & & & & $\circ $ & & & $\bullet $ & & & & & & \\
 200 & \begin{tabular}[c]{@{}l@{}}Bousoka-\\Mangombe\end{tabular} & & & $\bullet $ & & & & & & $\bullet $ & & & & & & & \\
 199 & \mbox{Ngbanja} & $\bullet $ & $\bullet $ & $\bullet $ & & & & & & & & & & & $\bullet $ & & \\
 198 & Bobulu & & & & $\circ $ & $\bullet $ & & & & & & & & & & & $\circ $ \\
 197 & Ebeka & & $\circ $ & & & $\bullet $ & & & & & $\circ $ & & $\bullet $ & & & & \\
 196 & Boyoka & & & & & $\bullet $ & $\bullet $ & & & & & & & & & & $\bullet $ \\
 194 & Bolumbu & & & & & & & & & & & & & & & $\circ $ & $\bullet $ \\
 193 & Loka & & $\circ $ & & & & $\bullet $ & & & & & & & & $\bullet $ & & $\bullet $ \\
 192 & Ilanga & & & & & & $\bullet $ & & & & & & & & & & $\bullet $ \\
 191 & Zamba & & & & & & & & & & & & & & $\circ $ & $\circ $ & $\circ $ \\
 190 & Bokwango & & & & & & $\bullet $ & & & & & & & & & & $\bullet $ \\
 189 & Bobangi & & & & & & & & & & & & & $\circ $ & & & $\bullet $ \\
 188 & Lokekya & & & & & & $\circ $ & & & & & & & & & & \\
 186 & Bruxelles* & & & & & & $\bullet $ & & & & & & & & & & \\
\bottomrule
\end{sftabular}

		}
	}
	\caption{\mbox{Ubangi}- und Lua-Gebiet: Nachgewiesene keramische Stilgruppen (siehe Anlage~3) nach Fundorten (von Nord nach Süd).\\$\bullet$ belegt, $\circ$ fraglich;~* Dorfwüstungen (\textit{bilali}; Sing. \textit{elali}).}
	\label{tab:UbangiLuaSequenz}
\end{table*}

Die Inventare aus dem Gebiet entlang des \mbox{Ubangi} ergeben einen etwa 850\,km langen (Tab.~\ref{tab:ArbeitsgebietFlussstrecken}), von der Baumsavanne nördlich von Bangui (Fpl.~215) bis in das Herz des äquatorialen Regenwalds bei Mbandaka (Fpl.~10) reichenden Nord--Süd-Transekts.\footnote{Die entlang des Lua gelegenen Fundstellen wurden, da sich innerhalb ihrer Inventare keine Unterschiede zu den keramischen Gruppen des mittleren \mbox{Ubangi}-Gebietes ergaben, in die regionale Perspektive des \mbox{Ubangi} integriert. Es wird folglich vom \mbox{Ubangi}- und Lua-Gebiet gesprochen.} Aufgrund des in Relation mit den anderen Flussabschnitten geringen Fundaufkommens (siehe Abb.~\ref{fig:FundGewVSFlussKM}) sowie der ausschließlich in Maluba am Lua durchgeführten Grabungen (Kat.-Nr.~1--5), lassen sich besiedlungsgeschichtlich relevante Aussagen nur in eingeschränktem Maße ableiten. Die tabellarische Übersicht der Stilgruppen zeichnet den keramischen Entwicklungsgang entlang des prospektierten Abschnitts der Flüsse \mbox{Ubangi} und Lua in Grundzügen nach (Tab.~\ref{tab:UbangiLuaSequenz}).

Den Beginn der regionalen Sequenz bilden die Stilgruppen Batalimo-Maluba (Kap.~\ref{sec:BTM-Gr}) und \mbox{Ngbanja} (Kap.~\ref{sec:NGB-Gr}), die vor allem an Fundstellen im Bereich des mittleren \mbox{Ubangi} sowie am Lua vertreten sind. Südlich von \mbox{Ngbanja} (Fpl.~199) sowie nördlich von Mokelo (Fpl.~213) finden sich keine Stücke, die hinreichend sicher den beiden, in die Frühe Eisenzeit datierenden Stilgruppen zugewiesen werden können (Kap.~\ref{sec:Zeitscheiben}). Ein ganz ähnliches Verbreitungsgebiet zeigt die potenziell jüngere Dongo-Keramik (Kap.~\ref{sec:DON-Gr}), mit der erstmals in sehr eingeschränktem Rahmen Rouletteverzierung am \mbox{Ubangi} und Lua belegt werden kann. Die im Verbreitungsgebiet dieser drei Gruppen ablesbare Grenze im Süden bei \mbox{Ngbanja} (Fpl.~199) wird erst von der Bobulu-Keramik \textit{überschritten} (Kap.~\ref{sec:BBL-Gr}). Diese ebenfalls in die Jüngere Eisenzeit zu datierende Stilgruppe findet sich direkt südlich angrenzend der Dongo-Gruppe an den Plätzen Boyoka (Fpl.~196), Ebeka (Fpl.~197) sowie dem eponymen Fundplatz Bobulu (Fpl.~198). Eine weitere, an den Beginn der Jüngeren Eisenzeit datierende und ebenfalls in begrenztem Rahmen Roulette-Verzierung aufweisende Stilgruppe ist die Mokelo-Gruppe (Kap.~\ref{sec:MKL-Gr}), die sich knapp nördlich der Dongo-Gruppe findet, mit einem Hauptverbreitungsgebiet direkt südlich des \mbox{Ubangi}-Bogens (Abb.~\ref{fig:MKL_Verbreitung}).\pagebreak

Mit der in die Jüngere Eisenzeit zu datierenden Motenge-Boma-Keramik (Kap. \ref{sec:MTB-Gr}) kommt in der Jüngeren Eisenzeit eine Stilgruppe auf, die ein sehr klar abgegrenztes Verbreitungsgebiet aufweist und mit der die Rouletteverzierung das bestimmende Verzierungselement der Keramik wird. Das Verbreitungsgebiet der Motenge-Boma-Gruppe ist dabei ähnlich klar zu beschreiben, wie vorher lediglich bei der Batalimo-Maluba-Gruppe. Dies ist zum Teil aber auch auf die deutlich diagnostischen, individuellen Merkmale zurückzuführen, die eine Ansprache entsprechender Stücke begünstigen. Vertreter der Motenge-Boma-Gruppe fanden sich lediglich am mittleren \mbox{Ubangi}, zwischen Dongo (Fpl.~202) im Süden und Mboko~I (Fpl.~217) im Norden, sowie entlang des Lua. Mit der Motenge-Boma-Gruppe werden in Roulettetechnik ausgeführte Verzierungen das bestimmende Charakteristikum der lokalen Töpfereitradition.\footnote{Siehe Anm.~\ref{ftn:EthnoToepfereiInVorb}.} Die drei rezenten, Rouletteverzierung verwendenden Stile Dama (Kap.~\ref{sec:DAM-Gr}), Mbati-Ngombe (Kap.~\ref{sec:MBN-Gr}) und Kptene (Kap.~\ref{sec:KPT-Gr}) sind südlich der Mündung des Lua nicht anzutreffen. Während die Mbati-Ngombe-Keramik vor allem am mittleren \mbox{Ubangi} sowie Lua verbreitet ist, liegt das Hauptverbreitungsgebiet der Dama-Gruppe weiter nördlich, stromauf des \mbox{Ubangi}-Bogens und des eponymen Platzes Dama~I (Fpl.~222). Ein zusätzliches Verbreitungsgebiet der Dama-Keramik befindet sich im Bereich der Mündung des Lua, zwischen Bousoka-Mangombe (Fpl.~200) und Motenge-Boma (Fpl.~206).\footnote{Ob dieses nachrangige Verbreitungsgebiet auf das Distributionsnetz der Dama-Keramik zurückgeführt werden kann, muss beim gegenwärtigen Quellenstand offenbleiben. Zum generellen Themenkreis des Handels mit keramischen Erzeugnissen und der daraus ableitbaren Schlüsse für die Interpretation von Verbreitungsgebieten von Keramiken sei an dieser Stelle lediglich schlaglichtartig auf einige Arbeiten verwiesen. R.~K. \textcite{Eggert.1991} beschreibt ethnografische Beobachtungen des Keramikhandels im Inneren Kongobecken, während der kurze Beitrag von \textcite{VanderLinden.1996} den marktbasierten Keramikhandel in Nordkamerun umreisst.} Ausschließlich im nördlichen Teil des Arbeitsgebietes finden sich die rezenten Stilgruppen Kpetene (Kap.~\ref{sec:KPT-Gr}) und Bangui (Kap.~\ref{sec:BAN-Gr}). Beide Stilgruppen sind lediglich durch wenige Einzelfunde belegt und stellen einen geringen Anteil an den jeweiligen Inventaren. Die ebenfalls Schnitzroulette verwendende Kpetene-Keramik findet sich ausschließlich im Bereich des \mbox{Ubangi}-Bogens, stromauf von Bangui. Die Gefäße unterscheiden sich von der im gleiche Gebiet verbreiteten Dama-Keramik durch eine aus mehreren Schnitzroulette-Bändern bestehende Verzierung. Die Bangui-Keramik spiegelt hingegen eine rezente Töpferei wider, die keine Rouletteverzierung verwendet und lediglich in einem kleinen Gebiet stromab von Bangui verbreitet ist.

Der südlichste Teil des \mbox{Ubangi} zeichnet sich ausschließlich durch keramische Formen aus, die entweder starke Ähnlichkeiten zu Stilen des Inneren Kongobeckens aufweisen, wie die Bokwango-Gruppe (Kap.~\ref{sec:BKW-Gr}), oder sogar eindeutig als Teil entsprechender Stilgruppen angesprochen werden konnten. Bei Letzteren handelt es sich um Vertreter der Stile Bondongo \parencite[Kap.~\ref{sec:BDG-Gr};][128--139]{Wotzka.1995}, Mbandaka (Kap.~\ref{sec:MBA-Gr}; ebd. 139--143) sowie Botendo (Kap.\ref{sec:BOT-Gr}; ebd. 150--158). Nördlich von \mbox{Ngbanja} fanden sich keine Vertreter dieser vier Gruppen. Auffällig ist, dass all diese Gruppen in die Jüngere Eisenzeit datieren und daher die früheste Besiedlung dieses etwa 250\,km langen Abschnitts des \mbox{Ubangi}, zwischen seiner Mündung in den Kongo im Süden und \mbox{Ngbanja} (Fpl.~199) im Norden, erst ab dem \mbox{10.--11.~Jh.} n.~Chr. postuliert werden kann.

Zusammenfassend können drei Regionen entlang des \mbox{Ubangi}- und Lua unterschieden werden, in denen die Besiedlungsgeschichte unterschiedlich ablief: Die erste Zone bilden dabei die südlichsten etwa 250--350 Kilometer zwischen der Mündung des \mbox{Ubangi} und \mbox{Ngbanja} (Fpl.~199) beziehungsweise Dongo (Fpl.~202). Hier finden sich praktisch keine Zeugnisse der frühen Eisenzeit und die Inventare sind von Vertretern keramischer Stilgruppen bestimmt, die hauptsächlich aus dem Inneren Kongobecken stammen oder starke Ähnlichkeiten zu diesen zeigen und die allesamt in die Jüngere Eisenzeit datieren, folglich jünger als etwa das 10.~Jh. n.~Chr. sind. Diese Zone lässt sich als Teil der von \textsc{Wotzka} (ebd. 222 Abb.~4, 273, 283) herausgearbeiteten \textit{Äquator-Co}-Tradition des Inneren Kongobeckens begreifen (Kap.~\ref{sec:ICB_StilGrDatierungen}). Die zweite Zone reicht von \mbox{Ngbanja} beziehungsweise Dongo bis zum \mbox{Ubangi}-Bogen und umfasst auch den befahrenen Abschnitt des Lua. Diese zweite Zone, die etwa 200--300\,km in Nord--Süd-Richtung abdeckt, zeigt Zeugnisse einer Besiedlung durch keramikproduzierende Gruppen, die mit den frühesten entsprechenden Zeugnissen im Arbeitsgebiet in Form der Batalimo-Maluba-Keramik beginnt und sich bis in die rezente Töpferei nachzeichnen lässt. Diese Abfolge keramischer Stilgruppen ist in keinem Fall zusammenhängend. Vielmehr ergibt sich bei Betrachtung der vorliegenden Datierungsansätze eine etwa 300 Jahre lange Lücke (siehe Abb.~\ref{fig:Chronologiesystem}; Kap.~\ref{sec:Zeitscheiben}), die beim gegenwärtigen Quellenstand nicht hinreichend geschlossen werden kann. Relativchronologische Bezüge zwischen den Stilgruppen der Älteren Eisenzeit zu jenen der Jüngeren Eisenzeit lassen sich lediglich auf Merkmalsebene beobachten, so dass offenbleiben muss, ob sich diese in Form einer Keramiktradition beschreiben lassen (Kap.~\ref{sec:Horizonte}). Die dritte und nördlichste Zone ist der stromauf seines Bogens prospektierte Flussabschnitt des \mbox{Ubangi}. Diese etwa 200\,km lange Flussstrecke erbrachte ausschließlich Inventare, die Zeugnisse der rezenten, regionalen Töpfereitraditon widerspiegeln. Inwieweit ein ältere keramischer Formenschatz vorhanden ist, jedoch nicht entdeckt wurde, lässt sich derzeit nicht entscheiden.

\begin{table*}[!tb]
	\centering
	{\footnotesize 
		\begin{sftabular}{@{}rlcccccccccc@{}}
\toprule
\textbf{Nr.} & \textbf{Fundort} & \textbf{PKM} & \textbf{NGO} & \textbf{MAT} & \textbf{EBA} & \textbf{EPE} & \textbf{MKA} & \textbf{KON} & \textbf{BBS} & \textbf{BDG} & \textbf{BOT} \\
\midrule
 306 & Epena & $\circ $ & & & & $\bullet $ & & & & & \\
 305 & Itanga & $\circ $ & & & $\bullet $ & $\bullet $ & $\bullet $ & & & & \\
 304 & Munda & $\bullet $ & & & $\bullet $ & $\bullet $ & & $\circ $ & & & \\
 303 & Jeke & $\circ $ & $\circ $ & & $\bullet $ & $\bullet $ & $\bullet $ & & & & \\
 302 & Botongo & & & & $\circ $ & $\circ $ & & & & & \\
 301 & \begin{tabular}[c]{@{}l@{}}Likwala\\Fkm 401\end{tabular} & & & $\bullet $ & & & & & & & \\
 300 & Likunda & & & & $\bullet $ & $\bullet $ & & & & & \\
 299 & Mosenge & $\circ $ & & & $\bullet $ & $\bullet $ & & & & & \\
 298 & Bwanela & $\circ $ & & & & $\circ $ & $\circ $ & & & & \\
 297 & Ebambe & $\circ $ & & & $\bullet $ & $\bullet $ & $\bullet $ & & & $\circ $ & \\
 296 & Mosengi & & & & $\bullet $ & $\bullet $ & & & & & \\
 295 & Bokoma & & $\circ $ & & & $\bullet $ & $\bullet $ & & & & \\
 294 & Boenja & & & & & $\bullet $ & & & & & \\
 292 & Bojenjo & $\circ $ & $\circ $ & & $\bullet $ & $\bullet $ & $\bullet $ & & & & $\circ $ \\
 290 & Lokolélé & $\circ $ & & & $\bullet $ & $\bullet $ & $\circ $ & & & & \\
 289 & Yumba & $\circ $ & & $\circ $ & $\bullet $ & $\bullet $ & $\bullet $ & & & & $\bullet $ \\
 288 & Misongo & $\circ $ & & & & $\circ $ & $\bullet $ & & & & $\bullet $ \\
 287 & Ilebo & & & & & $\bullet $ & $\circ $ & & & & \\
 286 & Botwale & & & & $\circ $ & $\bullet $ & $\bullet $ & & & & $\bullet $ \\
 285 & Boleko & $\circ $ & $\bullet $ & & & $\bullet $ & $\bullet $ & & $\bullet $ & & $\bullet $ \\
 284 & Boyenge & & $\circ $ & & $\bullet $ & & & & & & $\bullet $ \\
 283 & Ngombe\textsuperscript{1} & & & & $\bullet $ & $\bullet $ & $\bullet $ & & & & $\bullet $ \\
\bottomrule
\end{sftabular}

	}
	\caption{Likwala-aux-Herbes-Gebiet: Nachgewiesene keramische Stilgruppen (siehe Anlage~3) nach Fundorten (von Nord nach Süd).\\$\bullet$ belegt, $\circ$ fraglich; 1 Vermischt mit Inventar aus Ngombe (Fpl.~252; siehe Tab.~\ref{tab:SanghaNgokoSequenz}).}
	\label{tab:LikwalaSequenz}
\end{table*}

\subsection{Likwala-aux-Herbes-Gebiet}\label{sec:SequenzLikwala}

Die knapp über 500\,km lange Befahrung des \mbox{Likwala}-\mbox{aux}-\mbox{Herbes} (Tab.~\ref{tab:ArbeitsgebietFlussstrecken}) erbrachte einen Einblick in eine sehr eigenständige, regionale, keramische Entwicklung (Tab.~\ref{tab:LikwalaSequenz}). Die Inventare entlang dieses Flusses zeichnen sich vor allem durch Vertreter der drei subrezenten bis rezenten Stilgruppen Ebambe (Kap.~\ref{sec:EBA-Gr}), Epena (Kap.~\ref{sec:EPE-Gr}) sowie Mobaka (Kap.~\ref{sec:MKA-Gr}) aus. Alle drei Stilgruppen sind entlang des gesamten 1987 prospektierten Abschnitts des \mbox{Likwala}-\mbox{aux}-\mbox{Herbes} von seiner Mündung in den \mbox{Sangha} bei Ngombe (Fpl.~283) im Süden bis nach Epena im Norden (Fpl.~306) vertreten.

Zweifelsfreie Zeugnisse der frühesten Besiedlung des südlichen Abschnitts des Arbeitsgebietes, welche durch die Pikunda-Munda-Gruppe gekennzeichnet ist (Kap.~\ref{sec:PKM-Gr}), finden sich entlang des \mbox{Likwala}-\mbox{aux}-\mbox{Herbes} lediglich in Munda am oberen \mbox{Likwala}-\mbox{aux}-\mbox{Herbes} (Fpl.~304). Im Oberflächenmaterial konnte häufig keine hinreichend sichere Ansprache vorgenommen werden. Dieser Umstand lässt sich auch auf eine Eigenheit der keramischen Inventare entlang des \mbox{Likwala}-\mbox{aux}-\mbox{Herbes} zurückführen: die hauptsächlich anzutreffenden, genannten Stilgruppen zeichnen sich allesamt das gleiche \textit{Fabric} aus (Tab.~\ref{tab:Fabrics_StilGr_Pct}, Abb.~\ref{fig:Fabrics_Verbreitung}). Eine Unterscheidung einzelner Stile ist somit nur unter Einbeziehung morphologischer wie ornamentaler Charakteristika möglich. Die Verzierung der Pikunda-Munda-Gruppe ist von Rillen- und Riefenzier geprägt. Auch die subrezenten bis rezenten Stile Ebambe und Epena zeigen noch vielfältige Rillen- und Riefenverzierung, so dass bei stark fragmentiertem Material häufig keine sichere Abgrenzung möglich war. In der Folge konnten entlang des \mbox{Likwala}-\mbox{aux}-\mbox{Herbes} an vielen Fundstellen Stücke lediglich nur unter Vorbehalt der Pikunda-Munda-Gruppe zugewiesen werden (Abb.~\ref{fig:PIKMUN_Verbreitung}). Entsprechende Formen, also nur bedingt der Pikunda-Munda-Gruppe zuweisbare Stücke, finden sich entlang des gesamten \mbox{Likwala}-\mbox{aux}-\mbox{Herbes}. Eine Besiedlung des \mbox{Likwala}-\mbox{aux}-\mbox{Herbes}-Gebietes durch die Träger der Pikunda-Munda-Keramik erscheint beim gegenwärtigen Quellenstand durchaus als wahrscheinlich. 

Im Süden des \mbox{Likwala}-\mbox{aux}-\mbox{Herbes} finden sich des Weiteren Vertreter der aus dem Inneren Kongobecken bekannten Botendo-Stils (Kap.\ref{sec:BOT-Gr}; ebd. 150--158). Botendo-Keramik ist auf einer Strecke von etwa 60\,km bis nach Yumba (Fpl.~289) verbreitet. Dieser Keramikstil schließt den südlichen Teil des \mbox{Likwala}-\mbox{aux}-\mbox{Herbes} direkt an die jüngste keramische Entwicklung des Inneren Kongobeckens an. Wie auch die lokalen Töpfereierzeugnisse, zeichnet sich die Keramik der Botendo-Gruppe ausschließlich durch die \textit{Fabrics} 1 sowie 2 aus. Dies weist auf starke technologische Ähnlichkeiten zur \textit{Äquator-Co}-Tradition des Inneren Kongobeckens hin (siehe Abb.~\ref{fig:Fabrics_Verbreitung}).\footnote{Eine Assoziation der Keramik des Inneren Kongobeckens mit den \textit{Fabrics} 1 und 2 ist bislang nicht empirisch belegt, lässt sich durch Aussagen Wotzkas (mündl. Mitt. 2015) sowie eigene Beobachtungen jedoch postulieren.}

Auch die regionale Sequenz des \mbox{Likwala}-\mbox{aux}-\mbox{Herbes} offenbart eine vor dem aktuellen Quellenstand nicht zu überbrückende Lücke zwischen den in die Ältere Eisenzeit datierenden, frühesten keramischen Erzeugnissen im Arbeitsgebiet -- der Pikunda-Munda-Gruppe -- sowie den subrezenten bis rezenten keramischen Ausprägungen (Abb.~\ref{fig:Chronologiesystem}). An verschiedenen Fundstellen fanden sich zudem potenzielle Vertreter weiterer keramischer Stilgruppen, die vor allem an benachbarten Flüssen wie dem \mbox{Sangha} verbreitet sind: gemeint sind Vertreter der Stilgruppen Ngombe (Kap.~\ref{sec:NGO-Gr}), Matoto (Kap.~\ref{sec:MAT-Gr}), Konda (Kap.~\ref{sec:KON-Gr}), Bobusa (Kap.~\ref{sec:BBS-Gr}) sowie Bondongo (Kap.~\ref{sec:BDG-Gr}; ebd. 128--139). Mit Ausnahme einiger Stücke der Ngombe-Gruppe sind all diese Stilgruppen lediglich durch Einzelfunde am \mbox{Likwala}-\mbox{aux}-\mbox{Herbes} belegt. Einen signifikanten Einfluss auf die Stilgruppensequenz und die sich daraus ergebenden Rückschlüsse auf den Besiedlungsgang haben sie nicht.


\subsection{\mbox{Sangha}- und \mbox{Ngoko}-Gebiet}\label{sec:SequenzSanghaNgoko}

Die Auflistung der belegten Stilgruppen entlang des \mbox{Sangha}- und \mbox{Ngoko} (Tab.~\ref{tab:SanghaNgokoSequenz}) zeichnet -- ähnlich der Situation entlang des \mbox{Ubangi} -- eine vielschichtige und regional differenzierte Abfolge keramischer Stile nach.\footnote{Eine umfangreiche Synthese der Quellenlage zu Paläo-Umwelt, Linguistik und Archäologie mit besonderer Berücksichtigung der Region des \mbox{Sangha} streben \textcite{Bostoen.2015} an. Die von den Autoren (ebd.~366) vorgetragene Hypothese, dass der \mbox{Sangha} ein Korridor für die Erstbesiedlung des Kongobeckens -- deren Träger zugleich direkt als Bantu-Sprecher identifiziert werden -- gewesen sei, entbehrt allerdings jeder archäologischen Basis. Die Argumentation beruht lediglich auf einer zeitlichen Korrelation zwischen dem frühesten Auftreten von Keramik im Inneren Kongobecken in der zweiten Hälfte des 1.~Jt. v.~Chr. (Kap.~\ref{sec:ICB_StilGrDatierungen}) und der Regenwaldkrise des 1.~Jt. v.~Chr. (Kap.~\ref{sec:Palaeoumwelt}). Jedwedes Postulat einer Kausalität kann gegenwärtig nur als Spekulation gewertet werden.} Die knapp 600\,km lange Befahrung und Prospektion entlang des \mbox{Sangha} von seiner Mündung in den Kongo bei Bonga (Fpl.~234) bis nach Bomasa am Dreiländereck der Republik Kongo, Kameruns und der Zentralafrikanischen Republik liefert einen weiteren Nord--Süd-Transekt durch den äquatorialen Regenwald. Dieser reicht im Süden bis kurz vor die südliche Grenze des immergrünen Regenwaldes, der etwa 50\,km stromab am Kongo allmählich in eine Baumsavanne übergeht (siehe Abb.~\ref{fig:PalaeoumweltArch_Karte}). 

\begin{table*}[!tb]
	\centering
	\resizebox{1.02\textwidth}{!}{%
		{\footnotesize 
			\begin{sftabular}{@{}rlcccccccccccccccccc@{}}
\toprule
\textbf{Nr.} & \textbf{Fundort} & \textbf{NGB} & \textbf{PKM} & \textbf{BOG} & \textbf{NGO} & \textbf{MAT} & \textbf{EBA} & \textbf{EPE} & \textbf{MKA} & \textbf{MDB} & \textbf{KON} & \textbf{OUE} & \textbf{PDM} & \textbf{MBJ} & \textbf{BBS} & \textbf{IMB} & \textbf{BDG} & \textbf{MBA} & \textbf{BOT} \\
\midrule
 281 & Ngama & & & & & & & & & & $\bullet $ & & $\bullet $ & $\bullet $ & & & & & \\
 280 & Ponga & & & & & & & & & & & & $\bullet $ & & & & & & \\
 279 & Bonga & & & & & & & & & & & & $\bullet $ & $\bullet $ & & & & & \\
 278 & Ngwangala & & & & & & & & & & & & & $\bullet $ & & & & & \\
 277 & Mbenja & & & & & & $\circ $ & & & & $\bullet $ & & $\bullet $ & $\bullet $ & & & & & \\
 276 & Pandama & $\bullet $ & & & & & & & & $\circ $ & $\bullet $ & $\bullet $ & $\bullet $ & $\bullet $ & & & & & \\
 275 & \begin{tabular}[c]{@{}l@{}}\mbox{Ngoko}\\Fkm 17*\end{tabular} & & & & & & & & & $\bullet $ & $\bullet $ & & $\bullet $ & $\bullet $ & & & & & \\ \hdashline[0.5pt/5pt]
 274 & Bomasa & & & & & & & & & & & $\circ $ & & & & & & & \\
 273 & Sakao & & & & & & & & & & & & & $\circ $ & & & & & \\
 272 & Bonda & & & & & & & & & $\circ $ & & $\circ $ & & $\bullet $ & & & & & \\
 271 & Mai mpembe & & & & & & $\circ $ & & & $\circ $ & $\circ $ & & $\circ $ & $\bullet $ & & & & & \\
 270 & Gbagbale & & & & & & & & & & & $\circ $ & $\bullet $ & & & & & & \\
 269 & Leme & & & & & & & $\bullet $ & & & $\bullet $ & & $\circ $ & & & & & & \\
 268 & Konda & & & & & & & & & $\bullet $ & $\bullet $ & & $\bullet $ & & & & & & \\
 267 & Maboko & & & & & & $\bullet $ & & & $\circ $ & $\bullet $ & & $\bullet $ & $\bullet $ & & & & & \\
 266 & Gatongo & & & & & & & & & & $\circ $ & & $\circ $ & & & & & & \\
 265 & Ouesso & $\bullet $ & & & & & $\bullet $ & & & $\circ $ & $\bullet $ & $\bullet $ & $\bullet $ & $\bullet $ & $\circ $ & & & & \\
 & Mbou Mboua & & $\circ $ & & & & & & & $\circ $ & & $\circ $ & $\circ $ & $\circ $ & & & & & \\
 264 & Matoto & $\bullet $ & & & & $\bullet $ & $\bullet $ & & $\circ $ & $\circ $ & $\bullet $ & $\circ $ & $\bullet $ & & & & & & \\
 263 & \begin{tabular}[c]{@{}l@{}}\mbox{Sangha}\\Fkm 428\end{tabular} & & & & & $\bullet $ & & & & $\circ $ & $\bullet $ & & $\circ $ & & & & & & \\
 262 & Mosanya & $\circ $ & $\bullet $ & & & & $\circ $ & & & & $\bullet $ & & $\circ $ & $\bullet $ & & & & & \\
 261 & Motoli & & & & & $\bullet $ & & & & $\circ $ & $\circ $ & & $\circ $ & & & & & & \\
 260 & Ikelemba & $\circ $ & $\bullet $ & & & $\bullet $ & & & $\circ $ & & $\circ $ & & $\circ $ & $\bullet $ & $\circ $ & & & & \\
 259 & Mandombe & & $\circ $ & & $\circ $ & $\bullet $ & $\circ $ & & & $\bullet $ & $\bullet $ & & $\bullet $ & $\bullet $ & $\circ $ & & & & \\
 258 & Molanda & & $\circ $ & & & $\bullet $ & $\circ $ & & & $\bullet $ & $\bullet $ & & $\bullet $ & $\bullet $ & & & & & \\
 256 & Itandi* & & $\bullet $ & & & & & & & $\bullet $ & & & $\circ $ & & & & & & \\
 255 & Pikunda & $\bullet $ & $\bullet $ & $\bullet $ & & $\circ $ & $\bullet $ & $\circ $ & & $\bullet $ & $\circ $ & $\bullet $ & $\bullet $ & $\circ $ & & & $\circ $ & & $\circ $ \\
 253 & Ifondo & & $\circ $ & & & & $\circ $ & & & & & & $\circ $ & & & $\circ $ & & & \\
 252 & Ngombe\textsuperscript{1} & & & $\bullet $ & $\bullet $ & $\circ $ & & & & & $\circ $ & & $\circ $ & & & & & & \\
 251 & Mitula & & $\bullet $ & & & & & $\circ $ & & & & & & & $\circ $ & $\bullet $ & $\circ $ & & $\circ $ \\
 250 & Bokonongo & & $\circ $ & $\bullet $ & $\bullet $ & & $\bullet $ & & & & & & & & $\circ $ & $\circ $ & & & \\
 249 & Inyenge* & & & & $\bullet $ & & & & & & & & & & & & & & \\
 248 & Loboko & & & $\bullet $ & $\circ $ & & $\bullet $ & $\circ $ & $\bullet $ & & & & & & $\circ $ & & & & $\bullet $ \\
 247 & Likaya & & $\bullet $ & & & & & & & & & & & & & & & & \\
 246 & Mobaka & & & & & & & & $\bullet $ & & & & & & & $\bullet $ & & & \\
 245 & \begin{tabular}[c]{@{}l@{}}Bondo-\\Mission\end{tabular} & & & & & & & & $\bullet $ & & & & & & & & & & \\
 244 & \begin{tabular}[c]{@{}l@{}}\mbox{Sangha}\\Fkm 85\end{tabular} & & & & & & & $\circ $ & & & & & & & & & & & \\
 243 & Monjolomba & & $\circ $ & & $\bullet $ & & $\bullet $ & $\bullet $ & $\bullet $ & & & & & & $\circ $ & & & & \\
 242 & \begin{tabular}[c]{@{}l@{}}\mbox{Sangha}\\Fkm 72\end{tabular} & & & & & & & $\bullet $ & & & & & & & & & & & \\
 241 & Sosolo & & $\circ $ & $\bullet $ & $\bullet $ & & $\circ $ & $\circ $ & $\bullet $ & & & & & & $\bullet $ & & $\circ $ & $\circ $ & $\bullet $ \\
 240 & \begin{tabular}[c]{@{}l@{}}\mbox{Sangha}\\Fkm 40\end{tabular} & & & $\bullet $ & & & $\bullet $ & & & & & & & & & & & & \\
 239 & Bobusa & & $\circ $ & & & & & $\bullet $ & & & & & & & $\bullet $ & & $\circ $ & $\bullet $ & $\bullet $ \\
 238 & Bonga* & & & & & & & $\bullet $ & & & & & & & $\bullet $ & & $\bullet $ & & $\bullet $ \\ \hdashline[0.5pt/5pt]
 237 & Gombe & & & & & & $\circ $ & $\bullet $ & $\circ $ & & & & & & $\circ $ & $\bullet $ & & $\bullet $ & $\bullet $ \\
 236 & Sungu & & $\bullet $ & $\bullet $ & & & & $\bullet $ & & & & & & & $\bullet $ & & $\circ $ & $\bullet $ & $\bullet $ \\
 235 & Maberu & & $\circ $ & $\bullet $ & $\circ $ & & & $\circ $ & & & & & & & $\circ $ & & $\circ $ & $\bullet $ & \\
 234 & Lukolela & & & & & & & & & & & & & & $\bullet $ & & $\circ $ & $\bullet $ & $\circ $ \\
\bottomrule& 
\end{sftabular}

		}
	}
	\caption{\mbox{Sangha}- und \mbox{Ngoko}-Gebiet: Nachgewiesene keramische Stilgruppen (siehe Anlage~3) nach Fundorten (von Nord nach Süd). Zusätzlich sind noch die Fundstellen vom befahrenen Abschnitt des Kongo-Stroms aufgetragen.\\$\bullet$ belegt, $\circ$ fraglich;~* Dorfwüstungen (\textit{bilali}; Sing. \textit{elali}); 1 vermischt mit Inventar aus Ngombe (Fpl.~283; siehe Tab.~\ref{tab:LikwalaSequenz}); Fpl. Mboua Mboua siehe \textcite[114 Abb.~42]{Gillet.2013}.}
	\label{tab:SanghaNgokoSequenz}
\end{table*}

Die ältesten keramischen Zeugnisse stammen vom südlichen Teil des \mbox{Sangha}; es handelt sich um wenige GE, die der Imbonga-Gruppe des Inneren Kongobeckens zuzurechnen sind \parencite[Kap.~\ref{sec:IMB-Gr};][59--68]{Wotzka.1995}. Zeugnisse dieses ältesten, in das 4.--1.~Jh. v.~Chr. zu datierenden Keramikstils (Kap.~\ref{sec:ICB_StilGrDatierungen}), der die früheste bekannte Besiedlung durch keramikproduzierende Gruppen im Kongobecken repräsentiert, nehmen im Fundgut aus dem nordwestlichen Kongobecken lediglich einen marginalen Anteil ein. Wirklich diagnostische Stücke fanden sich lediglich an zwei Fundstellen: in Mobaka (Fpl.~246; Abb.~\ref{fig:IMB_Typentafel}.5) sowie Mitula (Fpl.~251; Abb.~\ref{fig:IMB_Typentafel}.3). Das potenzielle Verbreitungsgebiet der Imbonga-Gruppe im nordwestlichen Kongobecken und die Intensität von mit dieser Keramik in Zusammenhang stehenden Gemeinschaften können vor dem gegenwärtigen Quellenstand nur unzureichend nachvollzogen werden.\footnote{Zukünftige Feldforschungen entlang des unteren \mbox{Sangha} könnten eventuell weitere Befunde und Funde erbringen. Die 1987 durch das \textit{River Reconnaissance Project} durchgeführte Prospektion lieferte sicher nur einen ersten Einblick. Eine in den späten 2000er Jahren durchgeführte Prospektion am \mbox{Sangha} beschränkte sich auf den oberen und mittleren Abschnitt des Flusses und gelangte stromab lediglich bis etwa Pikunda \parencite[Fpl.~255; siehe][211 Abb.~1]{MorinRivat.2014}.} Die älteste in hinreichendem Umfang erfasste Keramik entlang des \mbox{Sangha} -- wie weiter östlich entlang des  \mbox{Likwala}-\mbox{aux}-\mbox{Herbes} -- ist dem Pikunda-Munda-Stil (Kap.~\ref{sec:PKM-Gr}) zuzurechnen. Vertreter der Stilgruppe finden sich am 1987 prospektierten Abschnitt des Kongo-Stroms sowie vor allem entlang des mittleren \mbox{Sangha}, zwischen Likaya (Fpl.~247) im Süden und Mosanya (Fpl.~262) im Norden. Zwischen Likaya und der Mündung des \mbox{Sangha} in den Kongo finden sich immer wieder vereinzelte Stücke, die nur unter Vorbehalt der Pikunda-Munda-Gruppe zurechenbar sind. In Zusammenschau mit den Erkenntnissen vom \mbox{Likwala}-\mbox{aux}-\mbox{Herbes} fällt auf, dass der nördliche Teil des Verbreitungsgebietes der Pikunda-Munda-Keramik die dichteste Belegung zeigt (Abb.~\ref{fig:PIKMUN_Verbreitung}). Neben den hier bearbeiten Inventaren liegen potenzielle Belege für Pikunda-Munda-Keramik von vier Plätzen vor, die im Rahmen paläo-ökologischer Untersuchungen im Norden der Republik Kongo erschlossen wurden \parencite[Abb.~\ref{fig:PIKMUN_Verbreitung};][114 Abb.~42]{Gillet.2013}.\footnote{Die Fundstellen zeichnen sich neben den keramischen Funden durch bei Bohrungen erfasste Holzkohlekonzentrationen aus (\textsc{Gillet} 2013: 94 Tab.~14). Da meine Ansprache lediglich anhand von veröffentlichten Fotografien erfolgte, steht meine Zuweisung unter starkem Vorbehalt.} Akzeptierte man diese Fundpunkte, so würde sich das Verbreitungsgebiet der Pikunda-Munda-Gruppe um etwa 50\,km nach Norden ausweiten, auf eine Ausdehnung von etwa 300\,km in Nord--Süd-Richtung.

Potenziell ähnlich alte keramische Inventare finden sich in Form der Stilgruppe Bokonongo (Kap.~\ref{sec:BOG-Gr}), die in einem sich mit der Pikunda-Munda-Keramik überlagernden bis südlich angrenzenden Areal verbreitet ist. Weiter nördlich an die Pikunda-Munda-Keramik angrenzend sowie entlang des mittleren \mbox{Ubangi} (Tab.~\ref{tab:UbangiLuaSequenz}) finden sich keramische Ausprägungen, die unter der Bezeichnung \mbox{Ngbanja} (Kap.~\ref{sec:NGB-Gr}) subsumiert wurden. Deren Vertreter zeichnen sich vor allem durch ein unterschiedliches Spektrum an \textit{Fabrics} aus: Während die bislang genannten Gruppen entlang des \mbox{Sangha} sämtlich die \textit{Fabrics} 1 sowie 2 aufweisen, wird die \mbox{Ngbanja}-Keramik durch die nicht-plastische Partikel enthaltenden \textit{Fabrics} 4 und 5 bestimmt.

Entlang des \mbox{Sangha} lassen sich keine direkt an die durch die Pikunda-Munda-Gruppe geprägte und bis etwa in das 5.~Jh. n.~Chr. reichende Ältere Eisenzeit anschließende keramischen Phänomene beobachten (Abb.~\ref{fig:Chronologiesystem}). Spuren einer erneuten Aktivität lassen sich erst etwa 500 Jahre später, mit den Stilgruppen Ngombe (Kap.~\ref{sec:NGO-Gr}) sowie Matoto (Kap.~\ref{sec:MAT-Gr}) fassen. Die Ngombe-Keramik weist starke Ähnlichkeiten zur Longa-Keramik des Inneren Kongobeckens auf und findet sich vornehmlich entlang des Unterlaufs des \mbox{Sangha} (Tab.~\ref{tab:SanghaNgokoSequenz}) sowie des weiter östlichen \mbox{Likwala}-\mbox{aux}-\mbox{Herbes} (Tab.~\ref{tab:LikwalaSequenz}). Die Verbreitung der Ngombe-Gruppe unterstreicht die Anbindung des Unterlaufs des \mbox{Sangha}- und \mbox{Likwala}-\mbox{aux}-\mbox{Herbes} an das Innere Kongobecken ab der Jüngeren Eisenzeit beziehungsweise dem 11./12.~Jh. n.~Chr. Auch Material der Stilgruppen Bondongo (Kap.~\ref{sec:BDG-Gr}) und Mbandaka \parencite[Kap.~\ref{sec:MBA-Gr}; ][139--143]{Wotzka.1995} ist im südlichen Teil des \mbox{Sangha} sowie entlang des Kongo zwischen den Mündungen des \mbox{Sangha} und \mbox{Ubangi} verbreitet. Die ebenfalls an den Beginn der Jüngeren Eisenzeit datierende Matoto-Gruppe, die sich auch am \mbox{Ubangi} sowie \mbox{Likwala}-\mbox{aux}-\mbox{Herbes} findet, hat ihr Hauptverbreitungsgebiet am mittleren \mbox{Sangha}; stromauf des Verbreitungsgebietes der Ngombe-Gruppe (Tab.~\ref{tab:SanghaNgokoSequenz}).

Erst mit dem Aufkommen der Mandombe-Keramik im 13.--15.~Jh. n.~Chr. (Kap.~\ref{sec:MDB-Gr}) lassen sich keramische Formen einer systematischen Besiedlung des Oberlaufs des \mbox{Sangha} zuweisen. Funde dieser vor allem entlang des oberen bis mittleren \mbox{Sangha} verbreiteten Keramik finden sich am \mbox{Ngoko} nur äußerst spärlich. Erst die Stile Konda (Kap.~\ref{sec:KON-Gr}) und Pandama (Kap.~\ref{sec:PDM-Gr}) sind in den Inventaren entlang des \mbox{Ngoko} hinreichend vertreten. Die Stile Mandombe, Konda und Pandama sowie die Vertreter der Ouesso-Gruppe (Kap.~\ref{sec:OUE-Gr}) und die rezente Mbenja-Keramik (Kap.~\ref{sec:MBJ-Gr}) zeichnen einen eigenständigen keramischen Entwicklungsstrang nach, der sich aus keiner der vorangegangen Stilgruppen des Arbeitsgebiets ableiten lässt und als \textit{\mbox{Ngoko}-Tradition} (Kap.~\ref{sec:NgokoTradition}) systematisiert wurde. Mit den Stilgruppen der \textit{\mbox{Ngoko}-Tradition} ist die sukzessive Übernahme von Rouletteverzierung verknüpft (siehe Kap.~\ref{sec:Zeitscheiben}). Auffällig ist, dass keine der Stilgruppen der \textit{\mbox{Ngoko}-Tradition} weiter stromab als Pikunda (Fpl.~255) verbreitet ist.

Die jüngere Entwicklung im weiter stromab gelegenen südlichen Teil des \mbox{Sangha} ist von Vertretern der vor allem entlang des Likwala-aux-Herbes (Kap.~\ref{sec:SequenzLikwala}) verbreiteten Stilgruppen Ebambe (Kap.~\ref{sec:EBA-Gr}) und Epena (Kap.~\ref{sec:EPE-Gr}) gekennzeichnet. Während sich vereinzelte Vertreter der Ebambe-Gruppe entlang fast des gesamten \mbox{Sangha} finden, beschränkt sich die Epena-Keramik, mit Ausnahme einer GE in Leme am oberen \mbox{Sangha} (Fpl.~269, Taf.~61.16), auf den Bereich zwischen der Mündung des \mbox{Sangha} in den Kongo bei Bonga (Fpl.~238) und der Einmündung des \mbox{Likwala}-\mbox{aux}-\mbox{Herbes} in den \mbox{Sangha} bei Monjolomba (Fpl.~243). Auch die ebenfalls entlang des gesamten \mbox{Likwala}-\mbox{aux}-\mbox{Herbes} verbreitete rezente Mobaka-Keramik (Kap.~\ref{sec:MKA-Gr}) findet sich im südlichen Abschnitt des \mbox{Sangha}. Diese drei Stilgruppen weisen mit ihrem Fokus auf die Region des Likwala-aux-Herbes sehr ähnliche Verbreitungsgebiete auf (Abb.~\ref{fig:EBA_Verbreitung}, \ref{fig:EPE_Verbreitung}, \ref{fig:MKA_Verbreitung}). Lediglich in der Frage, wie weit stromauf entlang des \mbox{Sangha} sich Zeugnisse der jeweiligen Stile finden, ergeben sich Unterschiede, wobei die Mobaka-Keramik etwas weiter nördlich als die Epena-Keramik zu finden ist und das Verbreitungsgebiet der Ebambe-Keramik noch einmal etwas weiter nach Norden beziehungsweise stromauf reicht. Ein sehr ähnliches Verbreitungsgebiet entlang des \mbox{Sangha} zeigt die aus dem Inneren Kongobecken bekannte, subrezente Botendo-Keramik (Kap.\ref{sec:BOT-Gr}; ebd. 150--158).

Im äußersten Süden des Arbeitsgebietes, im Mündungsbereich des \mbox{Sangha} sowie am 1987 untersuchten Abschnitt des Kongo findet sich überdies die technologisch mit keiner anderen keramischen Ausprägung des Arbeitsgebietes in Zusammenhang zu bringende Bobusa-Keramik (Kap.~\ref{sec:BBS-Gr}). Diese zeichnet sich -- als einzige Stilgruppe des Arbeitsgebietes -- durch eine systematische Verwendung von Schamott-Magerung beziehungsweise dadurch charakterisierte \textit{Fabric} 9 aus (Tab.~\ref{tab:Fabrics_StilGr_Pct}). Nur einige lediglich unter Vorbehalt der Bobusa-Gruppe zuweisbare Stücke finden sich entlang des \mbox{Sangha} weiter stromauf.\footnote{Es mag vor dem gegenwärtigen Quellenstand als Hypothese ausreichen, dass die Begrenzung der Verbreitung der Bobusa-Keramik im Süden wohl ausschließlich durch die Grenze der Feldaktivitäten des \textit{River Reconnaissance Project} bestimmt ist und es als wahrscheinlich gelten kann, dass zukünftige südlich anschließende Feldarbeiten weitere Funde dieser Stilgruppe aufdecken würden.}