\section{Stiltraditionen und Stilhorizonte des nordwestlichen Kongobeckens}\label{sec:Horizonte}\label{sec:NgokoTradition}

Im Folgenden soll, in Anlehnung an das Vorgehen von \textsc{Wotzka} (ebd. 217--225), für die Keramik des Inneren Kongobeckens, der Versuch unternommen werden, die aus den Regionalsequenzen (Kap.~\ref{sec:Sequenzen}) ableitbaren Entwicklungslinien zu einem Gesamtbild der keramischen Entwicklung des Arbeitsgebietes weiterzuentwickeln. Da im Zusammenhang dieser Arbeit aber auch technologische Betrachtungsebenen untersucht wurden (Kap.~\ref{sec:Herstellung}) und in die weitere Analyse integriert werden (Kap.~\ref{sec:TechnologieTrad}), ergibt sich die Notwendigkeit, die hierfür bislang herangezogenen Konzepte \textit{Keramiktradition}\footnote{An dieser Stelle sei exemplarisch auf die Nutzung der Bezeichnung \textit{Keramiktradition} durch \textcites{Modderman.1970}{Modderman.1974}{Modderman.1982} hingewiesen, um die über ein großes Verbreitungsgebiet und lange Laufzeit reichende formale Stabilität der Limburg-Keramik des Rhein-Maas-Schelde-Mesolithikums zu beschreiben \parencite[nach][41]{Constantin.2010}.} sowie \textit{Keramikhorizont} differenzierter zu betrachten. Beide Konzepte wurden von \textcites[295]{Eggert.1983}[250, 257 Anm.~20]{Eggert.1984}[28\,f.]{Eggert.1988} und von \textcite[217--225]{Wotzka.1995} entwickelt und genutzt, um die keramischen Entwicklungslinien des Inneren Kongobeckens in ein Gesamtbild zu integrieren. In der von \textcite[285--295]{Eggert.1983} erarbeiteten ersten, groben Sequenz für das Innere Kongobecken werden \enquote{keramische Entwicklungslinien regionaler Prägung als \enquote{Traditionen}} konzipiert \parencite[217]{Wotzka.1995}. Bereits \textcite{Wotzka.1995} schlägt zur eindeutigen Unterscheidung der in Eggerts Sinne definierten Auslegung von Tradition die Bezeichnung \textit{Stiltradition} vor.\footnote{Eine mit Blick auf die Terminologie eingehende, hier nur in den nötigen Auszügen referierte Diskussion findet sich bei \textsc{Wotzka} (1995: 25\,f., 217--219). In dieser Arbeit wird nur insofern vom Verständnis und der damit implizierten Terminologie Wotzkas abgewichen, als die von ihm synonym zum Begriff \textit{Stiltradition} verwendete Bezeichnung \textit{Keramiktradition} keine Verwendung findet. Dies ist vor allem von Belang, um die durch technologische Beobachtungen erzielten Ergebnisse in separaten \textit{Technologietraditionen} (Kap.~\ref{sec:TechnologieTrad}) beschreiben zu können. Für alle auf stilistischen und formalen Aspekten ausgearbeiteten Traditionen wird daher hier der Begriff \textit{Stiltradition} verwendet.} Das Verständnis, nach dem einer Stiltradition die chronologische Reihung jeweils auseinander hervorgegangener keramischer Stile zu Grunde liegt, bildet auch die Grundlage für die Intergration der Stilentwicklung im Inneren Kongobecken durch \textcite[]{Wotzka.1995}. Dabei orientiert sich \textsc{Wotzka} (ebd. 217\,f.) weniger stark an dem durch \textcite[53]{Willey.1945} eingeführten Konzept der \textit{pottery tra"-di"-tion}, sondern an einer von Irving \textcite{Rouse.1957} vertretenen Auffassung. Dabei sieht er Traditionen ausschließlich als Einheiten höherer Ordnung an und ersetzt den ursprünglich von \textsc{Rouse} (ebd. 126--127 Abb. 1--2) benutzten Terminus \textit{Phase} durch die von ihm beschriebenen Stilgruppen beziehungsweise Keramikstile. Eine keramische Stiltradition kann postuliert werden, wenn eine \enquote{lineare Abfolge jeweils auseinander hervorgegangener und somit ein kulturelles Kontinuum mehr oder minder langer Zeitdauer  bildender keramischer Stile} beschrieben werden kann \parencite[218]{Wotzka.1995}. Die jeweiligen Stilgruppen müssen dabei jedoch nachweislich direkt aus einem oder mehreren Vorläuferstilen hervorgegangen sein und die \enquote{Genese nachfolgender Stile prägen} (ebd.). Zusätzlich geht auf \textcite{Rouse.1957} die Definition der Co-Tradition\footnote{Siehe hierzu auch \textcites[17--18]{Huffman.1970}{Huffman.2005}{Schmidt.1975}{Vogel.1978}{Hall.1983}.} zurück, die Verzweigungen einzelner Stiltraditionen beschreibt, bei der diese aus einem gemeinsamen Ursprungsstil oder aus mehr oder minder zeitgleichen Bestandteilen eines Stilhorizontes hervorgehen \parencite[218; siehe unten]{Wotzka.1995}. Stiltraditionen zeichnen folglich eine vertikale Gliederungsebene innerhalb eines regionalen bis überregionalen keramischen Entwicklungsganges nach, während die Co-Tradition  Verzweigungen beziehungsweise Verknüpfungen einzelner Traditionslinien abbildet.

Die im Arbeitsgebiet an allen drei großen Flussabschnitten zu beobachtenden Unterbrechungen zwischen den Stilen der Älteren und jenen der Jüngeren Eisenzeit (siehe Kap.~\ref{sec:Sequenzen}) machen die Beschreibung durchlaufender Stiltraditionen unmöglich. Die einzige klar fassbare Stiltradition des Arbeitsgebietes bildet sich in der Jüngeren Eisenzeit im Bereich des oberen \mbox{Sangha} sowie dem prospektierten Abschnitt des \mbox{Ngoko} aus und besteht aus den Stilgruppen Mandombe, Konda, Ouesso, Pandama sowie der rezenten Mbenja-Gruppe (Kap.~\ref{sec:MDB-Gr}--\ref{sec:MBJ-Gr}; siehe Tab.~\ref{tab:SanghaNgokoSequenz}). Die Stilgruppen dieser \textit{\mbox{Ngoko}-Tradition} teilen einen sich entwickelnden formalen -- und technischen (siehe Kap.~\ref{sec:TechnologieTrad}) -- Merkmalsschatz, in dem Gefäße mit mehr oder weniger stark geschweifter Wandung und ausbiegendem Rand vom Typ D1 dominieren (Abb.~\ref{fig:MDB_Typverteter}; \ref{fig:KON_Typvertreter}.2--7; \ref{fig:OUE_Typvertreter}; \ref{fig:PDM_Typvertreter}.5, 7; \ref{fig:MBJ_Typverteter}.3, 5, 7). Innerhalb der Stilgruppen Konda und Pandama wird das Spektrum morphologischer Formen um kleine rundbodigen Schälchen vom Typ I4 erweitert (Abb.~\ref{fig:KON_Typvertreter}.8--10; \ref{fig:PDM_Typvertreter}.8--11). Innerhalb der \textit{\mbox{Ngoko}-Tradition} lässt sich eine schrittweise Adaption und Integration von in Rou\-lette\-technik angebrachten Verzierungselementen beobachten. Auffällig ist, dass alle Gruppen der \textit{\mbox{Ngoko}-Tradition} ein mehr oder minder gleiches Verbreitungsgebiet zeigen (Abb.~\ref{fig:MDB_Verbreitung}; \ref{fig:KON_Verbreitung}; \ref{fig:OUE_Verbreitung}; \ref{fig:PDM_Verbreitung}; \ref{fig:MBJ_Verbreitung}; Tab.~\ref{sec:SequenzSanghaNgoko}). Die Verbreitung aller Gruppen ist stromauf des \mbox{Sangha} wie des \mbox{Ngoko} lediglich durch die Endpunkte der Befahrungen von 1987 begrenzt, während keine Gruppe weiter stromab als Pikunda (Fpl.~255) beobachtet ist.

Entlang des \mbox{Ubangi} weisen lediglich die jüngeren, \mbox{Roulette} verwendenden Stilgruppen deutliche Ähnlichkeiten untereinander auf. Die sukzessive Integration dieser Verzierungstechnik beginnt innerhalb der Stile Dongo, Bobulu und Mokelo (Kap.~\ref{sec:DON-Gr}--\ref{sec:MKL-Gr}), die potenziell noch in die ältere Phase der Jüngeren Eisenzeit datieren. Deutliche Ähnlichkeiten lassen sich bei den Randformen der Dongo-Gruppe und der in die nachfolgende mittlere Phase der Jüngeren Eisenzeit datierenden Motenge-Boma-Gruppe (Kap.~\ref{sec:MTB-Gr}) feststellen. Dies deutet eine Herausbildung des Motenge-Boma-Stils aus der Dongo-Keramik an. Zu den älteren Stilen Batalimo-Maluba und \mbox{Ngbanja} (Kap.~\ref{sec:BTM-Gr}--\ref{sec:NGB-Gr}) lassen sich nur schwächere Parallelen erkennen. So zeigt die Keramik der Dongo-Gruppe die auch bei den älteren Stilen beobachtbaren innenseitig mit feinen Rillenbündeln versehenen Ränder, und die Mokelo-Keramik weist wie auch die \mbox{Ngbanja}-Keramik Winkelmuster, bogenförmige Rillen und unverzierte Gefäßunterteile auf. Zwischen den Stilgruppen \mbox{Ngbanja}, Dongo, Mokelo sowie Motenge-Boma lässt sich eine schwache typologische Kette knüpfen, die beim gegenwärtigen Quellenstand jedoch nicht die Bedingungen für eine Stiltradition erfüllt. Die rezenten Stile Dama und Mbati-Ngombe (Kap.~\ref{sec:DAM-Gr}--\ref{sec:MBN-Gr}) zeigen mit Ausnahme der Rouletteverwendung starke formale Ähnlichkeiten, weisen aber in technischer Hinsicht gravierende Unterschiede auf: in Mbati-Ngombe ist entsprechende Keramik in Aufbautechnik hergestellt, während die rezente Produktion in Dama~I in Abformtechnik erfolgt.\footnote{Siehe Anm.~\ref{ftn:EthnoToepfereiInVorb}.} Beide Gruppen sind folglich unterschiedlichen Technologietraditionen zuzuordnen (Kap.~\ref{sec:TechnologieTrad}). 

\textsc{Wotzka} (1995: 221) wies bereits darauf hin, dass die Systematisierung von stilistischen Ähnlichkeiten zwischen Keramikgruppen als Stiltraditionen nur unter \enquote{Inkaufnahme einer partiell zirkulären Argumentation} vorgenommen werden kann, da eine Existenz \textit{genetischer} Bezüge zwischen den Stilen bereits eine Prämisse innerhalb der Betrachtungen darstellt. Eine Überprüfung der Beziehungen zwischen den konzeptualisierten Keramikstilen ließe sich lediglich durch Hinzuziehung ausreichender \textit{externer} Kriterien überprüfen; hierzu zählt \textcite{Wotzka.1995} entsprechend aussagekräftige stratigraphische Grabungsbefunde, Serien geschlossener Funde, historische Datierungsansätze sowie umfangreiche, hinreichend gut aufgelöste Serien von Radiokohlenstoffdatierungen. Vor der Annahme, dass die stilistischen Ähnlichkeiten zwischen zeitlich benachbarten Gruppen als Belege für entwicklungsgeschichtliche Relationen anzusehen sind, können die hier vorgestellten Betrachtungen, wie jene von \textsc{Wotzka} (ebd. 221--225), lediglich einen heuristischen Versuch darstellen. Die hier präsentierte Chronologie (Kap.~\ref{sec:Zeitscheiben}) ist auch nur eine \textit{in toto} stilunabhängige Betrachtung, da sie nur bedingt auf externen Datierungskriterien basiert.\footnote{Generell ist die Quellenlage im hier präsentierten Arbeitsgebiet jedoch noch rudimentärer als in dem von \textcite{Wotzka.1995} bearbeiteten Inneren Kongobecken, so dass auch die Belastbarkeit der herausgearbeiteten Sequenz deutlich geringer eingeschätzt werden muss. Es sei darauf hingewiesen, dass \textsc{Wotzka} (1995) für seine Arbeit zur Besiedlungsgeschichte des Inneren Kongobeckens insgesamt 63 Grabungsbefunde von 20 individuellen Fundplätzen zur Verfügung standen, darunter 34 Gruben \parencite[siehe][]{Wotzka.1993}. Insgesamt lagen der Untersuchung von \textcite{Wotzka.1995} etwa 11\,100 Objekte zugrunde. In die vorliegende Untersuchung flossen etwa 10\,500 Fundobjekte ein. Es wurden aber lediglich an neun Fundstellen Grabungen durchgeführt und dabei 19 Grabungsschnitte angelegt (Kap.~\ref{sec:GrabungenBefunde}). Das Gros der Funde aus dem nordwestlichen Kongobecken stammt aus Surveys (Kap.~\ref{sec:Quellen}).}

Eine \enquote{Art horizontale konzeptuelle Klammer von geringer zeitlicher Tiefe} (ebd. 219) bildet das vom \textit{horizon style} \parencite[108--111]{Kroeber.1944} abgeleitete Konzept des Stilhorizonts. Die Beschreibung von Stilhorizonten für das Innere Kongobecken hängt nach \textcite[224]{Wotzka.1995} von einem \enquote{annähernd gleichzeitigen Bestehen mehrerer regional determinierter, aber untereinander eng verwandter, insgesamt weiträumig verbreiteter Keramikstile} ab. Für die Ältere Eisenzeit des Inneren Kongobecken lässt sich keine Koexistenz verschiedener, die genannte Definition erfüllender Stile belegen. Dies gilt aufgrund der starken Regionalisierung auch für die subrezenten und rezenten Phasen \parencite{Wotzka.1995}. Innerhalb der jüngeren Phase der Späten Eisenzeit, zwischen dem \mbox{11.--17./18.~Jh.} n.~Chr., beschreibt \textsc{Wotzka} (ebd. 224\,f.) zwei Stilhorizonte. Ein älterer \textit{Bondongoid-Horizont} umfasst die Stilgruppen Bondongo, Wafanya, Besongo und Weme, während ein jüngerer \textit{Nkiloid-Horizont} aus den Gruppen Nkile, Malelembe, Bosanga und Bolondo besteht. Diese Stile kommen mehr oder weniger zeitgleich vor und weisen jeweils morphologische Gemeinsamkeiten zueinander auf. Beide von \textcite{Wotzka.1995} postulierten Stilhorizonte des Inneren Kongobeckens decken den größten Teil des Inneren Kongobeckens ab. Innerhalb des hier untersuchten keramischen Materials aus dem nordwestlichen Kongobecken ergeben sich keine ausreichend starken Argumente für die Beschreibung von Stilhorizonten.