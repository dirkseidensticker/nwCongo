\section*{Anlage 2: Radiokohlenstoffdatierungen auf Befunden (Katalog A)}
\sectionmark{Anlage 2: Radiokohlenstoffdatierungen}

Die vorgelegten Befunde sind mit 23 Radiokohlenstoffdatierungen assoziiert. Eine dezidierte Diskussion der Datierungen erfolgt im Zuge der Befundbeschreibungen (Katalog A). Alle aus der Literatur entnommenen und nicht direkt mit den bearbeiten Befunden und Funden in Zusammenhang stehenden Radiokohlenstoffdatierungen faden Eingang in das \textit{Archive des datations radiocarbones d’Afrique centrale} (aDRAC)\footnote{Siehe \url{https://github.com/dirkseidensticker/aDRAC} und \url{https://zenodo.org/record/61113}.} und werden hier nicht gesondert aufgelistet. Insgesamt sind aus dem in Kap.~\ref{sec:Zeitscheiben} umrissenen Gebiet zwischen 6$^\circ$~Nord und 2$^\circ$~Süd sowie 15$^\circ$~Ost und 26$^\circ$~Ost 166 veröffentlichte Radiokohlenstoffdatierungen bekannt.

\paragraph*{Laboratorien}
$\;$ \\

\vspace{.75em}
\begin{tabular}{@{}p{.05\textwidth}p{.95\textwidth}@{}}
GrN & Groningen, Laboratorium voor Algemene Natuurkunde, Rijksuniversiteit Groningen \\ 
KI & Kiel, Institut für Reine und Angewandte Kernphysik der Universität Kiel, C14-Labor \\ 
Poz & Poznan, Poznańskie Laboratorium Radiowęglowe \\ 
\end{tabular}

\vspace{.75em}
{\footnotesize
{\renewcommand{\arraystretch}{1.5}%
\begin{longtable}{@{}p{1.3cm}p{1.5cm}p{1.3cm}p{1cm}p{3.75cm}p{1cm}p{2cm}p{1,3cm}@{}}
\toprule
\textbf{Fundplatz} & \textbf{Befund} & \textbf{Labor"-nummer} & \textbf{\textsuperscript{14}C-Datum} & \textbf{Datierung (2-Sigma)} & \textbf{Befund} & \textbf{Veröff.} & \textbf{Stil} \\ 
\midrule 
\endhead
\bottomrule
\caption{\textsuperscript{14}C-Datierungen aus dem Arbeitsgebiet (Kalibration: OxCal 4.2.2/IntCal 13).}
\label{tab:14Cdatings}
\endfoot
Maluba & MLB 85/1-3-1 & KI-2444 & 1930 \( \pm \) 120 & 342–327 v. Chr. (0,6 \%) \newline 204 v. Chr.–383 n. Chr. (94,8 \%) & Grube & \textsc{Eggert} 1993, 314 Tab. 16.6 & BTM \\ 
Maluba & MLB 85/1-3-1 & GrN-13584 & 1670 \( \pm \) 110 & 125–605 n. Chr. & Grube & \textsc{Eggert} 1993, 314 Tab. 16.6 & BTM \\ 
Maluba & MLB 85/1-3-2 & KI-2445 & 2140 \( \pm \) 200 & 764–680 v. Chr. (3,6 \%) \newline 674 v. Chr.–245 n. Chr. (91,8 \%) & Grube & \textsc{Eggert} 1993, 314 Tab. 16.6 & BTM \\ 
Maluba & MLB 85/1-3-2 & GrN-13585 & 1990 \( \pm \) 60 & 166 v. Chr.–129 n. Chr. & Grube & \textsc{Eggert} 1993, 314 Tab. 16.6 & BTM \\ 
Maluba & MLB 85/1-4-3 & Poz-62102 & 580 \( \pm \) 30 & 1300–1369 n. Chr. (63,6 \%) \newline 1381–1419 n. Chr. (31,8 \%) & Grab & - & - \\ 
Maluba & MLB 85/1-4-3 & Poz-62103 & 810 \( \pm \) 80 & 1030–1297 n. Chr. & Grab & - & - \\ 
Pikunda & PIK 87/1 & KI-2891 & 600 \( \pm \) 75 & 1276–1438 n. Chr. & Grube & - & MDB  \\ 
Pikunda & PIK 87/1 & KI-2877 & 1980 \( \pm \) 100 & 350–304 v. Chr. (2,1 \%) \newline 210 v. Chr.–251 n. Chr. (93,3 \%) & Grube & \textsc{Eggert} 1993, 314 Tab. 16.6 & PKM \\ 
Pikunda & PIK 87/3 & KI-2892 & 840 \( \pm \) 41 & 1048–1086 n. Chr. (8,1 \%) \newline 1123–1138 n. Chr. (2,5 \%) \newline 1150–1271 n. Chr. (84,4 \%) & Ofen & - & EBA (?) \\ 
Munda & MUN 87/1-0-1 & KI-2882 & 1110 \( \pm \) 110 & 674–1059 n. Chr. (88,0 \%) \newline 1075–1155 n. Chr. (7,4 \%) & Grube & - & EBA (?) \\ 
Munda & MUN 87/1-0-1 & KI-2883 & 870 \( \pm \) 180 & 779–1411 n. Chr. & Grube & - & EBA (?) \\ 
Munda & MUN 87/1-0-2 & KI-2884 & 250 \( \pm \) 40 & 1513–1601 n. Chr. (24,2 \%) \newline 1616–1684 n. Chr. (41,5 \%) \newline 1735–1805 n. Chr. (23,3 \%) \newline 1933–1955 n. Chr. (6,4 \%) & Grube & - &  \\ 
Munda & MUN 87/2-1-1 & KI-2885 & 1800 \( \pm \) 80 & 46 v. Chr.–337 n. Chr. & Grube/ Ofen & \textsc{Eggert} 1993, 314 Tab. 16.6 & PKM \\ 
Munda & MUN 87/2-1-1 & KI-2887 & 2020 \( \pm \) 180 & 486–462 v. Chr. (0,4 \%)/ 450–441 v. Chr. (0,1 \%) \newline 417 v. Chr.–416 n. Chr. (94,8 \%) & Grube/ Ofen & \textsc{Eggert} 1993, 314 Tab. 16.6 & PKM \\ 
Munda & MUN 87/2-1-1 & KI-2881 & 1990 \( \pm \) 45 & 108 v. Chr.–90 n. Chr. (92,7 \%) \newline 100–124 n. Chr. (2,7 \%) & Grube/ Depot & \textsc{Eggert} 1993, 314 Tab. 16.6 & PKM \\ 
Munda & MUN 87/2-1-1 & KI-2886 & 1910 \( \pm \) 80 & 102 v. Chr.–260 n. Chr. (93,0 \%) \newline 283–324 n. Chr. (2,4 \%) & Grube/ Depot & \textsc{Eggert} 1993, 314 Tab. 16.6 & PKM \\ 
Munda & MUN 87/2-1-3 & KI-2888 & 1990 \( \pm \) 65 & 182 v. Chr.–135 n. Chr. & Grube & \textsc{Eggert} 1993, 314 Tab. 16.6 & PKM \\ 
Munda & MUN 87/2-1-3 & KI-2876 & 1980 \( \pm \) 41 & 89–75 v. Chr. (1,5 \%) \newline 56 v. Chr.– 125 n. Chr. (93,9 \%) & Grube & \textsc{Eggert} 1993, 314 Tab. 16.6 & PKM \\ 
Munda & MUN 87/3 & KI-2890 & 1680 \( \pm \) 90 & 134–553 n. Chr. & Ofen & \textsc{Eggert} 1993, 314 Tab. 16.6 & PKM \\ 
Munda & MUN 87/3 & KI-2889 & 1650 \( \pm \) 80 & 220–592 n. Chr. & Ofen & \textsc{Eggert} 1993, 314 Tab. 16.6 & PKM \\ 
Likwala-aux-Herbes & LKW 87/186 & KI-2893 & 1960 \( \pm \) 90 & 197 v. Chr.–245 n. Chr. & Grube & \textsc{Eggert} 1993, 314 Tab. 16.6 & - \\ 
Mitula & MIT 87/103 & KI-2895 & 2230 \( \pm \) 100 & 706–695 v. Chr. (0,3 \%) \newline 540–19 v. Chr. (94,7 \%) \newline 13–1 v. Chr. (0,4 \%) &  & \textsc{Eggert} 1993, 314 Tab. 16.6 & IMB \\ 
Mobaka & MKA 87/102 & KI-2894 & 2270 \( \pm \) 160 & 781–5 v. Chr. (95,2 \%) \newline 12–17 n. Chr. (0,2 \%) &  & \textsc{Eggert} 1993, 314 Tab. 16.6 & IMB \\ 
\end{longtable}}}
\clearpage