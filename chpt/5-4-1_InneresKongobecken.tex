\subsection{Inneren Kongobecken: \textit{Äquator-Co-Tradition}}\label{sec:ICB_StilGrDatierungen}
%\subsubsection*{Inneres Kongobecken}

Das von \textcite{Wotzka.1995} für das Inneren Kongobecken vorgelegte Chronologiesystem liefert erstmals eine umfängliche Übersicht über die keramischen Entwicklungslinien und darauf aufbauende Rückschlüsse auf den Besiedlungsgang dieser östlich an das Arbeitsgebiet angrenzenden Großregion. Die Sequenz Wotzkas gründet auf einem relativchronologischen System aus typologischen Verkettungen, welches durch absolutchronologische Kalibrationspunkte erweitert wurde. Im Folgenden sollen die jeweiligen Synthesen Wotzkas zur zeitlichen Einordnung der 34 von ihm beschriebenen Stilgruppen aus dem Inneren Kongobecken kurz referiert werden. \textcite[]{Wotzka.1995} kondensiert die sich aus den einzelnen Stilgruppen ergebenden typologischen Abhängigkeiten zu sechs Stiltradtionen, von denen fünf genealogisch zusammenhängen und die \textit{Äquator-Co-Tradition} bilden (Abb.~\ref{fig:Wotzka1995_222Abb4}).
% \todo{wichtig, weil Übernahme in eigenes System!!!}{}

\begin{figure*}[p]
\centering
\includegraphics[width = \textwidth]{lit/Wotzka1995_TypenICB_EIA1-3.pdf}
\caption{Inneres Kongobecken: Typvertreter des ältere Abschnitts (4.~Jh.~v.~Chr.--6.~Jh.~n.~Chr.)  der \textit{Äquator-Co-Tradition} nach \textcite{Wotzka.1995}.\\{\footnotesize 1--4: Imbonga-Gruppe (ebd. 453 Taf.~19.5, 490 Taf.~56.2, 496 Taf.~62.10, 453 Taf.~19.6); 5--6: Bonkake-Gruppe (ebd. 495 Taf.~61.6, 495 Taf.~61.8); 7--8: Ingende-Gruppe (ebd. 479 Taf.~45.1, 484 Taf.~50.1); 9--10: Inganda-Gruppe (ebd. 472 Taf.~38.3, 472 Taf.~38.4); 11--12: Lokondolo-Gruppe (ebd. 500 Taf.~66.16, 455 Taf.~21.6); 13--14: Yete-Gruppe (ebd. 505 Taf.~71.9, 505 Taf.~71.14); 15: Monkoto-Gruppe (ebd. 487 Taf.~53.4); 16: Bokele-Gruppe (ebd. 454 Taf.~20.7); 17--18: Lingonda-Gruppe (ebd. 510 Taf.~76.10, 511 Taf.~77.4); 19: Lusako-Gruppe (ebd. 485 Taf.~51.3); 20--21: Bokuma-Gruppe (ebd. 526 Taf.~92.5, 474 Taf.~40.4a).}}
\label{fig:Wotzka1995_TypenICB_EIA1}
\end{figure*}

Für den ältesten, im Inneren Kongobecken nachgewiesenen Keramikstil, die Imbonga-Gruppe, kommt \textsc{Wotzka} (ebd. 67) zu dem Schluss, dass die \enquote{akzeptablen Datierungen in den Zeitraum zwischen etwa 400 und 100 v.~Chr. fallen}. Von insgesamt zwölf aus Imbonga-Kontexten stammenden Radiokohlenstoffdatierungen (ebd. 67 Tab.~9); Tab.~\ref{tab:Wotzka1995-412_14C_Repr}) können lediglich sechs als repräsentativ angesehen werden, während zwei Datierungen (Hv-11574, Hv-12627) deutlich zu alt und vier Daten (Hv-11573, Hv-11571, GrN-13586, KN-4205) deutlich zu jung ausfallen.\footnote{Die urprüngliche Definition Der Imbonga-Gruppe durch \textcite{Eggert.1983} umfasste auch das von \textcite{Wotzka.1995} den Stilgruppen Bonkake, Ingende und Inganda zugewiesene Formenspektrum. Siehe auch Anm.~\ref{ftn:AufteilungIMB}.} Die Keramik der Bonkake-Gruppe kann als \enquote{teilweise mit einer fortgeschrittenen Phase des Imbonga-Stiles identisch} angesehen werden (ebd. 72). Die ebenfalls \enquote{in etwa zeitgleiche, teilweise mit dem Imbonga-Stil parallele} Ingende-Gruppe wird von \textsc{Wotzka} (ebd. 88) hingegen als das um zirka 100 v.~Chr. zu verortende Ende von Imbonga überdauernd angesehen. Die Keramik der Inganda-Gruppe setzt \textsc{Wotzka} (ebd. 84) \enquote{aufgrund stilistischer Erwägungen im Wesentlichen [als] jünger} als das Material der Bonkake- sowie Ingende-Gruppe an. Es zeigt sich jedoch eine \enquote{stilistische Nähe zwischen Imbonga- und Bonkake-Gruppe einerseits und Bonkake- und Inganda-Gruppe andererseits} (ebd.), welche als Effekt einer zeitlichen Nähe der besagten Gruppen anzusehen ist. Drei übereinstimmende Radiokarbondaten für die Inganda-Gruppe aus Bokuma (ebd. 84 Tab.~22; Tab.~\ref{tab:Wotzka1995-412_14C_Repr}) \enquote{umfassen einen Gesamtzeitraum zwischen 196 v.~Chr. und 80 n.~Chr.} und stützen somit die relativ-chronologischen Beobachtungen für die Datierung der Inganda-Keramik. Stilistisch eng mit der Inganda-Keramik verbunden, zeigt das Fundmaterial der Lokondola-Gruppe Bezüge zu jüngeren Stilgruppen und muss \enquote{als [ein] im Wesentlichen später als die Inganda-Gruppe zu datierendes Phänomen gelten} (ebd. 89). Die Keramik der Yete-Gruppe zeigt einerseits zwar deutliche Ähnlichkeiten zum Material der Lokondola-Gruppe, andererseits sind die Verbindungen zur Inganda-Gruppe jedoch deutlich unspezifischer (ebd. 93). Wie die Lokondola-Keramik, so zeigt auch das Yete-Material Bezüge zu jüngeren Gruppen wie der Bokuma- und Lingonda-Gruppe, so dass \enquote{Lokondola- und Yete-Stil zunächst als mehr oder minder zeitgleiche Phänomene} aufgefasst werden können (ebd. 94). Der \enquote{Monkoto-Stil [repräsentiert] eine sowohl der Lokondola- als auch der Yete-Gruppe [zeitlich] benachbarte, im Wesentlichen aber nach diesen beiden Stilgruppen anzusetzende} Stilgruppe (ebd. 99). Basierend auf den relativ-chronologischen Beobachtungen sieht \textsc{Wotzka} (ebd.) zeitliche Überschneidungen dieser drei Stilgruppen als sehr wahrscheinlich an. Aus Kontexten mit Fundgut der Monkoto-Gruppe liegen fünf Radiokohlenstoffdatierungen vor (ebd. 99 Tab.~32; Tab.~\ref{tab:Wotzka1995-412_14C_Repr}), von denen nach Wotzkas Betrachtung allein das jüngste Datum (Hv-12613), welches kalibriert einen Zeitraum vom 1.~Jh.~v.~Chr. bis zum 2.~Jh.~n.~Chr. abdeckt, als Repräsentativ angesehen werden kann. Die Keramik der Bokele-Gruppe ist als etwa zeitgleich mit jener des Monkoto-Stils anzusehen (ebd. 103). Aufgrund von Fundvergesellschaftungen am Fundplatz Bokele (Fpl.~14) sei \enquote{eine Überschneidung der Laufzeiten von Bokele-, Lingonda-, Longa-, und Mbandaka-Keramik [...] belegt} sowie durch stilistische Bezüge zwischen den Gruppen Bokele-, Lokondola- und Yete-Gruppe als wahrscheinlich anzusehen (ebd. 103). Ein Radiokohlenstoffdatum aus einer Deponierung mit Bokele- sowie Lingonda-Keramik in Bokele liefert die einzige absolut-chronologischen Referenz für die Keramik der Bokele-Gruppe (ebd. 312 Kat.-Nr.~9; Tab.~\ref{tab:Wotzka1995-412_14C_Repr}). Das Datum (KN-4204) deckt einen Zeitraum vom 1.--3.~Jh.~n.~Chr. ab, was nach \textsc{Wotzka} (ebd. 104) \enquote{vorläufig grob die ersten beiden nachchristlichen Jahrhunderte als Epoche des Bokele-Stiles} wahrscheinlich macht. Die Keramik der Lusako-Gruppe wird von Wotzka (ebd. 106--107) \enquote{im Wesentlichen [als] später als der Inganda-Stil und in etwa zeitgleich mit dem Lokondola- und dem Yete-Stil} angesehen. Das einzige für die Lusako-Keramik repräsentative Radiokohlenstoffalter stammt auf dem Grabungsbefund PIK~87/1 in Pikunda am Sangha, der erst im Rahmen dieser Arbeit final aufgearbeitet wurde (Kat.-Nr.~8). Das betreffende Datum (KI-2877) deckt einen Zeitraum vom späten 3.~Jh.~v.~Chr. bis in die Mitte des 3.~Jh.~n.~Chr. ab. Dieser Zeitraum deckt sich mit denen von \textcite[107]{Wotzka.1995} für die Keramik der Inganda-Gruppe präsentierten drei Datierungen und stützt somit den relativ-chronologischen Ansatz der Lusako-Gruppe. Für die Keramik der Lingonda-Gruppe liegen sechs Radiokohlenstoffdatierungen vor, von denen lediglich drei (GrN-14004, GrN-13076, KN-4206) den von \textsc{Wotzka} (ebd. 115 Tab.~44; Tab.~\ref{tab:Wotzka1995-412_14C_Repr}) erarbeiten chronologischen Ansatz stützen. Generell nimmt \textsc{Wotzka} (ebd. 115) \enquote{für die Lingonda-Keramik ein[en] Zeitrahmen zwischen der jüngeren Hälfte des dritten und der Mitte des siebten Jahrhunderts} an. Als zeitgleich zur Lingonda-Keramik sieht \textsc{Wotzka} (ebd. 120) die Funde der Bokuma-Gruppe. Die Fundvergesellschaftung mit Material der Stilgruppen Inganda, Lokondola und Lingonda innerhalb des Grabungsbefund BOK~85/3 (ebd. 330--331 Kat.-Nr.~20) zeigt eine \enquote{partielle Überschneidung der Laufzeiten der in dem genannten Befund assoziierten Stilgruppen} an (ebd. 120). Absolut-chronologische Ansätze liefert lediglich eine Datierung, die einen Zeitraum vom 3.--5.~Jh.~n.~Chr. abdeckt (ebd. 120; Tab.~\ref{tab:Wotzka1995-412_14C_Repr}).
%Während die bislang vorgestellten Stilgruppen einen frühen Abschnitt der Besiedlung des Inneren Kongobeckens repräsentieren, welcher im Allgemeinen auch als Frühe Eisenzeit (\textit{Early Iron Age}) bezeichnet werden kann, stellen die folgenden Gruppen in einem gewissen chronologischen Bruch da, der im Anschluss weiter ausgeführt werden soll.
%\todo{!!!}{}

\begin{figure*}[p]
 \centering
 \begin{subfigure}{\textwidth}
 \centering
 \includegraphics[width = .85\textwidth]{lit/Wotzka1995_56Abb3}
 \caption{Schema der relativ-chronologischen Bezüge der keramischen Stilgruppen des Inneren Kongobeckens \parencite[56 Abb.~3]{Wotzka.1995}. \vspace{2em}}
 \label{fig:Wotzka1995_56Abb3}
\end{subfigure}
\begin{subfigure}{\textwidth}
 \centering
 \includegraphics[width = .85\textwidth]{lit/Wotzka1995_222Abb4}
 \caption{Die \textit{Äquator-Co-Tradition} des Inneren Kongobeckens \parencite[222 Abb.~4]{Wotzka.1995}.}
 \label{fig:Wotzka1995_222Abb4}
\end{subfigure}
 \caption{Chronologiesystem für das Innere Kongobecken.}
 \label{fig:Wotzka1995_Chronologiesysteme}
\end{figure*}

\begin{figure*}[tb!]
\centering
\includegraphics[width = \textwidth]{lit/Wotzka1995_TypenICB_LIA1-3.pdf}
\caption{Inneres Kongobecken: Typvertreter des jüngeren Abschnitts (12.--20.~Jh.~n.~Chr.) der \textit{Äquator-Co-Tradition} nach \textcite{Wotzka.1995}.\\{\footnotesize 1--3: Longa-Gruppe (ebd. 485 Taf.~51.9, 485 Taf.~51.10, 531 Taf.~97.8); 4--9: Bondongo-Gruppe (ebd. 532 Taf.~98.4, 445 Taf.~11.3, 443 Taf.~9.17, 521 Taf.~87.7, 490 Taf.~56.1, 449 Taf.~15.1); 10--11: Mbandaka-Gruppe (ebd. 446 Taf.~12.2, 446 Taf.~12.1); 12--13: Nkile-Gruppe (ebd. 527 Taf.~93.4, 531 Taf.~97.7); 14--15: Besongo-Gruppe (ebd. 509 Taf.~75.4, 522 Taf.~88.4); 16--18: Botendo-Gruppe (ebd. 458 Taf.~24.5, 482 Taf.~48.6, 481 Taf.~47.2); 19--22: Ikenge-Gruppe \parencite[427 Abb.~23.5, 426 Abb.~22.1a--b, 428 Abb.~24.2, 429 Abb.~25.2]{Eggert.1980c}.}}
\label{fig:Wotzka1995_TypenICB_LIA1}
\end{figure*}

Die Longa-Gruppe weist einen überbrückenden Charakter von den bereits besprochenen, früheren Stilgruppen zu den im Folgenden zu präsentierenden späteren Gruppen auf.\footnote{Die Longa-Keramik bildet eine Art \enquote*{genetischer Flaschenhals} für die von \textsc{Wotzka} (ebd. 65, 221, 274, 285) postulierte ungebrochene Traditionslinie zwischen den frühesten keramischen Zeugnissen des Inneren Kongobeckens und der rezenten Töpferei in der Region (siehe Abb.~\ref{fig:Wotzka1995_222Abb4}; Kap.~\ref{sec:Horizonte}).\label{ftn:LON-Flaschenhals}} \textsc{Wotzka} (ebd. 127) konstatiert, das die Keramik der Longa-Gruppe \enquote{stilistisch problemlos an das jüngere Ende der bisher entwickelten Keramiksequenz anzubinden ist}. Die Longa-Gruppe weist aber auch Bezüge zur Bokuma- sowie Bokele-Gruppe auf, die generell der älteren Periode zuzurechnen sind. Auf der anderen Seite deuten die noch engeren Bezüge zur Bondongo-Keramik auf ein jüngeres Alter des Longa-Materials hin (ebd.). Aus Kontexten mit Fundgut der Longa-Gruppe stammen drei sehr unterschiedliche Radiokohlenstoffdatierungen (ebd. 127 Tab.~53., 128; Tab.~\ref{tab:Wotzka1995-412_14C_Repr}), die eine verbindliche absolut-chronologische Einordnung unterbanden. Im Zusammenhang mit im Jahr 2015 erstellten Typentafeln für die einzelnen Stilgruppen wurde für die Datierung der Longa-Gruppe ein möglicher Zeitraum vom 12./13.~Jh.~n.~Chr. von Wotzka vorgeschlagen.\footnote{Siehe \url{http://www.fstafrika.phil-fak.uni-koeln.de/22669.html} (Stand 01.02.2016).\label{ftn:fstafrikaWebStilGr-Tafeln}} Dieser neuere Zeitansatz wird durch keine aktuelleren Datierungen getragen, er spiegelt vielmehr eine neue Gewichtung der starken typologischen Bezüge der Longa-Keramik zur Bondongo-Keramik und damit verbunden die Ansicht wider, dass das Jüngste der drei vorliegenden Radiokohlenstoffdatierungen (ebd. 127 Tab.~53: Hv-11572) als potentiell repräsentativ für die Longa-Gruppe gelten kann. Neben der Imbonga-Keramik stellt die Keramik der Bondongo-Gruppe die am besten durch absolute Daten chronologisch abgesicherte Stilgruppe des Inneren Kongobeckens dar. Neun der insgesamt 14 vorliegenden Datierungen (ebd. 138 Tab.~58; Tab.~\ref{tab:Wotzka1995-412_14C_Repr}), die einen Zeitraum zwischen dem Beginn des 11. Jh. bis zum Ende des 14 Jh.~n.~Chr. abdecken, werden von Wotzka als repärsentativ für die Stilgruppe angesehen (ebd. 138). Im Fall der Mbandaka-Gruppe \enquote{sprechen sowohl die Grabungsergebnisse von [der Fundstelle] Mbandaka als auch die engen Stilbezüge zwischen Mbandaka- und Bondongo-Keramik für eine weitgehende oder sogar vollständige zeitliche Kongruenz beider Stilgruppen} (ebd. 143). Von stilistischer Seite aus, bleiben für Wotzka Zweifel an einer \enquote{unmittelbaren chronologischen Nachbarschaft} (ebd. 149) der Keramik der Nkile-Gruppe zu jener der Bondongo-Gruppe. Auch \enquote{stellt die Nkile-Gruppe ein stilistisches Bindeglied zwischen der Bondongo- und der Botendo-Keramik dar} (ebd. 149). Absolut-chronologisch kann die Nkile-Gruppe lediglich auf der Basis eines Datums (Hv-8916) aus der namensgebenden Fundstelle Nkile (Fpl.~17) eingeordnet werden, welches einen Zeitraum vom Ende des 13. bis zum Ende des 14. Jh.~n.~Chr. abdeckt (ebd. 150; Tab.~\ref{tab:Wotzka1995-412_14C_Repr}). Neben Fundmaterial der Nkile-Gruppe repräsentiert dieses Datum jedoch auch Keramik der Bondongo-Gruppe. Die bereits erwähnte Botendo-Keramik stellt relativ- wie auch absolut-chronologisch ein jüngeres Phänomen dar, es handelt sich um die zeitgenössische Keramik des sogenannten \enquote{Unabhängigen Kongostaates} (ebd. 157). Vertreter dieses Stils konnten von \textsc{Wotzka} (ebd. 157) in der Veröffentlichung ethnografischen Fundmaterials aus dem ehemaligen \enquote*{Unabhängigen Freistaat Kongo} \parencite[155--157 Taf.~12]{Coart.1907} identifiziert werden. Zwei Radiokohlenstoffdatierungen aus der Grabung NKI~2 in Nkile \parencite[Fpl.~17; ][158 Tab.~70, 318--322 Kat.-Nr.~13; Tab.~\ref{tab:Wotzka1995-412_14C_Repr}]{Wotzka.1995}, ergaben ein rezentes Probenalter (Hv-9562). Die zweite Probe (Hv-9561) deckt einen Zeitraum zwischen dem Ende des 17.~Jh. bis in rezente Zeiten ab. Den Abschluss der für den westlichen Teil des Inneren Kongobeckens von Wotzka erarbeiteten Sequenz, den er unter der Bezeichnung \textit{West-Tradition} subsumiert (ebd. 221f.), bildet die Keramik aus Ikenge (Fpl.~20), deren Produktion noch Ende der 1970er-Jahre beobachtet werden konnte \parencite{Eggert.1980c}. 

Die chronologische Einordnung der keramischen Stilgruppen der Traditionen \textit{Luilaka}, \textit{Tshuapa}, \textit{Busira}, \textit{Maringa} sowie der sogenannten \textit{Nord-Tradition} \parencite[221--224]{Wotzka.1995} fiel aufgrund mangelnder Radiokohlenstoffdatierungen in vielen Fällen deutliche schwerer. Für die der \textit{Luilaka-Tradition} zugerechnete Bekongo-Gruppe sieht \textsc{Wotzka} (ebd. 162) auf Basis stilistischer Erwägungen \enquote{eine Zeitstellung im chronologischen Überschneidungsbereich zwischen Longa- und Bondongo-Stil} jedoch keine \enquote{vollständige zeitliche Kongruenz mit dem Longa-Stil} und auch keine \enquote{komplette Parallelisierung mit der Bondongo-Gruppe}. Für die Wafanya-Gruppe steht von stilistischer Seite \enquote{eine besondere Nähe zur Bondongo-Gruppe außer Frage} (ebd. 167), so dass eine zeitliche Parallelität mit dem \enquote*{klassischen} Bondongo sehr wahrscheinlich scheint. Darüber hinaus kann laut Wotzka als sicher gelten, dass sie zeitlich direkt auf die Bekongo-Gruppe folgt. \textit{Wotzka} (ebd.) sieht in der Keramik der Wafanya-Gruppe \enquote{keine generell späte, sondern vielmehr eine über längere Zeit parallel bestehende, lokale Fazies des Bondongo-Stiles}. 

Den Beginn der \textit{Tshuapa-Tradition} bildet die Keramik der Wema-Gruppe. Bei ihr handelt es sich stilistisch gesehen um eine \enquote{späte, formal und ornamental stark vereinfachte Ausprägung der Bondongo-Keramik} (ebd. 171). Lediglich eine der drei vorliegenden Radiokohlenstoffdatierungen aus Kontexten mit Wema-Keramik (ebd. 171 Tab.~81; Tab.~\ref{tab:Wotzka1995-412_14C_Repr}) können nach Wotzkas Ansicht als Repräsentativ für die Stilgruppe angesehen werden. Dieses eine Datum (GrN-13078) deckt einen Zeitraum vom 13.--14.~Jh.~n.~Chr. ab (ebd. 172). Zwischen die Keramik der Wema-Gruppe und jener der jüngeren Bolondo-Gruppe ist das Material der Bosanga-Gruppe zu verorten, welche zudem eine \enquote{stilistisch [wie] stratigrafisch zu begründende Gleichzeitigkeit mit der Nkile-Gruppe} aufweist (ebd. 175). Wie im Fall der Wema-Gruppe, so stehen auch für die Keramik der jüngeren Bolondo-Gruppe grundsätzlich drei Radiokohlenstoffdatierungen zur Verfügung (ebd. 180 Tab.~88; Tab.~\ref{tab:Wotzka1995-412_14C_Repr}), von denen Wotzka (ebd. 181) jedoch lediglich das jüngste, bis mindestens in das 16.~Jh. zurückreichende Datum (KN-4203) als Repräsentativ ansieht. Die beiden älteren Datierungen (Hv-12618, Hv-12625) lassen sich für ihn nicht mit der erarbeiten relativ-chronologischen Gliederung vereinen. Für die Bokone-Gruppe stellt Wotzka (ebd. 185) zusammenfassend fest, dass sie eine \enquote{sowohl stratigrafisch als auch stilistisch zu begründende relativ-chronologische Stellung zwischen dem Bolondo-Stil und der rezenten Ilemba-Bokonda-Keramik des Tshuapa-Gebietes} einnimmt. Dem Fundgut der Bokone-Gruppe lassen sich keine Radiokohlenstoffdatierungen zuordnen. Ebenfalls ohne absolut-chronologische Daten, lässt sich für die Keramik der Bolombi-Gruppe, \enquote{unter der Prämisse, dass es sich nicht um eine speziell verzierte Fazies der rezenten Tshuapa-Keramik handelt} nur eine \enquote{relativ-chronologische Position zwischen dem Bokone- und dem Ilemba-Bokonda-Stil} annehmen (ebd. 187). Die Keramik der Ilemba-Bokonda-Gruppe repräsentiert die \enquote{rezente Töpferei des unteren Tshuapa-Gebietes} (ebd. 188) wie sie bei den Feldaufenthalten des \textit{River Reconnaissance Project} im Jahr 1983 noch beobachtet werden konnte.

Die der \textit{Busira-Tradition} zugerechnete Keramik der Inkaka-Gruppe lässt sich nicht einwandfrei relativ-chronologisch einordnen. Basierend auf stilistischen Bezügen sowie der Fundumstände eines Gefäßes aus Ngombe-Malala (Fpl.~89) schließt \textsc{Wotzka} (ebd. 196) auf eine \enquote{grundsätzlich subrezente Zeitstellung der Inkaka-Gruppe}. Die der Liyolongo-Gruppe zugerechnete Keramik spiegelt die 1983 beobachtete \enquote{rezente Töpferei in Liyolongo [Fpl.~85] am mittleren Busira} (ebd.) wider und bildet den rezenten Abschluss der keramischen Sequenz in \linebreak\clearpage\noindent dieser Region des Inneren Kongobeckens. Die rezente Töpferei im Bereich des oberen Maringa ließ sich 1985 in Yopoko beobachtet (Fpl.~175; ebd. 197). 

Neben den referierten Stilgruppen definierte \textcite{Wotzka.1995} in dem von ihm untersuchten Fundgut einige stilistische und formale Einheiten, denen er zwar eigene Stilgruppen-Be"-zeich"-nungen gab, die aber eher regionale Ausprägungen überregionaler Gruppen darstellen. Eine dieser Gruppen ist die Lokongo-Gruppe, die \enquote{als regional geprägte Fazies des wesentlich weiträumiger verbreiteten Botendo-Stiles} (ebd. 200) angesehen werden kann. Die Malelembe-Gruppe hingegen kann als \enquote{lokale Variante des Nkile-Stiles} (ebd. 201) verstanden werden. Für die Keramik der Likuku-Gruppe konnte mit Blick auf die raren Vergleichsmöglichkeiten lediglich \enquote{eine in etwa mit dem Bokone-Stil identische Zeitstellung} (ebd.) postuliert werden. Die schmale Materialbasis für die Mpokioko-Gruppe liefert nur spärliche Hinweise auf die chronologische Postion, die \enquote{nur hypothetisch zwischen [der] Bolondo- und die Bokone-Gruppe} (ebd. 202) eingeordnet werden kann. Die Keramik der Besongo-Gruppe kann als \enquote{zeitweise parallel zum Bondongo-Stil bestehendes, im Wesentlichen aber jünger zu datierendes Phänomen} (ebd. 206) angesehen werden. 

%\onecolumn % AP-Stil
\afterpage{%
	\begin{footnotesize}
{\sffamily
\begin{longtable}{@{}P{.075\textwidth}P{.025\textwidth}P{.105\textwidth}P{.1\textwidth}P{.15\textwidth}P{.1\textwidth}P{.1\textwidth}P{.15\textwidth}@{}}
\toprule
\textbf{Fundort} & \textbf{Nr.} & \textbf{Befund} & \textbf{Labor-Nr.} & \textbf{Radiokarbonalter} & \textbf{Stilgruppe} & \textbf{Akzeptiert} & \textbf{Quelle} \\
\midrule
\endhead
\bottomrule
\caption{Inneres Kongobecken: Radiokohlenstoffdatierungen und mit den Proben assoziierte keramische Stile. Die von \textsc{Wotzka} (1995) als repräsentative für die Stile angesehenen Daten sind markiert ($\bullet$).}
\label{tab:Wotzka1995-412_14C_Repr}
\endfoot
 Iyonda & 8 & IYO 81/2 & Hv-12204 & 8750 \( \pm \) 205 bp & - & - & \textsc{Wotzka} 1995: 412 \\
 Bamanya & 12 & BAM 83/2/I & Hv-12616 & 5245 \( \pm \) 695 bp & - & - & ebd. 412 \\
 Imbonga & 43 & IMB 81/9/I & Hv-11574 & 3775 \( \pm \) 105 bp & Imbonga & - & ebd. 66--67 Tab.~9 \\
 Bokuma & 18 & BOK 83/1 & Hv-12627 & 3485 \( \pm \) 220 bp & Imbonga & - & ebd. 66--67 Tab.~9 \\
 Wafanya & 58 & WAF 83/16 & Hv-12612 & 3305 \( \pm \) 250 bp & - & - & ebd. 412 \\
 Imbonga & 43 & IMB 81/3 & Hv-11576 & 2900 \( \pm \) 285 bp & - & - & ebd. 412 \\
 Imbonga & 43 & IMB 81/1 & Hv-12207 & 2860 \( \pm \) 280 bp & Monkoto & - & ebd. 99 Tab.~32 \\
 Wafanya & 58 & WAF 83/16 & Hv-12611 & 2695 \( \pm \) 160 bp & Monkoto/""Longa & - & ebd. 99 Tab.~32, 127 Tab.~53 \\
 Imbonga & 43 & IMB 83/1 & Hv-12614 & 2665 \( \pm \) 110 bp & Monkoto & - & ebd. 99 Tab.~32 \\
 Boso-Njafo & 149 & BSN 85/1 & GrN-14005 & 2440 \( \pm \) 150 bp & Imbonga & $\bullet $ & ebd. 66--67 Tab.~9 \\
 Bokele & 14 & BKE 81/1 & GrN-13583 & 2290 \( \pm \) 70 bp & Imbonga & $\bullet $ & ebd. 66--67 Tab.~9 \\
 Boso-Njafo & 149 & BSN 85/3 & KI-2439 & 2270 \( \pm \) 70 bp & Imbonga & $\bullet $ & ebd. 66--67 Tab.~9 \\
 Boso-Njafo & 149 & BSN 85/3 & GrN-14006 & 2260 \( \pm \) 80 bp & Imbonga & $\bullet $ & ebd. 66--67 Tab.~9 \\
 Bokuma & 18 & BOK 83/1 & KI-2363 & 2260 \( \pm \) 60 bp & Imbonga & $\bullet $ & ebd. 66--67 Tab.~9 \\
 Bamanya & 12 & BAM 81/1 & Hv-11570 & 2245 \( \pm \) 195 bp & - & - & ebd. 412 \\
 Bamanya & 12 & BAM 83/2 & Hv-12615 & 2210 \( \pm \) 180 bp & - & - & ebd. 412 \\
 Boso-Njafo & 149 & BSN 85/1 & Erl-17763 & 2201 \( \pm \) 52 bp & Imbonga & $\bullet $ & \textsc{Kahlheber, Eggert} u.~a. 2014, 500 Tab.~4 \\
 Imbonga & 43 & IMB 81/9 & KI-2428 & 2160 \( \pm \) 90 bp & Imbonga & $\bullet $ & \textsc{Wotzka} 1995, 66--67 Tab.~9 \\
 Imbonga & 43 & IMB 81/1 & Hv-11575 & 2130 \( \pm \) 125 bp & Monkoto & - & ebd. 99 \\
 Bokuma & 18 & BOK 83/2 & GrN-14003 & 2090 \( \pm \) 70 bp & Inganda & $\bullet $ & ebd. 84 Tab.~22 \\
 Bokuma & 18 & BOK 83/2 & KI-2433 & 2025 \( \pm \) 75 bp & Inganda & $\bullet $ & ebd. 84 Tab.~22 \\
 Bokuma & 18 & BOK 83/2 & KI-2432 & 2020 \( \pm \) 100 bp & Inganda & $\bullet $ & ebd. 84 Tab.~22 \\
 Pikunda & 255 & PIK 87/1 & KI-2877 & 1980 \( \pm \) 100 bp & Pikunda-Munda/""Lusako & $\bullet $ & ebd. 412 \\
 Bamanya & 12 & BAM 83/2 & Hv-12617 & 1955 \( \pm \) 115 bp & - & - & ebd. 412 \\
 Wafanya & 58 & WAF 83/16 & Hv-12613 & 1920 \( \pm \) 90 bp & Monkoto/""(Longa) & $\bullet $ & ebd. 99 Tab.~32, 127 Tab.~53 \\
 Isaka-Elinga & 80 & ISK 83/104 & Hv-12626 & 1895 \( \pm \) 65 bp & Lingonda & - & ebd. 115 Tab.~44 \\
 Bokele & 14 & BKE 81/2 & KN-4204 & 1870 \( \pm \) 70 bp & Bokele/""Lingonda & $\bullet $ & ebd. 104 \\
 Bokele & 14 & BKE 81/3 & Hv-12205 & 1860 \( \pm \) 260 bp & - & - & ebd. 412 \\
 Bokele & 14 & BKE 81/1 & Hv-11573 & 1850 \( \pm \) 120 bp & Imbonga & - & ebd. 66--67 Tab.~9 \\
 Iyonda & 8 & IYO 81/2 & Hv-11577 & 1785 \( \pm \) 125 bp & Monkoto/""Bokuma & - & ebd. 303 \\
 Bolondo & 96 & BLD 83/1 & Hv-12624 & 1725 \( \pm \) 95 bp & Bondongo/""Wema & - & ebd. 138 Tab.~58, 171 Tab.~81 \\
 Bokuma & 18 & BOK 83/3 & GrN-14004 & 1670 \( \pm \) 70 bp & Bokuma/""Lingonda & $\bullet $ & ebd. 115 Tab.~44 \\
 Isaka-Elinga & 80 & ISK 83/104 & KN-4206 & 1590 \( \pm \) 60 bp & Lingonda & $\bullet $ & ebd. 115 Tab.~44 \\
 Isaka-Elinga & 80 & ISK 83/104 & GrN-13076 & 1450 \( \pm \) 45 bp & Lingonda & $\bullet $ & ebd. 115 Tab.~44 \\
 Wafanya & 58 & WAF 83/3 & Hv-12622 & 1245 \( \pm \) 90 bp & Wafanya & - & ebd. 167 \\
 Bolondo & 96 & BLD 83/2 & Hv-12619 & 1195 \( \pm \) 70 bp & Bondongo/""Wema & - & ebd. 138 Tab.~58, 171 Tab.~81 \\
 Bolondo & 96 & BLD 83/2 & Hv-12618 & 1175 \( \pm \) 210 bp & Bolondo & - & ebd. 180--181 Tab.~88 \\
 Bamanya & 12 & BAM 83/1 & Hv-12621 & 1170 \( \pm \) 120 bp & - & - & ebd. 412 \\
 Baringa & 161 & BAR 85/1 & KI-2431 & 950 \( \pm \) 70 bp & Bondongo & $\bullet $ & ebd. 138 Tab.~58 \\
 Bamanya & 12 & BAM 83/1 & Hv-12620 & 945 \( \pm \) 75 bp & Bondongo & $\bullet $ & ebd. 138 Tab.~58 \\
 Bolondo & 96 & BLD 83/1 & Hv-12625 & 915 \( \pm \) 105 bp & Bolondo & - & ebd. 180--181 Tab.~88 \\
 Mbandaka & 10 & MBA 81/2 & Hv-12206 & 810 \( \pm \) 90 bp & Bondongo/""Mbandaka & $\bullet $ & ebd. 138 Tab.~58 \\
 Bokele & 14 & BKE 81/4 & Hv-11572 & 755 \( \pm \) 115 bp & Longa/""Mbandaka & $\circ $ & ebd. 115 Tab.~44, 127 Tab.~53 \\
 Mbandaka & 10 & MBA 81/2 & KI-2364 & 740 \( \pm \) 55 bp & Bondongo/""Mbandaka & $\bullet $ & ebd. 138 Tab.~58 \\
 Longa & 24 & LON 81/1 & Hv-11571 & 730 \( \pm \) 75 bp & Imbonga & - & ebd. 66--67 Tab.~9 \\
 Baringa & 161 & BAR 85/1 & GrN-14002 & 710 \( \pm \) 60 bp & Bondongo & $\bullet $ & ebd. 138 Tab.~58 \\
 Bolondo & 96 & BLD 83/1 & GrN-13078 & 660 \( \pm \) 80 bp & Bondongo/""Wema & $\bullet $ & ebd. 138, 171--172 Tab.~81 \\
 Baringa & 161 & BAR 85/1 & KI-2430 & 650 \( \pm \) 65 bp & Bondongo & $\bullet $ & ebd. 138 Tab.~58 \\
 Bamanya & 12 & BAM 83/1 & KI-2361 & 640 \( \pm \) 70 bp & Bondongo & $\bullet $ & ebd. 138 Tab.~58 \\
 Nkile & 17 & NKI 1 & Hv-8916 & 625 \( \pm \) 50 bp & Nkile/""Bondongo & $\bullet $ & ebd. 138 Tab.~58 \\
 Longa & 24 & LON 81/1 & GrN-13586 & 500 \( \pm \) 90 bp & Imbonga & - & ebd. 66--67 Tab.~9 \\
 Bamanya & 12 & BAM 81/1 & GrN-13077 & 440 \( \pm \) 50 bp & Bondongo & - & ebd. 138 Tab.~58 \\
 Bamanya & 12 & BAM 83/1 & KI-2360 & 420 \( \pm \) 65 bp & Bondongo & - & ebd. 138 Tab.~58 \\
 Wafanya & 58 & WAF 83/16 & KI-2365.01 & 280 \( \pm \) 70 bp & Wafanya & $\bullet $ & ebd. 368 \\
 Longa & 24 & LON 81/1 & KN-4205 & 260 \( \pm \) 120 bp & Imbonga & - & ebd. 66--67 Tab.~9 \\
 Bolondo & 96 & BLD 83/2 & KN-4203 & 230 \( \pm \) 110 bp & Bolondo & $\bullet $ & ebd. 180--181 Tab.~88 \\
 Mbandaka & 10 & MBA 81/2 & Hv-11578 & 230 \( \pm \) 110 bp & Bondongo/""Mbandaka & - & ebd. 138 Tab.~58 \\
 Nkile & 17 & NKI 2 & Hv-9561 & 145 \( \pm \) 55 bp & Botendo & $\bullet $ & ebd. 158 Tab.~70 \\
 Bamanya & 12 & BAM 81/1 & Hv-12203 & 65 \( \pm \) 50 bp & - & - & ebd. 412 \\
 Nkile & 17 & NKI 2 & Hv-9562 & modern & Botendo & $\bullet $ & ebd. 158 Tab.~70 \\
\end{longtable}
}
\end{footnotesize}
}
%\twocolumn % AP-Stil

\begin{figure*}[p]
	\centering
	\includegraphics[height = .9\textheight]{../03_Projects/14C/Kongo/InnerCongo/InnerCongo_Stylegroups_published.pdf}
	\caption{Inneres Kongobecken: Kalibrierung der Radiokohlenstoffdatierungen die von \textcite[67--210, 412]{Wotzka.1995} als repräsentativ für die keramischen Stilgruppen angesehen werden (siehe Tab.~\ref{tab:Wotzka1995-412_14C_Repr}), inklusive der die Imbonga-Gruppe repräsentierenden Datierungen aus Mobaka (Fpl.~246) und Mitula (Fpl.~251) im nordwestlichen Kongobecken und einem neueren Datum aus Boso-Njafo \parencite{Kahlheber.2014}.}
	\label{fig:14C_InnerCongo_Stylegroups}
\end{figure*}

%\begin{figure*}[!tb]
% \centering
% \includegraphics[width = \textwidth]{../03_Projects/14C/Kongo/InnerCongo/w_multigroup.pdf}
% \caption{\textsuperscript{14}C-Datierungen: Vergleich der Kalibration der aller vorliegenden Radiokohlenstoffdatierungen aus dem Inneren Kongobecken unter Auschluss der in Hannover vorgenommenen Datierungen, ausschließlich die in Hannover erzielten Datierungen sowie alle neueren \textcite[nicht in][412]{Wotzka.1995} Datierungen.}
% \label{fig:14C_InnerCongo_Stylegroups}
%\end{figure*}

Funde aus dem Bereich des nördlichen Kongobogens, aus Lisala (Fpl.~184) sowie Nkomba (Fpl.~185) subsumiert Wotzka unter der \textit{Nord-Tradition} (ebd. 223f.). Ihre chronologische Einordnung gestaltet sich mangels absoluter Datierungen sowie dem Mangel an ausgegraben Befundkontexten als schwierig und die gemachten Angaben müssen vorerst als hypothetisch angesehen werden. Das Material der Lisala-Gruppe würde \enquote{nicht nur aufgrund seines schlichten Charakters, sondern insbesondere auch wegen der sehr gut erhaltenen Scherbenoberflächen, der durchweg frischen Brüche und der teilweise recht großen Fragmente einen jungen Eindruck} machen und werde folglich als rezent eingestuft (ebd. 207). Unter der Bezeichnung Nkomba-Gruppe subsumierte Formen konnten von Wotzka, wie schon die Keramik der Botendo-Gruppe, im Fundus der publizierten Keramik aus dem \enquote*{Freistaat Kongo} \parencite[189--191 Taf.~13]{Coart.1907} wiederentdeckt werden. Diese Beobachtung veranlasst ihn zu dem Schluss, die Nkomba-Keramik als \enquote{in etwa den letzten beiden Dekaden des vorigen Jahrhunderts} \parencite[210]{Wotzka.1995} zugehörig anzusehen. Die im Zuge der 2010 und 2013 durch das \enquote*{Boyekoli Ebale Congo}-Projektes\footnote{Siehe Anm.~\ref{ftn:BoyekoliEbaleCongo}.} durchgeführten Befahrung der in den Kongo mündenden Flüsse Itimbiri, Aruwimi und Lomami erbrachten ebenfalls Vertreter der Nkomba-Gruppe (Kap.~\ref{sec:NordCongo}).

Insgesamt lagen zum Abschluss des \textit{River Reconnaissance Project} Ende der 1980er Jahre 59 Radiokohlenstoffdatierungen aus dem Inneren Kongobecken vor (ebd. 412). Im Zusammenhang mit der absolut-chronologischen Einordnung der jeweiligen keramischen Stilgruppen diskutiert \textcite{Wotzka.1995} die Repräsentativität dieser Datierungen individuell und im Detail. Nach einem Abgleich der absoluten Datierungen mit der von ihm ausgearbeiteten relativ-chronologischen Position der jeweiligen Stilgruppen kommt Wotzka im Fall von insgesamt 20 Datierungen zum Schluss, dass sie für die jeweiligen Gruppen nicht repräsentativ seien.\footnote{16 dieser von \textcite{Wotzka.1995} abgelehnten Datierungen stammen aus dem Labor aus Hannover (Hv), während zwei Datierungen aus Groningen (GrN) sowie jeweils eine aus Köln (KN) sowie Kiel (Ki) als nicht-repräsentativ bewertet wurden (siehe Tab.~\ref{tab:Wotzka1995-412_14C_Repr}, \ref{tab:Wotzka1995-412_14C_Repr}).} Eine systematische Auflistung aller aus dem Inneren Kongobecken vorliegenden Radiokohlenstoffdatierungen nach ihrer Akzeptanz sowie ob die Proben in Hannover untersucht wurden oder nicht, verdeutlicht den hohen Anteil an aus Hannover stammenden, von \textcites[132f.]{Eggert.1987c}[328 Anm.~20]{Eggert.1993}[67--210]{Wotzka.1995} verworfenen Datierungen (Tab.~\ref{tab:14C_InnerCongo_Lab_Representation}). Ein Test nach Fischer auf Unabhängigkeit ergibt für die Verteilung einen \textit{p}-Wert von 0,01\,\%, während ein Chi$^2$-Test einen \textit{p}-Wert von 0,03\,\% liefert. Die Nullhypothese, nach der eine zufällige Verteilung vorliegen sollte, muss in beiden Testverfahren verworfen werden. Die von \textcite{Wotzka.1995} diskutierten Radiokohlenstoffdatierungen und die mit hoher Wahrscheinlichkeit nicht zufällig Verteilung der ausgeschlossenen Datierungen unterstreicht das systematische Problem mit in Hannover untersuchten Radiokohlenstoffdatierungen \parencite[siehe][]{Geyh.1990}.

\begin{table*}[!tb]
	\centering
	\begin{minipage}{.67\textwidth}
		{\small
			\begin{sftabular}{@{}lcc@{}}
\toprule
\textbf{Labor} & \textbf{repräsentativ} & \textbf{nicht repräsentativ} \\
\midrule
 Hannover (Hv-) & 6 & 16 \\
 \begin{tabular}[c]{@{}l@{}}Groningen (GrN-), Kiel (KI-)\\und Köln (Kn-)\end{tabular} & 21 & 4 \\
\bottomrule
\end{sftabular}

		}
		\caption{Inneres Kongobecken: Summarische Aufstellung von mit keramischen Stilgruppen assoziierten Radiokohlenstoffdatierungen nach Akzeptanz und Labor \parencite[nach][67--210, 412; Tab.~\ref{tab:Wotzka1995-412_14C_Repr}]{Wotzka.1995}.}
		\label{tab:14C_InnerCongo_Lab_Representation}
	\end{minipage}
\end{table*}

Zusammengenommen können nach Begutachtung der Ausführungen \textsc{Wotzkas} (ebd.) lediglich 27 Radiokohlenstoffdatierungen als repräsentativ für das mit ihnen vergesellschaftete keramische Fundgut angesehen werden (Tab.~\ref{tab:Wotzka1995-412_14C_Repr}, \ref{tab:14C_InnerCongo_Lab_Representation}).\footnote{Insgesamt acht bei \textsc{Wotzka} (ebd. 412) aufgeführte Radiokohenstoffdatierungen sind nicht mit keramischem Material assoziiert, darunter fünf aus den bislang nicht abschließend ausgewerteten Grabungen in Bamanya (Fpl.~12; siehe ebd. 310 Kat.-Nr.~7).} Stellt man zu diesen noch die beiden in Mobaka (Fpl.~246; Kat.-Nr.~13) und Mitula (Fpl.~251; Kat.-Nr.~12) im nordwestlichen Kongobecken gewonnenen Datierungen sowie eine neuere Datierung aus Boso-Njafo \parencite[Fpl.~149;][]{Kahlheber.2014} hinzu, alle drei spiegeln Fundgut der Imbonga-Gruppe wider, so ergeben sich 38 \enquote*{repräsentative} Datierungen für die absolute Chronologie des Inneren Kongobeckens (Abb.~\ref{fig:14C_InnerCongo_Stylegroups}). Die Stilgruppen Imbonga und Bondongo sind mit jeweils neun Radiokohlenstoffdatierungen am dichtesten belegt (Abb.~\ref{fig:14C_InnerCongo_Stylegroups}).\footnote{Neuere Datierungen, die im Zuge des zwischen 2015--2017 laufenden Kölner Regenwaldprojektes unter der Leitung von H.-P.~Wotzka gemacht wurden, flossen nicht in die hier geschilderten Betrachtungen ein, da sie bislang nicht veröffentlicht sind.} Die Keramik der Inganda-Gruppe ist durch drei Datierungsproben absolut-chronolgisch ansprechbar, während für die Stilgruppen Lingonda und Botendo jeweils zwei von \textcite{Wotzka.1995} als repräsentativ eingeschätzte Proben zur Verfügung stehen. Jeweils nur eine von \textcite{Wotzka.1995} akzeptierte Radiokohlenstoffdatierung stützen die Datierungen der Stilgruppen Monkoto, Bokele, Longa sowie Bolondo. Zwei Datierungen lassen sich mehr als einer Stilgruppe zuordnen, je eine den Stilgruppen Bokuma und Lingonda sowie Bondongo und Wema (Abb.~\ref{fig:14C_InnerCongo_Stylegroups}).

\begin{table*}[!tb]
	\centering
	\begin{subtable}{\textwidth}
		\centering
		{\scriptsize 
			\begin{sftabular}{@{}l@{\hskip 10pt}c@{\hskip 10pt}c@{\hskip 10pt}c@{\hskip 10pt}c@{\hskip 10pt}c@{\hskip 10pt}c@{\hskip 10pt}c@{\hskip 10pt}c@{\hskip 10pt}c@{\hskip 10pt}c@{\hskip 10pt}c@{\hskip 10pt}c@{\hskip 10pt}c@{\hskip 10pt}c@{\hskip 10pt}c@{}}
\toprule
\textbf{Stil} & \textbf{IMB} & \textbf{BON} & \textbf{ING} & \textbf{IGB} & \textbf{LOK} & \textbf{YET} & \textbf{MON} & \textbf{BKE} & \textbf{LUS} & \textbf{LDG} &  \textbf{BOK} & \textbf{LON} & \textbf{BDG} & \textbf{NKI} & \textbf{BOT} \\
n = & 316 & 15 & 13 & 22 & 1 & 4 & 67 & 4 & 1 & 31 & 12 & 3 & 27 & 4 & 38 \\
\midrule
\begin{tabular}[c]{@{}l@{}}rund\\B1--3\end{tabular} &  - & - & - & - &  - & - & - & - & - &  - & 8\,\% &  67\,\% &  70\,\% &  100\,\% & 95\,\% \\
\begin{tabular}[c]{@{}l@{}}flach\\B4--14\end{tabular} & 100\,\% & 100\,\% & 100\,\% & 100\,\% &  100\,\% & 100\,\% & 100\,\% & 100\,\% & 100\,\% &  100\,\% &  92\,\% &  33\,\% &  30\,\% &  - &  5\,\% \\
\bottomrule
\end{sftabular}

		}
		\caption{Inneres Kongobecken nach \textcite[63 Tab.~7, 70 Tab.~13, 75 Tab.~17, 80 Tab.~21, 86, 91, 96 Tab.~31, 101 Tab.~36, 105, 109 Tab.~42, 117 Tab.~48, 123, 131 Tab.~57, 140, 145, 154 Tab. 69]{Wotzka.1995}.}
		\label{fig:Wotzka1995Bodenformen}
	\end{subtable}
	\par\vspace{.5cm}
	\begin{subtable}{\textwidth}
		\centering
		{\scriptsize 
			\begin{sftabular}{@{}lcccccccc@{}}
\toprule
\textbf{Stil} & \textbf{BTM} & \textbf{NGB} & \textbf{DON} & \textbf{MKL} & \textbf{MTB} & \textbf{DAM} & \textbf{MBN} & \textbf{BAN} \\
n = & 10 & 1 & 1 & 1 & 2 & 3 & 6 & 3 \\
\midrule
\begin{tabular}[c]{@{}l@{}}rund\\B1--3\end{tabular} & 20\,\% & 100\,\% & 100\,\% & - & 100\,\% & 67\,\% & 83\,\% & 33\,\% \\
\begin{tabular}[c]{@{}l@{}}flach\\B4--14\end{tabular} & 80\,\% & - & - & 100\,\% & - & 33\,\% & 17\,\% & 67\,\% \\
\bottomrule
\end{sftabular}

		}
		\caption{Ubangi- und Lua-Gebiet.}
		\label{fig:nwCongoBodenformenUbangiLua}
	\end{subtable}
	\par\vspace{.5cm}
	\begin{subtable}{\textwidth}
		\centering
		{\scriptsize 
			\begin{sftabular}{@{}lccccccc@{}}
\toprule
\textbf{Stil} & \textbf{PKM} & \textbf{NGO} & \textbf{EBA} & \textbf{EPE} & \textbf{MKA} & \textbf{PDM} & \textbf{BBS} \\
n = & 44 & 10 & 27 & 34 & 11 & 5 & 2 \\
\midrule
\begin{tabular}[c]{@{}l@{}}rund\\B1--3\end{tabular} & 100\,\% & 100\,\% & 11\,\% & 9\,\% & 100\,\% & 100\,\% & 50\,\% \\
\begin{tabular}[c]{@{}l@{}}flach\\B4--14\end{tabular} & - & - & 89\,\% & 91\,\% & - & - & 50\,\% \\
\bottomrule
\end{sftabular}

		}
		\caption{Sangha-/Ngoko und Likwala-aux-Herbes-Gebiet.}
		\label{fig:nwCongoBodenformenSanghaLikwala}
	\end{subtable}
	\caption{Bodenformen: Prozentualer Anteil runder (B1--4) und flacher Böden (B4--14) für die einzelnen -- chronologisch vom ältesten zum jüngsten gereihten -- Stilgruppen des Inneren Kongobeckens sowie als Vergleich die Situation im Arbeitsgebiet.}
	\label{fig:VglBodenformen}
\end{table*}

Vor dem Hintergrund, dass das für das Inneren Kongobecken vorliegende Chronologieschema (Abb.~\ref{fig:Wotzka1995_Chronologiesysteme}) eine von \textsc{Wotzka} (ebd. 65, 221, 274, 285) postulierte ungebrochene Besiedlungskontinuität widerspiegelt, offenbart die hier präsentierte Zusammenstellung der chronologischen Ansprachen der einzelnen Stilgruppen und deren Integration in ein absolut-chro"-no"-lo"-gisches Schema (Abb.~\ref{fig:Chronologiesystem}) eine auffällige Zweiphasigkeit mit einer Unterbrechung von mehreren Jahrhunderten. Auch die Kalibrierung der als repräsentativ angesehenen Radiokohlenstoffdatierungen aus dem Inneren Kongobecken weist eine entsprechende Datierungslücke auf (Abb.~\ref{fig:14C_InnerCongo_Stylegroups}). Zwischen der Lingonda-Gruppe des 3.--7.~Jh.~n.~Chr. und der ins 11.--14.~Jh.~n.~Chr. datierenden Bondongo-Gruppe liegen keine Daten vor. 

Mit Blick auf einzelne formale Kerncharakteristika der Keramik des Inneren Kongobeckens ergeben sich zudem auf Merkmalsebene Auffälligkeiten zwischen den älteren und jüngeren Stilen, die Fragen an dem Postulat eines ungebrochenen Entwicklungsgangs durch \textsc{Wotzka} (ebd. 65, 221, 274, 285) aufwerfen. Eines dieser Charakteristika ist die Ausformung der Gefäßböden. Mit dem Aufkommen der Keramik im Inneren Kongobecken, der durch die Imbonga-Gruppe getragen ist, bestimmen bis zu den Stilgruppen Lingonda und Bokuma flache Standböden das Formenspektrum (Abb.~\ref{fig:Wotzka1995Bodenformen}). Beginnend mit der Bondongo-Gruppe dominieren hingegen runde Böden die Inventare.\footnote{Die geringen Anzahlen der identifizierten Böden innerhalb der Stilgruppen Bondongo, Nkile und Botendo ist auf das grundsätzlich schwere Erkennen runder Böden zurückzuführen. Diese lassen sich, anders als flache Böden, nur deutlich schwieriger von anderen Partien der Gefäßwandung unterscheiden.} Daneben fällt auf, dass das \textit{banfwa-nfwa}-Dekor (Tab.~\ref{tab:Verzierungselemente}: 08) zwar bereits innerhalb der Lingonda-Gruppe aufkommt und dort an den Innenseiten der Ränder sowie den Randlippen angebracht wird (ebd. 109--113), es aber mit dem Beginn der jüngeren Phase, die durch den Beginn des Bondongo-Stils repräsentiert ist, zum bestimmenden Verzierungselement für die Dekorierung großer Teile der Gefäßwandungen wird. Es löst dabei die in der ältere Phase regelhaft zu beobachtende Wiegebandtechnik ab.