\usepackage[style = authoryear-icomp, 
 sorting = nyt, 
 backend = biber, 
 uniquename = false, 
 language = auto, 
 mincitenames = 1,
 maxcitenames = 2, % nach 2 Autoren bereits u.a.
 maxbibnames = 100, 
 giveninits = true, % kürzt die Vornamen ab
 uniquename = false, 
 dashed = true, 
 ibidtracker = true,
 mergedate = false]{biblatex}
\bibliography{bib/bib.bib}
%\addbibresource{Lit.bib}
\renewcommand*{\mkbibnamefamily}[1]{\textsc{#1}}	% Autorennamen in Kapitälchen
%\renewcommand*{\mkbibnamelast}[1]{\textsc{#1}}	% bei älterem biblatex (<=3.3)
\renewcommand*{\bibfont}{\footnotesize}	% kleiner Text im Literaturverzeichnis

%\renewcommand*{\postnotedelim}{\ifciteibid{\addcolon\space}{\space}}

\renewcommand*{\postnotedelim}{\addcolon\space}	% Doppelpunkt anstatt Komma vor Seitenzahl

\renewcommand*{\finalnamedelim}{\addspace\&\space} % &-Zeichen zwischen Autoren

\DeclareFieldFormat{pages}{#1}	% entfernt 'S.' in Kurzzitaten & Literaturliste
\DefineBibliographyStrings{german}{%
	page = {{}{}},
	pages = {{}{}},
	% andothers = {{et\,al\adddot}},
}

% Bibliography

\AtEveryBibitem{
 \clearlist{publisher}
 %\clearfield{month}
 \clearfield{issn}
 \clearfield{isbn}
 \clearfield{url}
 \clearfield{doi}
}  

\renewcommand{\labelnamepunct}{\quad}         % Geviertleerzeichen zwischen Autor (2006) und Titel im LV
%\renewcommand{\bibnamedash}{--\hspace{7mm}}      % (Halbgeviert) als Ersatz für wiederkehrende Autoren oder Herausgeber in der Bibliografie verwendet wird
\renewcommand{\bibpagespunct}{\addcolon\addspace} % Seitenzahlen abtrennen mit : 
\DeclareFieldFormat{title}{\normalfont{#1}}    % Titel in normaler Schrift im LV
\DeclareFieldFormat[article,incollection]{title}{#1}      % Titel in Journals und Sammelbänden nicht in "Hochkomma" im LV
%\DeclareFieldFormat{myjournaltitle}{#1} % Zeitschriftentitel kursiv (funktioniert nur ohne \emph{} Auszeichnung!???)
\DeclareFieldFormat{series}{\emph{#1}} % Reihentitel kursiv
\DeclareFieldFormat{issuetitle}{#1} % Ausgabetitel nicht kursiv
\DeclareFieldFormat{maintitle}{#1} % Sammelbandtitel nicht kursiv
\DeclareFieldFormat{booktitle}{#1} % Sammelbandtitel nicht kursiv


\setlength\bibitemsep{1.8mm}   % Abstand zwischen Eintr�gen im Literaturvereichnis (1.8mm) im LV
\setlength\bibhang{9mm}
\renewcommand{\bibnamedash}{--\hspace{7mm}}

% Kein In: bei Artikeln 
\renewbibmacro{in:}{   
\ifentrytype{article}{}{\printtext{\bibstring{in}\intitlepunct}}}

% Volume gefolgt von Nummer in Klammern  
\DeclareFieldFormat[article]{number}{\mkbibparens{#1}}
\renewbibmacro*{volume+number+eid}{%
	\printfield{volume}%
	\printfield{number}}     

%% Bei Artikeln die zweite Jahreszahl nicht in Klammer setzen
\renewbibmacro*{issue+date}{%
	\setunit{\addcomma\space}% NEW
	%  \printtext[parens]{% DELETED
	\iffieldundef{issue}
	{\usebibmacro{date}}
	{\printfield{issue}%
		\setunit*{\addspace}%
		%       \usebibmacro{date}}}% DELETED
		\usebibmacro{date}}% NEW
	\newunit}

%%%  Inbook, Incollection, Inproceedings
%  Editor nach dem In: und vor dem Titel
\renewbibmacro*{in:}{
	\ifentrytype{incollection}{%
		\DeclareNameAlias{editor}{first-last}
		\printtext{In: }
		\ifnameundef{editor}
		{}
		{\printnames{editor}%
			\addspace 
			\usebibmacro{editorstrg} % \mkbibparens{\usebibmacro{editorstrg}}, in diesem Falle keine Klammern nach Autor, ed. (2016)
			%\mkbibparens{\usebibmacro{editorstrg}}
			\setunit{\addcomma\addspace}% 
		}%
		\usebibmacro{maintitle+booktitle}
		\clearfield{maintitle}    %% Folgende Eintr�ge sollen nun nicht noch einmal ausgegeben werden 
		\clearfield{booktitle}
		\clearfield{volume}
		\clearfield{part}
		\clearname{editor}
	}
	{}%
}

%  ed./eds. immer in Klammern (auch nach Erstautor!)
\DeclareFieldFormat{editortype}{\mkbibparens{#1}}

% kein Komma vor ed. (z.B. wenn in 1. Zeile nach Erstautor)
\usepackage{xpatch}
\xpatchbibmacro{bbx:editor}
{\addcomma\space}{\addspace}{}{}    

%%%%%  Inbook, Incollection, Inproceedings, book
%% Ort und Jahr in Klammern, ohne Komma (Köln 2017)
\renewbibmacro*{publisher+location+date}{%
	%\printtext[parens]{%
	{%
		%\printlist{publisher}%   % Verlag wird nicht ausgegeben
		%\iflistundef{location}
		%{\setunit*{\addcomma\space}}
		%{\setunit*{\addcolon\space}}%
		\printlist{location}%
		\setunit*{\space}%
		\usebibmacro{date}%
	}\newunit%
}

%% Kein Punkt nach Titel bei book
%\xpatchbibdriver{book}
%{\newunit\newblock\usebibmacro{publisher+location+date}}{%
%	\setunit{\addspace}\newblock%
%	\usebibmacro{publisher+location+date}%
%}{}{}      
%
%% Kein Punkt nach Titel bei incollection 
%\xpatchbibdriver{incollection}
%{\newunit\newblock\usebibmacro{publisher+location+date}}{%
%	\setunit{\addspace}\newblock%
%	\usebibmacro{publisher+location+date}%
%}{}{}      
%
%% Kein Punkt nach Titel bei inbook 
%\xpatchbibdriver{inbook}
%{\newunit\newblock\usebibmacro{publisher+location+date}}{%
%	\setunit{\addspace}\newblock%
%	\usebibmacro{publisher+location+date}%
%}{}{}  