\subsubsection{Einzelfunde}\label{sec:LUS-Gr}

Eine Randscherbe der Lusako-Gruppe (Taf.~45.16; \textsc{Wotzka} 1995: 104--107) findet sich im zwölften Abtrag, bei 2,32--2,52\,m unter der Oberfläche, in der durch Pikunda-Munda-Keramik charakterisierten Verfüllung der Grube B1/B2, die im Schnitt PIK~87/1 (Kat.-Nr.~8) in Pikunda am \mbox{Sangha} (Fpl.~255) teilweise ausgegraben wurde. Keramik der Lusako-Gruppe ist vornehmlich auf die Ruki-Momboyo-Luilaka-Region des Inneren Kongobeckens beschränkt (ebd. 107, 550\,f. Karte~5) und wurde dort größtenteils bei Surveys gefunden. Das Stück aus Pikunda zeichnet sich durch einen leicht einbiegen und verdickten Rand (C1) mit breiter, gerade abgestrichener Randlippe aus, die \textsc{Wotzka} (ebd. 105) als Randform R37 systematisierte. Die Verzierung besteht aus Wiegeband (Tab.~\ref{tab:Verzierungselemente}: 04.2), überlagert von horizontalen Rillen (Tab.~\ref{tab:Verzierungselemente}: 02.1) außen am Rand sowie einem Band aus feinen, dreieckigen Eindrücken (Tab.~\ref{tab:Verzierungselemente}: 04.17) und Resten von Wiegeband (Tab.~\ref{tab:Verzierungselemente}: 04.2) auf der Randlippe. Das Stück ist zweifelsfrei mit Keramik der Pikunda-Munda-Gruppe (Kap.~\ref{sec:PKM-Gr}) vergesellschaftet. Die Radiokohlenstoffdatierung KI-2877, die den stratigrafisch jüngeren Grubenabschnitt B1 datiert, zeigt ein Alter zwischen dem 4.~Jh. v.~Chr. bis 3./4.~Jh. n.~Chr. an. Mit Lusako-Keramik zu assoziierende Befunde, die zudem datierbares Material enthielten, wurden im Inneren Kongobecken nicht gefunden, so dass \textsc{Wotzka} (ebd. 107) sich bei der absoluten Altersangabe der von ihm bearbeiten Keramik auf das Datum aus Pikunda bezieht. Es unterstützte die von ihm vertretene relativ-chronologische Mittelstellung der Lusako-Keramik zwischen den Stilgruppen Inganda (ebd. 78--84) und Lingonda (ebd. 108--115). Keine weitere GE aus dem Arbeitsgebiet konnten der Lusako-Gruppe zugeordnet werden, was allerdings auch an der starken Ähnlichkeit mit der Pikunda-Munda-Keramik (Kap.~\ref{sec:PKM-Gr}) liegt. Wie die Pikunda-Munda-Keramik weist die beschriebene GE einen Scherben auf, der keine nichtplastischen Partikel enthält und in der Folge dem \textit{Fabric} 1 zugerechnet werden kann. Auch die Verzierung basiert auf Elementen, die ebenfalls innerhalb des Pikunda-Munda-Stils beobachtet werden können. Bereits \textsc{Wotzka} (ebd.) beschreibt morphologische und ornamentale Ähnlichkeiten zwischen den Stilgruppen Lokondola, Lusako, Lingonda und Bokuma des Inneren Kongobeckens und der Pikunda-Munda-Keramik des nordwestlichen Kongobeckens (Tab.~\ref{tab:PIKMUN_Vgl}).