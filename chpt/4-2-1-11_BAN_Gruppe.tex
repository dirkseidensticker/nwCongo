\subsubsection{Bangui-Gruppe}\label{sec:BAN-Gr}

Neben der rouletteverzierten Keramik der Stilgruppen Dama (Kap.~\ref{sec:DAM-Gr}) und Mbati-Ngombe (Kap.~\ref{sec:MBN-Gr}) wurden 1985 bei der Prospektion des Ubangi an dessen Oberlauf eine Reihe von Gefäßen angekauft sowie bei Surveys gefunden, die keine Rouletteverzierung zeigen. Das als Bangui-Gruppe systematisierte Material setzt sich aus zehn GE zusammen, die sich an sieben verschiedenen Fundstellen fanden (Abb.~\ref{fig:BAN_Verbreitung}). Sicher der Bangui-Gruppe zurechenbare GE finden sich lediglich in einem kleinen Bereich zwischen Libenge im Süden (Fpl.~208) und Bangui im Norden (Fpl.~215). Eventuell der Stilgruppe zurechenbare Stücke fanden sich noch weiter Flussauf, zwischen Gbandami (Fpl. 226) und Kouango (Fpl.~229). Neben den zwei in Balongoi (Fpl. 214; Abb.~\ref{fig:BAN_Typen}.1,4) und der einen in Bangui angekauften GE (Fpl. 215; Abb.~\ref{fig:BAN_Typen}.3) stammen alle GE von Absammlungen rezenter Dorfflächen.

\begin{figure*}[tb]
	\centering
	\includegraphics[width=.8\textwidth]{figs/BAN-Gruppe.pdf}
	\caption{Bangui-Gruppe: Typvertreter.\\{\footnotesize 1:~Taf.~20.7; 2:~Taf.~16.1; 3:~Taf.~21.1; 4:~Taf.~20.8.}}
	\label{fig:BAN_Typen}
\end{figure*}
\clearpage
\paragraph{Technologische Merkmale}
$\;$ \\
Die Scherben der Bangui-Keramik zeichnen sich durch einen geringeren Anteil nichtplastischer Partikel aus, die vornehmlich den Größenklassen \textit{fine} (43\,\%) sowie \textit{medium} (29\,\%) zuzurechnen sind. Die Anteile nichtplastischer Partikel im Scherben liegt in der Regel zwischen 3--5\,\% (57\,\%) und 7--10\,\% (29\,\%). Es handelt sich vornehmlich um heterogene Mischungen aus Quarz, Laterit und Glimmer. Vor allem der hohe Anteil von Glimmer in den Scherben ist auffällig. Aufgrund der kleinen Stichprobe können diese Zahlen selbstverständlich nur bedingt als repräsentativ angesehen werden. Mit Blick auf die \textit{Fabrics} ließen sich die Varianten 3c, 5c sowie 6b und 8a beobachten. Die Brennfarbe des genutzten Tons konnte in keinem Fall sicher bestimmt werden. Während sich keine weiß-gebrannten Scherben fanden, zeigte eine GE eine rote Färbung. Das Gros des Materials weist eine graue bis beige Färbung auf. Die Oberflächen der Stücke sind durchweg glatt, in einem Fall zeigt die Oberfläche sogar deutlich Hinweise auf eine intensive Glättung, welche als Politur angesprochen werden kann. Die Wandungsdicke der Scherben liegt im Mittel bei 6.4\,mm.

\paragraph{Formen}
$\;$ \\
Lediglich bei sechs der zehn GE der Bangui-Gruppe war eine Ansprache der Gefäßform möglich. Zweidrittel der GE sind flache Gefäße mit geschweifter Wandung (Typ~E; Abb.~\ref{fig:BAN_Typen}.3--4). Ebenfalls vertreten ist ein flaschenförmiges Gefäße (Typ~A; Abb.~\ref{fig:BAN_Typen}.1) sowie ein schalenförmiges Gefäße mit konvexer Wandung (Typ~I; Taf.~24.2). Die Randlippen sind etwa zu gleichen Anteilen rund (M1), spitz (M2) oder gerade (M3) ausgearbeitet, während die Ausrichtung der Ränder grundsätzlich ausbiegend ist. Gerade (B1) sowie konkav (B2) oder konvex ausbiegende Ränder (B3) kommen ebenfalls etwa zu gleichen Anteilen vor. Die Halsbereiche der Gefäße sind häufig sehr kurz gehalten oder überhaupt nicht dezidiert ausgearbeitet. Bei vier GE der Bangui-Gruppe konnte die Ausprägung des Bodens beobachtet werden. Neben zwei runden Böden (B1; Abb.~\ref{fig:BAN_Typen}.4; Taf.~24.8) wurden auch zwei flache Standböden (B4; Abb.~\ref{fig:BAN_Typen}.1,3) beobachtet.

\begin{figure*}[p]
	\centering
	\includegraphics[width=\textwidth]{GIS/output/3-1-1-10_BAN_Verbreitung.pdf}
	\caption{Bangui-Gruppe: Verbreitung.}
	\label{fig:BAN_Verbreitung}
\end{figure*}

\paragraph{Verzierungen}
$\;$ \\
Die GE der Bangui-Gruppe zeichnen sich durch eine klare Strukturierung ihrer Verzierung aus. Auf den Unterseiten sowie Standflächen findet sich eine flächige, ans \textit{banfwa-nfwa} (Tab.~\ref{tab:Verzierungselemente}: 08) erinnernde aber deutlich irreguläre Aufrauung der Oberfläche (Tab.~\ref{tab:Verzierungselemente}: 22.2; 27\,\%; Abb.~\ref{fig:BAN_Typen}.1--3). Die oberen Gefäßteile zeichnen sich regelhaft durch horizontale Bänder runder bis leicht ovaler Eindrücke (Tab.~\ref{tab:Verzierungselemente}: 04.11; 24\,\%; Abb.~\ref{fig:BAN_Typen}.1--4) sowie horizontaler Riefen aus (Tab.~\ref{tab:Verzierungselemente}: 02.1; 26\,\%; Abb.~\ref{fig:BAN_Typen}.4).\footnote{Diese Dekorierung der Gefäßoberteile durch horizontale Bänder erinnern lose an die Verzierungspraxis der Kpetene-Gruppe (Kap.~\ref{sec:KPT-Gr}).} Die Eindrücke können auch größere geometrische Flächen ausfüllen (Abb.~\ref{fig:BAN_Typen}.1). Ebenso ließen sich Kombinationen aus horizontalen (Tab.~\ref{tab:Verzierungselemente}: 02.1) sowie winkelförmig ausgearbeiteten Riefen-Bändern beobachten (Tab.~\ref{tab:Verzierungselemente}: 01.6 und 02.3; Abb.~\ref{fig:BAN_Typen}.3). Auch die auf einer GE (Abb.~\ref{fig:BAN_Typen}.1--4) zu beobachtenden bogenförmigen Riefen könnten auf eine Beziehung zwischen der Bangui-Keramik zur Mokelo-Gruppe (Kap.~\ref{sec:MKL-Gr}) hinweisen.


\paragraph{Datierung}
$\;$ \\
Absolute Daten für die Datierung der Bangui-Keramik liegen nicht vor. Die beiden in Balongoi (Fpl. 214; Abb.~\ref{fig:BAN_Typen}.1,4) sowie die in Bangui (Fpl. 215; Abb.~\ref{fig:BAN_Typen}.3) als \textit{Enthographica} angekauften GE belegen jedoch die rezente Nutzung der Gefäße und legen ein grundsätzlich zeitgenössisches Alter nahe. Basierend auf diesem Quellenstand kann die Bangui-Keramik gegenwärtig nur als rezent eingestuft werden.

Die morphologischen Eigenschaften der Bangui-Gruppe erinnern mit Blick auf die grundsätzlich rundbauchigen Gefäße, das Fehlen ausgeprägter Halspartien sowie das Vorhandensein kurzer, ausbiegender Ränder an die Dama-Keramik (Kap.~\ref{sec:DAM-Gr}). Die flächige \enquote*{Behandlung} beziehungsweise Aufrauung der Gefäßunterseiten stellt eine Parallele der Bangui-Keramik zur Kpetene-Gruppe dar (Kap.~\ref{sec:KPT-Gr}). Die horizontalen Bänder aus Eindrücken hingegen stellen einen Bezug zur älteren Keramik der Mokelo-Gruppe (Kap.~\ref{sec:MKL-Gr}) dar, die etwa in der gleichen Region verbreitet war (Abb.~\ref{fig:MKL_Verbreitung}).

\paragraph{Verbreitung}
$\;$ \\
Bangui-Keramik findet sich in einem kleinen Verbreitungsgebiet direkt südlich der Hauptstadt der Zentralafrikanischen Republik Bangui, die der eponyme Fundplatz für die Stilgruppe ist (Fpl.~215). Der südlichste Fundplatz von sicher als Bangui ansprechbarer Keramik ist Libenge (Fpl.~208), während der nördlichste Bangui selbst ist. Etwa 160\,km stromauf von Bangui, kurz vor dem Ende der Befahrung von 1985 fanden sich zwischen Gbandami (Fpl.~226) und Kouango (Fpl.~229) GE, die möglicherweise der Bangui-Gruppe zuzuweisen sind. Eine Erklärung dieses Verbreitungsbildes kann, basierend auf der gegenwärtig vorliegenden, sehr begrenzten empirischen Datengrundlage, nicht gegeben werden.\footnote{Das engere Verbreitungsgebiet zwischen Libenge (Fpl.~208) und Bangui (Fpl.~215) könnte mit begrenztem Handel der Stücke erklärt werden.}