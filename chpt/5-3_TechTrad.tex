\section{Technologietraditionen des nordwestlichen Kongobeckens}\label{sec:TechnologieTrad}

Den gerade beschriebenen stilistischen Entwicklungslinien des Arbeitsgebiets (Kap.~\ref{sec:Horizonte}) soll im Folgenden der erste Entwurf einer Synthese der ebenfalls untersuchten Keramiktechnologie (Kap.~\ref{sec:Herstellung}) zur Seite gestellt werden. Hierfür wird das Konzept der Technologietradition\footnote{Siehe hierzu \textcites[31]{Manem.2010}[25, 65--139]{Ard.2014}[10]{vanDoosselaere.2014}.} herangezogen, das eine der Stiltradition entsprechende vertikale Integrationsebene beschreibt. Rezente Belege für die drei im Zuge dieser Arbeit unterschiedenen grundsätzlichen Herstellungstechniken weisen auf recht klar abgegrenzte Verbreitungsgebiete für durch Treiben, Abformen oder Aufbauen hergestellte Gefäße hin.\footnote{Siehe Anm.~\ref{ftn:EthnoToepfereiInVorb}.} Treiben, die im gesamten angrenzenden Inneren Kongobecken anzutreffende Technik, findet sich lediglich im Süden, in Bokelo (Fpl.~285) an der Einmündung des \mbox{Likwala}-\mbox{aux}-\mbox{Herbes} in den \mbox{Sangha}, während Abformen ausschließlich im äußersten Norden des Arbeitsgebietes, in Dama~I (Fpl.~222), Boduna (Fpl.~225) und Sidi (Fpl.~228) beobachtet wurde. Zwischen diesen beiden Gruppen finden sich bisweilen Belege für Aufbautechnik, so in Mbati-Ngombe (Fpl.~204) sowie Pikunda (Fpl.~255). Zusammengenommen ließen sich aus den individuellen \textit{\mbox{chaîne} opératoires}\footnote{\textit{Chaînes opératoires} ließen sich durch die Untersuchung von \textit{Fabrics} und von Makrospuren rekonstruieren (Tab.~\ref{tab:Makrospuren_ChaineOperatoire}). Die \textit{Fabrics} erlauben dabei Rückschlüsse auf die genutzten Rohmaterialien und deren Aufbereitung (Kap.~\ref{sec:Herstellung2_Fabric}). Auf Basis der Makrospuren ließen sich die potenziellen Herstellungstechniken ableiten (Kap.~\ref{sec:Herstellung2_Toepferei}). Die Ergebnisse der archäologischen Betrachtung wurden zudem mit rezenten Belegen abgeglichen (siehe Anm.~\ref{ftn:EthnoToepfereiInVorb}).} vier Technologietraditionen ableiten:

\textit{Technologietradition~1} wird aus den in Aufbautechnik hergestellten und die \textit{Fabrics} 1--2 aufweisenden Gefäßen gebildet (Var. B in Tab.~\ref{tab:Makrospuren_ChaineOperatoire}). Sie umfasst Vertreter der Stilgruppen Pikunda-Munda (Kap.~\ref{sec:PKM-Gr}), Ngombe (Kap.~\ref{sec:NGO-Gr}), Ebambe (Kap.~\ref{sec:EBA-Gr}) sowie Epena (Kap.~\ref{sec:EPE-Gr}). Auch das Gefäß der Imbonga-Gruppe aus Mobaka (Abb.~\ref{MKA87-102-1_Makrospuren}) lässt sich der \textit{Technologietradition~1} zuordnen, während das Gefäß aus Mitula (Abb.~\ref{MIT87-103-1_Makrospuren}), das ebenfalls das \textit{Fabric} 1 aufweist, keine diagnostischen Makrospuren zeigt. Es fällt auf, dass die \textit{Technologietradition~1} nahezu alle entlang des \mbox{Likwala}-\mbox{aux}-\mbox{Herbes} verbreiteten Stilgruppen umfasst (Tab.~\ref{tab:LikwalaSequenz}). Mit Blick auf die bekannten Belege für Töpferei im Inneren Kongobecken, namentlich aus Ikenge, Balinga-Bokonda, Liyolongo sowie Yopoko\footnote{Siehe Anm.~\ref{ftn:EthnoToepfereiInVorb}.} sowie das mutmaßliche Vorherrschen der \textit{Fabrics} 1--2 (Kap.~\ref{sec:Herstellung2_Fabric}) lässt sich festhalten, dass die \textit{Technologietradition~1} wohl auch die technologische Entwicklungslinie in diesem östlich an das Arbeitsgebiet angrenzenden Großraum widerspiegelt. Als Arbeitshypothese für zukünftige Untersuchungen kann gegenwärtig nur postuliert werden, dass die \textit{Technologietradition~1} das technologische Äquivalent zur \textit{Äquator-Co-Tradition} des Inneren Kongobeckens darstellt \parencite[Kap.~\ref{sec:ICB_StilGrDatierungen},][222 Abb.~4]{Wotzka.1995}.

Die \textit{Technologietradition~2} beschreibt Inventare, die vornehmlich am oberen \mbox{Sangha} sowie am befahrenen Teil des \mbox{Ngoko} verbreitet sind und sich durch Hinweise auf Aufbautechnik (Var. A in Tab.~\ref{tab:Makrospuren_ChaineOperatoire}), das \textit{Fabric} 3 und in geringeren Anteilen die \textit{Fabrics} 4--7 auszeichnen. Insbesondere die Keramik des  Mandombe-Stils weist mit einem Anteil von 97\,\% des \textit{Fabric} 3 (Tab.~\ref{tab:Fabrics_StilGr_Pct}) einen auffällig hohen Standardisierungsgrad in Bezug auf die Rohstoffe und ihre Aufbereitung auf. Auch wenn die bestehende Quellenlage durch die nur kleine Stichprobe nicht zufriedenstellend sein mag, so weisen die geschilderten Einzelbeobachtungen darauf hin, dass die \textit{Technologietradition~2} mit den keramischen Erzeugnissen der \textit{\mbox{Ngoko}-Tradition} in Zusammenhang gebracht werden kann.

Eine aktuell nur sehr begrenzt nachvollziehbare \textit{Technologietradition~3} entlang des \mbox{Ubangi}-Flusses lässt sich aus hohen Anteilen nichtplastischer Partikel im Scherben sowie Hinweisen für eine Herstellung in Aufbautechnik ableiten. Ein Gefäße des Dongo-Stils aus dem gleichnamigen Fundplatz (Abb.~\ref{DON85-101-71_Makrospuren}; Kap.~\ref{sec:DON-Gr}) weist auf eine Herstellung in Aufbautechnik hin (Var. A in Tab.~\ref{tab:Makrospuren_ChaineOperatoire}). Eine entsprechende Töpferei konnte 1985 in Mbati-Ngombe beobachtet werden.\footnote{Siehe Anm.~\ref{ftn:EthnoToepfereiInVorb}.} Das aus Dongo stammende Gefäß zeichnet sich zudem durch einen hohen Anteil nichtplastischer Partikel im Scherben aus und ist dem \textit{Fabric}~4 zuzuordnen. Dieses ist zusammen mit den \textit{Fabrics}~5--7 charakteristisch für eine Vielzahl von Stilgruppen entlang der Flüsse \mbox{Ubangi} und Lua (Tab.~\ref{tab:Fabrics_StilGr_Pct}). Ob die gegenwärtig nur unter Vorbehalten bestimmbare \textit{Technologietradition~3} sich durch Aufbautechnik und Zuschläge in häufig rotbrennenden Tonen auszeichnet, muss Gegenstand zukünftiger Untersuchungen sein.

Weiter nördlich, am oberen \mbox{Ubangi} können die Beobachtungen zu in Abformtechnik hergestellten Gefäßen aus Dama~I (Fpl.~222), Boduna (Fpl.~225) und Sidi (Fpl.~228) als Hinweise auf eine weitere, jedoch gegenwärtig nur vorläufig beschreibbare vierte \textit{Technologietradition} angesehen werden. Die entsprechenden Stücke zeichnen sich, wie auch die Vertreter der \textit{Technologietradition~3}, durch hohe Anteile nichtplastischer Partikel im Scherben aus. Es handelt sich um mutmaßlich intentionelle Zuschläge und die Stücke sind den \textit{Fabrics}~4--7 zuzuordnen. Da Hinweise auf die Herstellung in Abformtechnik gegenwärtig nur an rezenten Gefäßen beobachtet werden konnten und somit keine chronologische Entwicklung dieser Technik innerhalb des Arbeitsgebietes nachvollzogen werden kann, muss derzeit von der Beschreibung einer entsprechenden eigenständigen \textit{Technologietradition} Abstand genommen werden.

Gleiches gilt für die im äußersten Süden verbreiteten Inventare, die sich durch große Anteile zerstoßener Keramik beziehungsweise Schamott im Scherben und damit dem \textit{Fabric}~9 auszeichnen. Hinweise auf die Technik, in der diese Stücke hergestellt wurden, liegen gegenwärtig nicht vor. Da die entsprechenden, stilistisch größtenteils der Bobusa-Gruppe (Kap.~\ref{sec:BBS-Gr}) zuordenbaren Stücke ein auf die südliche Grenze des Arbeitsgebietes begrenztes Verbreitungsgebiet aufweisen (Abb.~\ref{fig:Fabrics_Verbreitung}, \ref{fig:BBS_Verbreitung}), das potenziell den Kongo weiter stromab reicht, kann -- zumindest hypothetisch -- postuliert werden, dass sie ebenfalls Repräsentanten einer eigenständigen \textit{Technologietradition} sind. Gegenwärtig fehlen für diese Gruppe hinreichende Quellen, um ihr tatsächliches Verbreitungsgebiet sowie ihre chronologische Entwicklung nachzeichnen zu können.