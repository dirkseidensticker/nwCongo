\chapter{Zusammenfassung}\label{sec:Zusammenfassung}
%\chapter*{Zusammenfassung und Ausblick}\label{sec:Zusammenfassung}
%\addcontentsline{toc}{chapter}{Zusammenfassung und Ausblick} 
%\chaptermark{Zusammenfassung und Ausblick}

Die vorliegende Arbeit liefert eine umfassende Auswertung der, durch das von der Deutschen Forschungsgemeinschaft finanzierte und von Manfred K.~H. Eggert geleitete \textit{River Reconnaissance Project}, in den Jahren 1985 und 1987 im nordwestlichen Kongobecken erschlossenen archäologischen Befunde und Funde. Sie stellt damit auch eine Fortführung der durch Hans-Peter \textcite{Wotzka.1995} vorgelegten Bearbeitung der zwischen 1977 und 1985 im Inneren Kongobecken untersuchten Fundkomplexe dar. Das Arbeitsgebiet umfasst die befahrenen Abschnitte der Flüsse Ubangi und Lua, Sangha und Ngoko sowie Likwala-aux-Herbes und deckt ein Gebiet von etwa 500\,$\times$\,700\,km ab. Untersucht wurde ein Fundensemble, das sich aus 122 Fundstelleninventaren mit annähernd 10\,500 Fundobjekte zusammensetzt, darunter 4217 einzeln aufgenommene Gefäßeinheiten sowie 4252 summarisch ausgezählte Einzelscherben.

Die Untersuchung hat zwei Hauptziele: Die Erarbeitung eines räumlich-zeitlichen Bezugsrahmens für das Arbeitsgebiet auf Basis der aus den technologischen, formalen und ornamentalen Charakteristika der Gefäßkeramik ableitbaren Stilgruppen (Kap.~\ref{sec:Keramiksequenz}) und eine archäologische Synthese der Besiedlungsgeschichte des Kongobeckens auf Basis der vorliegenden Quellen (Kap.~\ref{sec:BesiedlGesch}).

Im Zuge der Untersuchung wurden 24 für die Besiedlungsgeschichte des nordwestlichen Kongobeckens relevante keramische Stilgruppen beschrieben (Kap.~\ref{sec:BTM-Gr}--\ref{sec:BBS-Gr}). Des Weiteren finden sich Vertreter von fünf Stilen, die hauptsächlich im östlich angrenzenden Inneren Kongobecken verbreitet sind und von \textcite{Wotzka.1995} ausgearbeitet wurden (Kap.~\ref{sec:IMB-Gr}--\ref{sec:BOT-Gr}). Die umfassende Beschreibung der formalen und ornamentalen Charakteristika dieser Keramikstile sowie ihre Zeitstellung und Verbreitung bildet das Kernstück der Arbeit. Die aus Grabungsbefunden stammenden Inventare sowie Radiokohlenstoffdatierungen (Katalog~A) bilden den Ausgangspunkt für eine schematische Stilgruppenabfolge (Abb.~\ref{fig:Chronologiesystem}). Die Sequenz beginnt mit der frühesten bekannten Stilen des Arbeitsgebietes, den um die Zeitenwende datierenden Stilen Batalimo-Maluba (Kap.~\ref{sec:BTM-Gr}) und Pikunda-Munda (Kap.~\ref{sec:PKM-Gr}), und reicht bist zu in den 1980er Jahren in Herstellung und Nutzung befindlichen Töpfereierzeugnissen. Da Grabungen nur an fünf Fundstellen im nordwestlichen Kongobecken durchgeführt wurden lässt sich eine regionale Perspektive lediglich unter Einbeziehung der bei Surveys erschlossenen Funde (Katalog~B) aufzeigen. Die zwischen den einzelnen Stilen zu beobachtenden morphologischen und ornamentalen Bezüge bildeten die Grundlage der regionalen Sequenzen (Kap.~\ref{sec:Sequenzen}). Diese Bezüge sind im nordwestlichen Kongobecken deutlich schwächer ausgeprägt als im benachbarten Inneren Kongobecken, wo \textcite{Wotzka.1995} eine aus \enquote{stilistischen Bindegliedern geknüpfte Kette} (ebd. 65) sowie \enquote{dichtes Netz relativ-chronologisch interpretierbarer Relationen} (ebd. 282) beschreibt.

Zusätzlich zur Untersuchung und Systematisierung nach stilistischen Gesichtspunkten wurde die Gefäßkeramik auch auf Herstellungstechniken und Technologie hin analysiert. Während auf die Herstellungstechnik hinweisende Makrospuren nur auf Basis einer 28 Gefäßeinheiten umfassenden Stichprobe untersucht werden konnten (Kap.~\ref{sec:Herstellung2_Toepferei}), wurden auf die Aufbereitung der genutzten Tone hinweisende \textit{Fabrics} systematisch für alle aufgenommenen Gefäßeinheiten erfasst (Kap.~\ref{sec:Herstellung2_Fabric}). Die an ethnologischem Material beobachteten Herstellungsspuren (Kap.~\ref{sec:ToepfereiEthnogr}) lassen sich auf das aus archäologischen Kontexten stammende Fundgut übertragen. Lediglich jüngere Funde aus dem nordöstlichen Kongobecken wurden in ähnlicher Weise untersucht (Kap.~\ref{sec:NordCongo}). Innerhalb der untersuchten Stichprobe aus dem nordwestlichen Kongobecken konnten drei unabhängige Technologietraditionen identifiziert werden, während in zwei weiteren Fällen Informationen zur zeitlichen Tiefe der Beobachtung fehlten (Kap.~\ref{sec:TechnologieTrad}).

Die Synthese der regionalen keramischen Entwicklungslinien erfolgte in Anlehnung an bereits von \textsc{Wotzka} (ebd. 217--225, 284f.) hierzu genutzte Konzepte und Methoden, die ursprünglich in der amerikanischen Archäologie entwickelt wurden (Kap.~\ref{sec:NgokoTradition}). Stiltraditionen spiegeln danach regionale Entwicklungslinien jeweils auseinander hervorgegangener Stile wider und zeichnen eine durch die Zeiten nachvollziehbare Entwicklung nach. Im nordwestlichen Kongobecken lassen lediglich entlang des Ngoko sowie oberen Sangha verbreitete, ab dem 13.--15.~Jh.~n.~Chr. nachweisbare Stile hinreichend starke Bezüge untereinander erkennen, die eine Beschreibung als Stiltradition rechtfertigen. Stilhorizonte bilden ein konzeptuelles Gegenstück. Sie repräsentieren gleichzeitige Ausprägungen stilistisch eng verwandter und über einen größeren Raum verbreiteter Stile. 

In der Zusammenschau der vorhandenen Quellen zur Besiedlungsgeschichte des Kongobeckens deuten sich vier regional unabhängig voneinander entstandene stilistische Entwicklungslinien an (Kap.~\ref{sec:Zeitscheiben}). Die durch die Imbonga-Gruppe begründete \textit{Äquator-Co-Tradition} des Inneren Kongobeckens und die im nordwestlichen Kongobecken erfassten Stile Batalimo-Maluba und Pikunda-Munda sowie die keramische Entwicklung im nordöstlichen Kongobecken. 

Die in dieser Arbeit erarbeitete Besiedlungssequenz für das nordwestliche Kongobecken kann an manchen Stellen aufgrund unzureichender Quellen nur mit Vorbehalt beschrieben werden. Lediglich neue Feldarbeiten und vor allem Grabungen stratifizierter Befunde können hier Abhilfe schaffen. Die untersuchten Befunde und Funde erbrachten stichhaltige Daten vor allem für die frühen Abschnitte der Sequenz, die erste Phase der Besiedlung in der Frühen \linebreak\clearpage\noindent Eisenzeit.\footnote{Zeugnisse einer ausgeprägten Eisenmetallurgie finden sich in Kontexten, die mit der ältesten Besiedlung der südlichen Hälfte des nordwestlichen Kongobeckens assoziiert werden können.} Für die Epochen ab der Mitte des 1.~Jt.~n.~Chr. kann der vorliegende Datenbestand grundsätzlich als ungenügend eingeschätzt werden. Neuere Feldarbeiten wie beispielsweise jene der Arbeitsgruppe um \textcites{Oslisly.2013b}{MorinRivat.2014} erbrachten zwar neue Radiokohlenstoffdatierungen, jedoch stehen gegenwärtig Analysen der angetroffenen Befunde sowie Funde aus. Auf das Potential neuer Funde zur Verdichtung der Datenlage sowie der damit verbundene Möglichkeit zur Revision der hier erarbeiteten Sequenz muss nicht explizit eingegangen werden. Ungeachtet des umfangreiches Materialkorpus der durch das \textit{River Reconnaissance Project} in den 1980er Jahren gewonnen wurde, kann die vorliegende Arbeit lediglich einen ersten Einblick in die keramischen Variabilität des Arbeitsgebietes bieten. Während die Untersuchung der stilistischen Variation, die vornehmlich morphologische wie dekorative Eigenschaften der Keramik abdeckte und auf deren Basis die Sequenz grundsätzlich beruht (Kap.~\ref{sec:Keramiksequenz}--\ref{sec:BesiedlGesch}), stehen Untersuchungen zur Keramiktechnologie noch ganz am Anfang (Kap.~\ref{sec:Herstellung}).

%\section*{Summary and outlook}
%\addcontentsline{toc}{section}{Summary and outlook} 
%\chaptermark{Summary and outlook}

%\section*{Résumé et prévisions}
%\addcontentsline{toc}{section}{Résumé et prévisions} 
%\chaptermark{Résumé et prévisions}

%...