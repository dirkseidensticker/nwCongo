\chapter{Zusammenfassung}\label{sec:Zusammenfassung}
%\chapter*{Zusammenfassung und Ausblick}\label{sec:Zusammenfassung}
%\addcontentsline{toc}{chapter}{Zusammenfassung und Ausblick} 
%\chaptermark{Zusammenfassung und Ausblick}
\begin{multicols}{2}
\raggedcolumns
\noindent Die vorliegende Arbeit liefert eine umfassende Auswertung der archäologischen Befunde und Funde, die in den Jahren 1985 und 1987 im nordwestlichen Kongobecken durch das von der Deutschen Forschungsgemeinschaft finanzierte und von Manfred K. H. Eggert geleiteten, \textit{River Reconnaissance Project} gemacht wurden. Sie stellt damit eine Fortführung der durch Hans-Peter \textcite{Wotzka.1995} vorgelegten Bearbeitung der zwischen 1977 und 1985 im Inneren Kongobecken untersuchten Fundkomplexe dar. Das Arbeitsgebiet umfasst Abschnitte der Flüsse Ubangi, Lua, Sangha, Ngoko und Likwala-aux-Herbes und deckt ein Gebiet von etwa 500 mal 700 km ab. Untersucht wurde ein Ensemble aus 122 Fundstelleninventaren mit annähernd 10 500 Objekten, darunter jeweils etwa 4200 einzeln aufgenommene Gefäßeinheiten und summarisch ausgezählte Einzelscherben.

Die beiden Hauptziele der Untersuchung waren die Erarbeitung eines chronologisch-chorologischen Bezugsrahmens für das Arbeitsgebiet auf Basis der aus den technologischen, formalen und ornamentalen Charakteristika der Gefäßkeramik ableitbaren Stilgruppen (Kap.~\ref{sec:Keramiksequenz}) und die Vorlage einer Synthese der archäologischen Quellen zur Besiedlungsgeschichte des Kongobeckens (Kap.~\ref{sec:BesiedlGesch}).

Im Zuge der Untersuchung wurden 24 für die Besiedlungsgeschichte des nordwestlichen Kongobeckens relevante keramische Stilgruppen beschrieben (Kap.~\ref{sec:StilGr_nwCongo}). Des Weiteren konnten Vertreter von Stilen des östlich angrenzenden Inneren Kongobeckens, welche von \textsc{Wotzka} (ebd.) detailliert beschrieben wurden, identifiziert werden (Kap.~\ref{sec:InneresKongobeckenGruppen}). Die umfassende Beschreibung der formalen und ornamentalen Charakteristika dieser Keramikstile sowie ihre Zeitstellung und Verbreitung bildet das Kernstück der Arbeit. Die Grabungsinventare und mit ihnen assoziierte Radiokohlenstoffdatierungen (Katalog A) bilden den Ausgangspunkt für die Stilgruppenabfolge (Abb.~\ref{fig:Chronologiesystem}). Diese reicht von den kurz vor die Zeitenwende datierenden ältesten Stilgruppen Batalimo-Maluba (Kap.~\ref{sec:BTM-Gr}) und Pikunda-Munda (Kap.~\ref{sec:PKM-Gr}) bis zum Ende des 20. Jahrhunderts. Grabungen sind nur von fünf Fundstellen bekannt, was eine regionale Perspektive lediglich unter Einbeziehung der bei Surveys erschlossenen Funde zulässt (Katalog B). Zwischen den einzelnen Stilen beobachtete morphologische und ornamentale Ähnlichkeiten bilden die Grundlage der regionalen Sequenzen (Kap.~\ref{sec:Sequenzen}).

Zusätzlich zur Untersuchung und Systematisierung nach stilistischen Gesichtspunkten wurde die Gefäßkeramik auch auf Herstellungstechniken und Keramiktechnologie hin analysiert. Die Aufbereitung der die genutzten Tone beschreibenden Fabrics ist systematisch für alle aufgenommenen Gefäßeinheiten erfasst (Kap.~\ref{sec:Herstellung2_Fabric}). Auf die Herstellungstechnik hinweisende Makrospuren wurden auf Basis einer 28 Gefäßeinheiten umfassenden Stichprobe im Detail untersucht (Kap.~\ref{sec:Herstellung2_Toepferei}). Innerhalb der archäologischen Stichprobe konnten drei unabhängige Technologietraditionen identifiziert werden (Kap.~\ref{sec:TechnologieTrad}). Die auch an ethnologischem Material beobachteten Herstellungsspuren  lassen sich auf das aus archäologischen Kontexten stammende Fundgut übertragen. Allein jüngere Funde aus dem nordöstlichen Kongobecken wurden bisher in ähnlicher Weise untersucht (Kap.~\ref{sec:NordCongo}). 

Die Synthese der regionalen keramischen Entwicklungslinien erfolgte in Anlehnung an bereits von \textsc{Wotzka} (ebd.) genutzte Konzepte und Methoden (Kap.~\ref{sec:Horizonte}). Im nordwestlichen Kongobecken lassen entlang des Ngoko sowie oberen Sangha verbreitete Stile des 13.–15. Jh. n. Chr. hinreichend starke Ähnlichkeiten untereinander erkennen, um eine Beschreibung als Stiltradition zu rechtfertigen. In der Zusammenschau der existenten Quellen zur Besiedlungsgeschichte des Kongobeckens deuten sich drei regional unabhängig voneinander entstandene stilistische Entwicklungslinien an (Kap.~\ref{sec:Zeitscheiben}): die durch die Imbonga-Gruppe begründete Äquator-Co-Tradition des Inneren Kongobeckens,  eine im nördlichen Bereich des Kongobeckens durch die Batalimo-Maluba-Keramik eingeleitete Entwicklung und eine wiederum davon unabhängige im nordöstlichen Kongobecken.

Während die Keramik des nördlichen Abschnitts des Arbeitsgebietes einer eigenständigen Entwicklung folgt, weisen die Keramikgruppen des südlicheren Teils, im westlichen Abschnitt des Kongobeckens, entlang der Flüsse Sangha und Likwala-aux-Herbes, immer wieder Ähnlichkeiten zur Keramik aus dem Inneren Kongobecken auf. Gerade mit Blick auf die untersuchten Fabrics und Töpfereitechniken lassen sich nur wenige Unterschiede aufzeigen. Stile wie die Ngombe-Gruppe weisen sehr starke Ähnlichkeiten zu Stilen der Äquator-Co-Tradition auf und können überzeugend als Teil dieser angesehen werden. Für die älteste Stilgruppe in dieser Region, der Pikunda-Munda-Gruppe, stehen den starken technologischen Ähnlichkeiten eine Reihe von stilistischen Unterschieden entgegen. Dieser Stil ist bislang nur lose mit der zeitgleichen Keramik des Inneren Kongobeckens in Zusammenhang zu bringen. Die Pikunda-Munda-Keramik begründet auch keine eigene stilistische Entwicklung, da sich nach ihrem Ende in keiner jüngeren Stilgruppe belastbare Anknüpfungspunkte finden. Auffällig ist auch die Einführung von Roulette-Verzierung in die Region, die häufig die moderne Keramik bestimmt. Innerhalb der Ngoko-Tradition (Kap.~\ref{sec:Horizonte}) lässt sich eine sukzessive Übernahme und intensive Nutzung dieser Verzierungspraxis beobachten. Im äußersten Süden des Arbeitsgebietes wurde mit der Schamott-gemagerten Keramik der Bobusa-Gruppe mutmaßlich eine weitere eigenständige keramische Entwicklungslinie angeschnitten.

Die hier erarbeitete Besiedlungssequenz des nordwestlichen Kongobeckens wird an manchen Stellen aufgrund unzureichender Quellenlage mit Vorbehalt beschrieben. Diesen Lücken können lediglich neue Feldarbeiten und Grabungen stratifizierter Befunde Abhilfe schaffen. Die vorliegenden Befunde und Funde erbrachten vor allem absolut chronologische Datierungsindizien für die frühen Abschnitte der Sequenz, die erste Phase der Besiedlung in der frühen Eisenzeit.  Für die Epochen ab der Mitte des 1. Jt. n. Chr. muss der vorliegende Datenbestand grundsätzlich als ungenügend eingeschätzt werden. Trotz des umfangreichen Materialkorpus, der durch das River Reconnaissance Project in den 1980er Jahren gewonnen wurde, kann die vorliegende Arbeit nur einen ersten Einblick in die keramische Variabilität des Arbeitsgebietes gewähren. Während die Untersuchung der stilistischen Variation, die vornehmlich morphologische wie dekorative Eigenschaften der Keramik abdeckte, die Basis der Sequenz bildet (Kap.~\ref{sec:Keramiksequenz}--\ref{sec:BesiedlGesch}), stehen Untersuchungen zur Keramiktechnologie noch ganz am Anfang (Kap.~\ref{sec:Herstellung}).
\end{multicols}

\vspace{1.5em}

\section*{Summary and outlook}
\addcontentsline{toc}{section}{Summary and outlook} 
\chaptermark{Summary and outlook}
\begin{multicols}{2}
\raggedcolumns
\noindent This work provides a comprehensive analysis of archaeological findings from the north-western Congo Basin that were discovered in 1985 and 1987 within the framework of the River Reconnaissance Project, which was led by Manfred K. H. Eggert and funded by the German Research Foundation (DFG). It also represents a continuation of Hans-Peter\textsc{ Wotzka’s} (1995) work on the findings from the Inner Congo Basin. The study area is comprised of sections of the rivers Ubangi, Lua, Sangha, Ngoko and Likwala-aux-Herbes and covers an area of about 500 by 700 km. An inventory of 122 sites and approximately 10,500 objects were examined, including roughly 4,200 vessel units and a similar amount of individual sherds.

The study’s main objectives were to develop a spatio-temporal reference framework for the area based on pottery groups derived from the technological, formal, and ornamental characteristics of the ceramics (chapter~\ref{sec:Keramiksequenz}) and to provide an archaeological synthesis of the available knowledge on the settlement history of the Congo Basin (chapter~\ref{sec:BesiedlGesch}).

During the study, 24 new ceramic style groups were described for the northwestern Congo basin (chapter~\ref{sec:StilGr_nwCongo}). Furthermore, five styles that have been found mainly within the Inner Congo basin and described by \textsc{Wotzka} (ibid.) could be identified (chapter~\ref{sec:InneresKongobeckenGruppen}). The comprehensive description of formal and ornamental characteristics of these ceramic styles as well as their chronology and distribution form this work’s core. The inventories and radiocarbon datings of excavations (catalogue A) are the starting point for a novel chronological sequence of pottery styles (Fig.~\ref{fig:Chronologiesystem}). This chronological framewok begins with the earliest known styles of the area, Batalimo-Maluba (chapter~\ref{sec:BTM-Gr}) and Pikunda-Munda (chapter~\ref{sec:PKM-Gr}), dating back to the first century BC, and continues until modern-day potter’s traditions that were recorded during the 1980’s. As excavations were carried out at only five sites within the study area, a regional perspective is only revealed by including the findings from surveys (catalogue B). The morphological and ornamental references observed between styles formed the basis of the regional sequences (chapter~\ref{sec:Sequenzen}).

In addition to stylistic characterizations, the studied ceramics were analyzed regarding fabrication techniques and technology. Fabrics, which reflect processing of clays during potting, were systematically recorded for all potsherds (chapter~\ref{sec:Herstellung2_Fabric}). Macroscopic traces of manufacturing techniques were examined within a representative sample of 28 vessels (chapter~\ref{sec:Herstellung2_Toepferei}). Three independent technology traditions could be identified within this sample (chapter~\ref{sec:TechnologieTrad}). Traces of production observed on ethnological vessel are reasonable analogies for the the archaeological material. Only recent finds from the northeastern Congo basin were investigated in a similar way (chapter~\ref{sec:NordCongo}). 

The synthesis of the regional ceramic development is based on concepts and methods already applied by \textsc{Wotzka} (1995) that were originally developed in American archaeology (chapter~\ref{sec:Horizonte}). Stylistic traditions thus reflect regional lines of development that emerge from different styles and indicate change through time. In the northwestern Congo Basin, along the Ngoko and upper Sangha, various styles occurring after the 13th to 15th c. AD show sufficiently strong similarities among each other to justify including them in the same stylistic tradition. The syntheses of the existing sources on the settlement history of the Congo Basin suggests three distinct stylistic trends that developed independently (chapter~\ref{sec:Zeitscheiben}): the Equator Co-tradition of the Inner Congo Basin beginning with the Imbonga style; an independent line of development in the northern parts of the Congo Basin starting with the Batalimo-Maluba style; and the ceramic development of the northeastern Congo Basin (see chapter~\ref{sec:NordCongo}).

While the ceramics found in the northern parts of the study area follow an independent trajectory, the styles of the southern and western parts of the Congo Basin show substantial similarities to contemporaneous finds from the Inner Congo Basin. Fabrics and shaping technology show only a few observable differences. Furthermore, styles like Ngombe have close stylistic ties to pottery from the Equator Co-Tradition and can be regarded as part of them. For Pikunda-Munda, the oldest ceramic style in this region, strong technological similarities stand in the way of a number of stylistic differences. So far, this style can only be loosely associated with contemporaneous ceramics of the Inner Congo Basin. Furthermore, the Pikunda-Munda style did not develop as an individual line of stylistic development. After its end there are no reliable links with any younger styles in the region. Equally striking is the introduction of roulette decorations into the region, which often govern decoration practices of the modern-day ceramics. A gradual adoption and intensive use of this ornamentation practice can be observed within the Ngoko tradition (chapter~\ref{sec:Horizonte}). In the extreme south of the study area, another distinct line of pottery development was observed within the grog-tempered Bobusa group.

The settlement sequence of the northwestern Congo Basin sketched out within this study must, at least in part, be taken cautiously due to the limited sources available. Only new fieldwork and excavations can remedy this situation. Thus far, the available data constitutes valid proof for the chrono-temporal position of the early parts of the sequence, during the Early Iron Age. However, available data from the middle of the 1st millennium AD onwards must be considered incomplete. Despite the extensive body of material obtained by the River Reconnaissance Project in the 1980s, the present work only provides a first insight into the ceramic variability of the region. While the study of stylistic variation covering morphological and decorative properties constitutes the basis of the sequence (chapters~\ref{sec:Keramiksequenz}--\ref{sec:BesiedlGesch}), studies on ceramic technology are still in their infancy (chapter~\ref{sec:Herstellung}).
\end{multicols}

\vspace{1.5em}

\section*{Résumé et prévisions}
\addcontentsline{toc}{section}{Résumé et prévisions} 
\chaptermark{Résumé et prévisions}
\begin{multicols}{2}
\raggedcolumns
\noindent Ce travail apporte une analyse complète des découvertes archéologiques du \textit{River Reconnaisance Project}, financé par la Fondation allemande de la recherche (DFG) et dirigé par Manfred K. H. Eggert, dans la partie nord-ouest du bassin du Congo entre 1985 et 1987. Il s’inscrit également dans la continuité de la thèse de Hans-Peter \textcite{Wotzka.1995} portant sur les résultats de recherche dans le bassin intérieur du Congo. La zone d’étude, d’une superficie d’environ 500 km sur 700 km comprend des sections des rivières \mbox{Ubangi}, Lua, \mbox{Sangha}, \mbox{Ngoko} et \mbox{Likwala}-\mbox{aux}-\mbox{Herbes}. 122 sites et environ 10\,500 objets ont été examinés, dont environ 4\,200 récipients et une quantité similaire des tessons de céramiques.

Les objectifs principaux de l’étude étaient de développer un cadre de référence spatio-temporel pour la région, basé sur des groupes dérivés des caractéristiques technologiques, morphologique et ornementales de la céramique (chapitre~\ref{sec:Keramiksequenz}) et de fournir une synthèse archéologique des connaissances disponibles sur l’histoire du peuplement du bassin du Congo (chapitre~\ref{sec:BesiedlGesch}).

Au cours de l'étude, 24 nouveaux groupes stylistiques ont ainsi été décrits pour le nord-ouest du bassin du Congo (chapitre~\ref{sec:StilGr_nwCongo}), auxquels s’ajoutent cinq groupes précédemment décrits par \textsc{Wotzka} (ibid.) dans le bassin intérieur du Congo (chapitre~\ref{sec:InneresKongobeckenGruppen}). La description complète des caractéristiques formelles et ornementales de ces groupes stylistiques, ainsi que leur chronologie et leur distribution constituent le cœur de cet ouvrage. Les inventaires et les datations radiocarbone des fouilles (catalogue A) constituent le point de départ d'une nouvelle séquence de groupes stylistiques (Fig.~\ref{fig:Chronologiesystem}). Cette séquence chronologique commence avec les premiers styles de céramiques connus de la région, Batalimo-Maluba (chapitre~\ref{sec:BTM-Gr}) et Pikunda-Munda (chapitre~\ref{sec:PKM-Gr}), qui datent des  derniers siècles avant notre ère et se poursuit jusqu'aux traditions modernes qui ont été enregistrées au cours des années 1980. Comme les fouilles ont été effectuées sur cinq sites seulement, la perspective régionale (catalogue B) ne peut réellement être dégagée que par la mise en avant du résultat des prospections. Les similitudes morphologiques et ornementales observées entre les styles ont constitué la base des séquences régionales (chapitre~\ref{sec:Sequenzen}).

En plus de la caractérisation stylistique, les technologies de fabrication des céramiques étudiées ont été analysées. Les recettes de pâtes indiquant le traitement des argiles utilisées ont été systématiquement enregistrées pour tous les tessons (chapitre~\ref{sec:Herstellung2_Fabric}). Les traces macroscopiques ont été examinées sur un échantillon représentatif de 28 récipients (chapitre~\ref{sec:Herstellung2_Toepferei}). Cet échantillonnage a permis d’identifier trois traditions technologiques distinctes (chapitre~\ref{sec:TechnologieTrad}). Par ailleurs, les traces de production observées sur les récipients ethnologiques ont été utilisées comme données de comparaison pour interpréter le matériel archéologique. Seules les découvertes récentes dans le nord-est du bassin du Congo ont fait l’objet d’une étude similaire (chapitre~\ref{sec:NordCongo}).

La synthèse du développement régional de la céramique est basée sur des concepts et des méthodes déjà appliqués par \textsc{Wotzka} (ibid.; chapitre~\ref{sec:Horizonte}). Ainsi,  les traditions stylistiques reflètent des lignées régionales qui émergent de différents styles et présentent un changement dans le temps. Dans le nord-ouest du bassin du Congo, le long du Ngoko et de la Sangha supérieure, certains groupes stylistiques apparus entre le 13e et le 15e siècle de notre ère présentent des similitudes suffisantes  pour justifier leur inclusion une tradition stylistique commune. La synthèse des sources existantes sur l'histoire du peuplement du bassin du Congo suggère trois lignées d’évolutions stylistiques distinctes qui se sont développées indépendamment (chapitre~\ref{sec:Zeitscheiben}): la \textit{Co-Tradition Equatorial} du bassin intérieur du Congo, qui a commencé avec le style Imbonga, une lignée indépendante dans les parties nord du bassin du Congo à partir du style Batalimo-Maluba et celle du nord-est du bassin du Congo.

Alors que les céramiques trouvées dans les parties nord de la zone d'étude suivent une évolution indépendante, les styles des parties sud et  ouest du bassin du Congo présentent des similitudes importantes avec les découvertes contemporaines du bassin intérieur du Congo. En ce qui concerne les recettes de pâtes et la technologie de mise en forme, seules quelques différences peuvent être observées. De plus, des styles comme Ngombe ont des liens stylistiques étroits avec la poterie de la \textit{Co-Tradition Equatorial} et peuvent être considérés comme en faisant partie. Pour Pikunda-Munda, le plus ancien style de céramique de cette région, les fortes similitudes technologiques sont atténuées par des différences stylistiques. Les points de comparaison avec les céramiques du bassin intérieur du Congo sont en effet limités. En outre, le style Pikunda-Munda n'a pas développé  de lignée stylistique, rien ne permettant de le relier à un style plus récent de la région. L'introduction du décor à la roulette dans la région est également frappante, celui-ci étant souvent prépondérant dans la décoration des céramiques modernes. Une adoption progressive et une utilisation intensive de cette pratique d'ornementation peuvent être observées dans la tradition Ngoko (section~\ref{sec:Horizonte}). Dans l'extrême sud de la zone d'étude, une autre tradition a été observée avec le groupe Bobusa dégraissé par la chamotte.

La séquence du peuplement du nord-ouest du bassin du Congo présentée dans cette étude doit être, pour partie, abordée avec précaution en raison des sources limitées disponibles. Seuls de nouveaux travaux sur le terrain et des fouilles pourraient remédier à cette situation. Les données disponibles constituent néanmoins une indication valable de la position chrono-temporelle des premières parties de la séquence, au début de l'âge du fer. À partir du milieu du 1\textsuperscript{er} millénaire de notre ère, les données disponibles doivent cependant être considérées comme incomplètes. Malgré le vaste corpus de matériaux obtenu par le \textit{River Reconnaissance Project} dans les années 1980, le présent travail ne donne qu'un premier aperçu de la variation des styles céramiques dans la région. Alors que l'étude de la variation stylistique, comprenant les propriétés morphologiques et décoratives, constitue la base de la séquence (chapitres~\ref{sec:Keramiksequenz}--\ref{sec:BesiedlGesch}), les études sur la technologie de la céramique sont toujours en cours (chapitre~\ref{sec:Herstellung}).


\end{multicols}