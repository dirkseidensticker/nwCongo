\section*{\begin{tabular*}{\linewidth}{@{}l @{\extracolsep{\fill}} r@{}}
Nr.~11 & NGO~87/102 \\
\end{tabular*} 
}

\textsf{\textbf{Ngombe (\mbox{Sangha}, Fpl.~252)}}

\vspace{1em}

\noindent\begin{tabular}{@{}rl@{}}
	\textbf{Feldarbeit:} & \begin{tabular}[t]{@{}l@{}}\textbf{15.06.1987 (M. K. H. Eggert,}\\ \textbf{H. Holsten, K. Misago)}\end{tabular} \\ 
	\textbf{Abb.:} & \textbf{\ref{fig:NGO87-102}} \\ 
	\textbf{Taf.:} & \textbf{42.15--4.2} \\ 
	\textbf{Lit.:} & \textbf{--} \\ 
\end{tabular} 

\paragraph{Grabung und Befunde}\hspace{-.5em}|\hspace{.5em}%
In Ngombe am \mbox{Sangha} wurde bei der Prospektion der Dorffläche eine Keramikkonzentration entdeckt und notdürftig untersucht (Abb.~\ref{fig:NGO87-102}). Eine Verfärbung wurde nicht beobachtet. Es ist unklar, wie die Eingrabung aussah, in der die Gefäße deponiert wurden. Mindestens fünf Gefäße lagen eng gepackt in einem Bereich von etwa 0,35\,m Durchmesser, der bis knapp 0,3\,m unter die rezente Oberfläche reichte. Gefäßteile und Scherben wurden teilweise freigelegt und die Situation fotografisch dokumentiert (Abb.~\ref{fig:NGO87-102}). Ein großes, bauchiges Gefäß (Taf.~43.1) und weitere GE waren bereits \textit{in situ} in große Fragmente zerbrochen und wurden in- oder aneinander liegend aufgefunden.

\paragraph{Keramik}\hspace{-.5em}|\hspace{.5em}%
Die Keramik auf der Konzentration ist, aufgrund der potenziellen Geschlossenheit des Befundes, das charakterisierende Inventar für die Beschreibung des Ngombe-Stils (Kap.~\ref{sec:NGO-Gr}). Neben dem -- bereits erwähnten -- großen, bauchigen Gefäß des Typs D2 (Taf.~43.1), wurden zwei Teller vom Typ J1 (Taf.~44.1--2), eine größere Schale des Typs F6 (Taf.~42.1) sowie ein keines Gefäß mit scharfem Bauchknick vom Typ F4 entdeckt (Taf.~42.2). Letzteres erinnert sehr stark an die Gefäße der Longa-Gruppe \parencite[121 Typ 44]{Wotzka.1995} aus dem Inneren Kongobecken.\footnote{Die besten Vergleiche bilden Gefäße des Typs 44 nach \textsc{Wotzka} (1995: 121) aus Bokele (ebd. 455 Taf.~21.2, 456 Taf.~22.7) und Ikenge (ebd. 475 Taf.~41.9). Während die beiden reich verzierten Gefäße dieses Typs aus Longa (ebd. 485 Taf.~51.9--10) eine sehr ähnliche Gefäßform aufweisen, unterscheidet sich die Verzierung doch deutlich von dem Gefäß aus Ngombe (Taf.~42.2).} Die Keramik weist nur sehr wenige Verzierungen auf (Tab.~\ref{tab:NGO_Vgl_LON}). Die Verzierung des Gefäßes mit Bauchknick aus Ngombe besteht lediglich aus diagonalem Kammeindruck, einigen Eindruck-Reihen sowie horizontalen Rillen und beschränkt sich auf das Gefäßoberteil. Der bei Gefäßen des Longa-Stils regelhaft innen gerillte Rand lässt sich hingegen auch in Ngombe beobachten.

\begin{table*}[!t]
	\centering{\footnotesize
		\begin{sftabular}{@{}llllll@{}}
			\toprule 
			\textbf{Lab-Nr} & \textbf{Komplex} & \textbf{Datum (bp)} & \textbf{Datum (2-Sigma)} & \textbf{Probe} & \textbf{Tiefe} \\ 
			\midrule 
			KI-2894 & MKA~87/102 & 2270\( \pm \)160 & 786 v.~Chr.--5 n.~Chr. (95.4\,\%) & - & - \\ 
			KI-2895 & MIT~87/103 & 2230\( \pm \)100 & \begin{tabular}[t]{@{}l@{}}726--721 v.~Chr. (0,2\,\%);\\703--696 v.~Chr. (0,02\,\%);\\541--20 v.~Chr. (94,6\,\%)\\ 12--1 v.~Chr. (0,4\,\%)\end{tabular} & - & - \\ 
			\bottomrule 
	\end{sftabular}}
	\caption{MIT~87/103 \& MKA~87/102: Radiokohlenstoffdatierungen.}
	\label{tab:MIT87-103_14C-Daten}\label{tab:MKA87-102_14C-Daten}
\end{table*}

\paragraph{Datierung}\hspace{-.5em}|\hspace{.5em}%
Eine Probe für eine Radiokohlenstoffdatierung, die aus der Verfüllung zwischen den Gefäßfragmenten stammt, erbrachte bei der Aufbereitung im Kieler Labor leider kein datierbares Material.\footnote{Die Probe konnte auch bei einer Recherche im Restprobenarchiv des Labors, basierend auf einer Anfrage im Herbst 2013, nicht mehr gefunden werden und wurde wohl 1987 verworfen.} Das Inventar der Keramikdeponierung lässt sich daher lediglich basierend auf stilistischen Erwägungen und Vergleichen zur Longa-Gruppe des Inneren Kongobeckens provisorisch in das 12.--14.~Jh. n.~Chr. datieren (siehe Kap.~\ref{sec:NGO-Gr}).