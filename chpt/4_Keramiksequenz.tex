\chapter{Entwurf einer Keramiksequenz für das nordwestliche Kongobecken}\label{sec:Keramiksequenz}
%\chapter[Keramiksequenz]{Entwurf einer Keramiksequenz für das nordwestliche Kongobecken}

\section{Prämissen, Methoden und Konzepte}

Das Hauptziel dieser Arbeit liegt in der Erarbeitung eines räumlich-zeitlichen Bezugssystems für das nordwestliche Kongobecken (Abb.~\ref{fig:ArbeitsgebietKarte}). Die Entwicklung einer Keramiksequenz gründet auf Gemeinsamkeiten sowie Unterschieden der das Fundgut bestimmenden Gefäßkeramik (siehe Kap.~\ref{sec:GefKeramik}). Das folgende Kapitel widmet sich der detaillierten Beschreibung von 24 erarbeiteten keramischen Stilgruppen des Arbeitsgebietes. Die einzelnen Stile werden dabei differenziert nach zwei Großregionen -- die prospektierten Abschnitte der Flüsse Ubangi und Lua einerseits sowie Sangha, Ngoko und Likwala-aux-Herbes anderseits -- in chronologischer grober Reihenfolge abgehandelt. Dies spiegelt einerseits die Erschließung der jeweiligen Fundstellen wider, die Befahrung des Ubangi und Lua erfolgte 1985, während Sangha, Ngoko und Likwala-aux-Herbes 1987 erforscht wurden. Andererseits wird in ihr eine im Rahmen der Integration der regionalen Sequenzen beobachtete grundsätzliche Differenz zwischen den beiden Räumen aufgegriffen (siehe Kap.~\ref{sec:BesiedlGesch}).

Die morphologischen Eigenschaften sowie die Verzierung der Gefäßeinheiten (GE) wurde nach dem bereits von \textcite[52--57]{Wotzka.1995} herangezogenen Konzepts der \enquote*{keramischen Stilgruppe} systematisiert. Formal und ornamental gemeinsame GE wurden zu Gruppen zusammengefasst. Diese auch als \enquote*{Keramikstile} bezeichneten Gruppen beschreiben eine spezifische, unverwechselbare Art, Keramik herzustellen und zu verzieren. Die stark an \textsc{Wotzka} (ebd. 52--57) angelehnte Konzeptionierung wurde um durch Anschliffe gewonnene Erkenntnisse zur Keramiktechnologie (Kap.~\ref{sec:Herstellung2_Fabric}) erweitert. Diese als \textit{Fabrics} systematisierten Eigenschaften der GE wurden parallel zu den die \textit{Stilgruppen} definierenden morphologischen und ornamentalen Gesichtspunkten erfasst. In den folgenden Beschreibungen der Stilgruppen wird die technologischen Varianz jeder Gruppe im Detail besprochen.

Während die Stilgruppen die Grundeinheit der erarbeiteten Sequenz bilden (Kap.~\ref{sec:BesiedlGesch}), konnten durch die Inventare aus den Grabungsbefunden (Kat.-Nr.~1--19) chronologische \enquote*{Fixpunkte} innerhalb der Sequenz erarbeitet werden. Die nicht direkt in Grabungen erfassten, lediglich aus Oberflächensurveys bekannten Gruppen wurden in Relation zu diesen Fixpunkten in die Sequenz eingefügt. Die Vergesellschaftung von Formen wurde vornehmlich auf Basis der ausgegrabenen Befunde identifiziert. Die Surveyfunde konnten hier nur mittelbare Eindrücke bereithalten.

\section{Keramische Stilgruppen im nordwestlichen Kongobecken}\label{sec:StilGr_nwCongo}

\subsection{Ubangi- und Lua-Gebiet}

\input{chapters/3-1-1-1_BatMLB-Gruppe}

\input{chapters/3-1-1-2_NGB-Gruppe}

\input{chapters/3-1-1-5_DON_Gruppe}

\input{chapters/3-1-1-3_MKL_Gruppe}

\input{chapters/3-1-1-6_BBL_Gruppe}

\input{chapters/3-1-1-4_BKW_Gruppe}

\input{chapters/3-1-1-7_MTB_Gruppe}

\input{chapters/3-1-1-8_KPT_Gruppe}

\input{chapters/3-1-1-9_DAM_Gruppe}

\input{chapters/3-1-1-11_MBN_Gruppe}

\input{chapters/3-1-1-10_BAN_Gruppe}


\subsection{Sangha-, Ngoko- und Likwala-aux-Herbes-Gebiet}

\input{chapters/3-1-2-1_PIKMUN_Gruppe}

\input{chapters/3-2-3-1_Oveng_Gruppe}

\input{chapters/3-1-2-2_NGO_Gruppe}

\input{chapters/3-1-2-3_MAT_Gruppe}

\input{chapters/3-1-2-5_EBA_Gruppe}

\input{chapters/3-1-2-12_EPE_Gruppe}

\input{chapters/3-1-2-11_MKA_Gruppe}

\input{chapters/3-1-2-7_MDB_Gruppe}

\input{chapters/3-1-2-8_KON_Gruppe}

\input{chapters/3-1-2-6_OUE_Gruppe}

\input{chapters/3-1-2-9_PDM_Gruppe}

\input{chapters/3-1-2-10_MBJ_Gruppe}

\input{chapters/3-1-2-4_BBS_Gruppe}

\input{chapters/3-1-2-13_Sonstiges}


\section{Überregionale Kontaktfunde und Vergleiche}

\subsection{Inneres Kongobecken}\label{sec:InneresKongobeckenGruppen}

\input{chapters/3-2-1-1_IMB_Gruppe}

\input{chapters/3-2-1-4_BDG_Gruppe}

\input{chapters/3-2-1-3_MBA_Gruppe}

\input{chapters/3-2-1-5_BOT_Gruppe}

\input{chapters/3-2-1-2_Lusako_Gruppe}