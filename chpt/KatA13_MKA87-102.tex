\section*{\begin{tabular*}{\linewidth}{@{}l @{\extracolsep{\fill}} r@{}}
Nr.~13 & MKA~87/102 \\
\end{tabular*} 
}

\textsf{\textbf{Mobaka (\mbox{Sangha}, Fpl.~246)}}

\vspace{1em}

\noindent\begin{tabular}{@{}rl@{}}
	\textbf{Feldarbeit:} & \begin{tabular}[t]{@{}l@{}}\textbf{18.06.1987 (M. K. H. Eggert,}\\ \textbf{H. Holsten)}\end{tabular} \\ 
	\textbf{Taf.:} & \textbf{39.5} \\ 
	\textbf{Lit.:} & \textbf{\textsc{Eggert} 1992, 1993} \\ 
\end{tabular} 

\paragraph{Grabung und Befunde}\hspace{-.5em}|\hspace{.5em}%
Bei einer kurzen Prospektion in dem etwa 20--25~Hütten umfassenden Dorf Mobaka am \mbox{Sangha} wurden durch H.~Holsten Fragmente eines strichverzierten Imbonga-Gefäßes gefunden. Der Rand des Gefäßes war nicht erhalten. Es liegt weder eine aussagefähige schriftliche noch eine fotografische Dokumentation der Situation vor.

\paragraph{Keramik}\hspace{-.5em}|\hspace{.5em}%
Neben dem der Imbonga-Gruppe zuordenbaren Gefäß (Taf.~39.5) fanden sich noch drei weitere Scherben in diesem Komplex. Eine direkte Anpassung an das Gefäß ist nicht möglich. Während letzteres dem \textit{Fabric} 1a zuzuordnen ist, zeigen diese drei Scherben eine randliche Oxidation auf der Außenseite und entsprechen dem \textit{Fabric} 1c. Sie zeigen, im Unterschied zum Gefäß, keine Merkmale, die eine Ansprache des Stils möglich machen, können aber auf Basis des \textit{Fabrics} ohne Weiteres ebenfalls in den Imbonga-Kontext gerechnet werden. 

Bei dem Gefäß (Taf.~39.5) handelt es sich nicht um eine der \textit{klassischen} Imbonga-Formen (siehe Abb.~\ref{fig:Wotzka1995_TypenICB_EIA1}.1--4). Im Material aus dem Inneren Kongobecken fand sich jedoch eine sehr gute Entsprechung für die Form und Verzierung in einem Gefäß aus Bokele am Ruki \parencite[453 Taf.~19,10; Fpl.~14]{Wotzka.1995}.

\paragraph{Datierung}\hspace{-.5em}|\hspace{.5em}%
Im Sediment aus dem Gefäß fand sich eine Holzkohle, die durch Radiokohlenstoffdatierung in das 8.--1.~Jh. v.~Chr. datiert wurde (Tab.~\ref{tab:MKA87-102_14C-Daten}). Die Datierung des Fundes fällt -- trotz eines vergleichsweise großen Standardfehlers -- in den aus dem Inneren Kongobecken bekannten Zeithorizont für den Imbonga-Stil \parencite[66--67; siehe Kap.~\ref{sec:ICB_StilGrDatierungen}]{Wotzka.1995}.