%\paragraph*{Unveröffentlichte, auf das Quellenmaterial bezogene Dokumente}
%\subsection*{Unveröffentlichte, auf das Quellenmaterial bezogene Dokumente}
%\chaptermark{Sonstige Quellen}
%\sectionmark{Sonstige Quellen}
%\begin{multicols}{2}
%\noindent Das Korpus aus unveröffentlichten Manuskript-Entwürfen und internen Berichten umfasst folgende Dokumente:\columnbreak\vfill\null
%\end{multicols}
%\begin{multicols}{2}
{\footnotesize
\noindent{\sffamily \textbf{Unveröffentlichte, auf das Quellenmaterial bezogene \\ Dokumente}}
\begin{itemize}[leftmargin=9mm]
\setlength\itemindent{-9mm}
\item[] \textsc{Caselitz} 1986: Anthropologischer Bericht zu den Skelettresten aus MLB~85/1-4-3 (Kat.-Nr.~3). Eine Kurzfassung findet sich bei \textcite[144]{Eggert.1987c}.
\item[] 1988 angefertigte summarische Kurzbeschreibung der Keramik von 1987 durch Hermann Holsten. Die Liste diente zur Überprüfung der Inventare auf Vollständigkeit.
\item[] \textsc{Eggert} 1988: Aktennotiz zur Vermischung der Keramik aus Ngombe am \mbox{Sangha} (Fpl.~252) und Ngombe am \mbox{Likwala}-\mbox{aux}-\mbox{Herbes} (Fpl.~283; siehe Anm.~\ref{ftn:Vermischungen}).
\item[] Unveröffentlichtes Manuskript von C. Kanimba Misago aus dem Jahr 1992 zur oralen Tradition und Eisenverarbeitung in der Region Mbandaka.
\item[] Gliederung von M. K. H. Eggert aus dem Jahr 1995 für eine gemeinsam mit C. Kanimba Misago und Thomas Knopf geplante monografische Vorlage der Ergebnisse der Prospektionen von 1985 und 1987 entlang der Flüsse \mbox{Ubangi}, Lua, \mbox{Sangha}, \mbox{Ngoko} und \mbox{Likwala}-\mbox{aux}-\mbox{Herbes}.
\item[] \textsc{Kanimba Misago} 1995: Unveröffentlichtes Manuskript zum Metallurgiebefund PIK~87/3 (Kat-Nr.~10).
\item[] \textsc{Eggert, Holsten} u.~a. 1996: Gliederung und Teile eines unveröffentlichten Manuskriptes von M. K. H. Eggert, H. Holsten, C. Kanimba Misago \& F. Nikulka zu den in Munda (Fpl.~304) ausgegrabenen Metallurgiebefunden (Kat.-Nr.~15--18).
\item[] \textsc{Francken 2008}: unveröffentlichter Bericht zu den Skelettresten aus der Fundstelle Lobethal (Kamerun) von Michael Francken.
\item[] \textsc{Humphris \& Nordland} 2016: Unveröffentlichter Bericht über archäometallurgische Untersuchungen an Schlackefunden aus dem nordwestlichen und Inneren Kongobecken.
\end{itemize}}
%\end{multicols}