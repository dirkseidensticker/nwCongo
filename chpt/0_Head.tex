% Drucklegung
% ===========

\documentclass[11pt, a4paper, twoside, numbers = noenddot, headings = optiontoheadandtoc, captions = nooneline]{scrbook}	% BCOR = Bindungskorrektur; noenddot entfertn Pkt nach Kap.-Nr., nooneline: alle Bildunterschiften sind immer linksbündlig
\usepackage[ngerman]{babel}

% Schriften
% ---------
\usepackage{ebgaramond}		% Schriftart auf Garamond
\usepackage[semibold, light]{sourcesanspro}
\usepackage[T1]{fontenc}
\usepackage{tipa}	% Nutzung phonetischer Zeichen
\setkomafont{disposition}{\normalfont\bfseries}

% Seitenstil
% ----------
\usepackage[top = 30mm, bottom = 20mm, inner = 30mm, outer = 20mm]{geometry}
%\setlength\topmargin{-5.6mm}
%\setlength\headheight{6mm}
%\setlength\headsep{5mm}
\setlength{\columnsep}{6mm}	% Plastz zwischen den Spalten
\usepackage{fancyhdr}
\pagestyle{fancy}

% https://tex.stackexchange.com/questions/228362/get-sectionmark-in-fancyhdr-without-chapter-number/228363#228363
\renewcommand{\chaptermark}[1]{\markboth{#1}{}}	% entfernt die Nummerierung der Kapitel aus der Kopfzeile
\renewcommand{\sectionmark}[1]{\markright{#1}}	% entfernt die Nummerierung der Unterkapitel aus der Kopfzeile

\fancyhead{}
\fancyfoot{}
\fancyhead[LE]{{\fontsize{7.7pt}{10pt}\textsf{\thepage}}}
\fancyhead[RE]{{\fontsize{7.7pt}{10pt}\textsf{\MakeUppercase{\leftmark}}}}
\fancyhead[RO]{{\fontsize{7.7pt}{10pt}\textsf{\thepage}}}
\fancyhead[LO]{{\fontsize{7.7pt}{10pt}\textsf{\MakeUppercase{\rightmark}}}}
\renewcommand{\headrulewidth}{0.5pt}
% Seiteneinstellung für plain -> Seiten mit Kapitelanfang
\fancypagestyle{plain}{%
	\fancyhead{}
	\fancyfoot{}
	\renewcommand{\headrulewidth}{0pt}
}

% Fließtext
% ---------
\usepackage{setspace}
%\linespread{1.272727272727273}	% Faktor 1.272727272727273 für 14pt Abstand bei 11pt Schriftgröße

% Überschriften
% -------------
\addtokomafont{chapter}{\fontsize{17}{24}\sffamily\mdseries}
%\addtokomafont{section}{\MakeUppercase}
\addtokomafont{section}{\fontsize{13}{17}\sffamily\mdseries}
\addtokomafont{subsection}{\fontsize{11}{16}\sffamily\mdseries}
\addtokomafont{subsubsection}{\fontsize{11}{16}\sffamily\mdseries}
\addtokomafont{paragraph}{\fontsize{10}{14}\sffamily}

% Fußnoten
% --------
%\usepackage{footmisc}
\let\raggedfootnote\raggedright % linksbündige Fußnoten
\addtokomafont{footnote}{\fontsize{8.2pt}{12pt}\selectfont} % größe der Fußnoten
\addtokomafont{footnotelabel}{\sffamily\mdseries}
\addtokomafont{footnotetext}{\sffamily\mdseries}
\deffootnote{8mm}{1em}{
	\makebox[8mm][l]{\thefootnotemark}}	% Einzug der Fußnoten
\setfootnoterule[0.5pt]{25mm}
\usepackage{chngcntr}
\counterwithout{footnote}{chapter}	% Fußnotennummerierung läuft über die Kapitel hinweg
\interfootnotelinepenalty=0	% =0 ermöglicht einfaches Umbrechen vs. =1000 verhindert den Umbruch von Fußnoten 


% Abbildungen & Tabellen
% ----------------------
\usepackage{caption}
\usepackage{graphicx}
\usepackage{morefloats}	% mehr float-Obj können verarbeitet werden
\usepackage[labelfont = bf]{caption}	% fette Abbildungsnummern
\usepackage{subcaption}

\DeclareCaptionFont{captionfont}{\fontsize{7.5}{10}\sffamily}	% Größe von Abb./Tab. Beschriftungen

% format = hang für hängenden Einzug
\captionsetup[figure]{font = captionfont, labelfont = bf, labelsep = quad, justification=raggedright, format = plain}	% Abstand zwischen Nr und Text

\captionsetup[subfigure]{labelfont = bf, labelsep = quad, labelformat = mysublabelfmt, font = captionfont, justification=raggedright, format = plain}	% keine Klammern um die Bezeichnung der Subfigure
\renewcommand{\thesubfigure}{\Alph{subfigure}}	% große Buchstaben als Bezeichner

\captionsetup[table]{font = captionfont, labelfont = bf, labelsep = quad, justification=raggedright, format = plain}	% Abstand zwischen Nr und Text

\captionsetup[subtable]{labelfont=bf, labelsep = quad, labelformat = mysublabelfmt, font = captionfont, justification=raggedright, format = plain}	% keine Klammern um die Bezeichnung der Subfigure
\renewcommand{\thesubtable}{\Alph{subtable}}	% große Buchstaben als Bezeichner

\addto\captionsngerman{%
	\renewcommand{\figurename}{Abbildung}
	\renewcommand{\tablename}{Tabelle}
}

\newenvironment{sftabular}[1]%
{\sffamily \begin{tabular}{#1}}%
	{\end{tabular}}

% Tafeln
% ------
% neue Umgebung definieren (http://tex.stackexchange.com/questions/6478/new-figure-environment)
\usepackage{float}

%\newcommand\fs@plates{}
\floatstyle{plain}
\newfloat{pl}{thp}{lop}
\floatname{pl}{Tafel}
\captionsetup[pl]{font = captionfont, labelfont = bf, labelsep = quad}	% Abstand zwischen Nr und Text

% Literatur
% ---------
\usepackage[style = authoryear-icomp, 
 sorting = nyt, 
 backend = biber, 
 uniquename = false,
 uniquelist = false,
 language = auto, 
 mincitenames = 1,
 maxcitenames = 2, % nach 2 Autoren bereits u.a.
 maxbibnames = 100, 
 giveninits = true, % kürzt die Vornamen ab
 dashed = true, 
 ibidtracker = true,
 mergedate = false]{biblatex}
\bibliography{bib/bib.bib}
%\addbibresource{Lit.bib}
\renewcommand*{\mkbibnamefamily}[1]{\textsc{#1}}	% Autorennamen in Kapitälchen
%\renewcommand*{\mkbibnamelast}[1]{\textsc{#1}}	% bei älterem biblatex (<=3.3)
\renewcommand*{\bibfont}{\footnotesize}	% kleiner Text im Literaturverzeichnis

\renewcommand*{\postnotedelim}{\addcolon\space}	% Doppelpunkt anstatt Komma vor Seitenzahl
%\renewcommand*{\postnotedelim}{\ifciteibid{\addcolon\space}{\space}}

\renewcommand*{\finalnamedelim}{\addspace\&\space} % &-Zeichen zwischen Autoren

\DeclareFieldFormat{pages}{#1}	% entfernt 'S.' in Kurzzitaten & Literaturliste
\DefineBibliographyStrings{german}{%
	page = {{}{}},
	pages = {{}{}},
	% andothers = {{et\,al\adddot}},
}

% Bibliography

\AtEveryBibitem{
 \clearlist{publisher}
 %\clearfield{month}
 \clearfield{issn}
 \clearfield{isbn}
 \clearfield{url}
 \clearfield{doi}
}  

\renewcommand{\labelnamepunct}{\quad}         % Geviertleerzeichen zwischen Autor (2006) und Titel im LV
%\renewcommand{\bibnamedash}{--\hspace{7mm}}      % (Halbgeviert) als Ersatz für wiederkehrende Autoren oder Herausgeber in der Bibliografie verwendet wird
\renewcommand{\bibpagespunct}{\addcolon\addspace} % Seitenzahlen abtrennen mit : 
\DeclareFieldFormat{title}{\normalfont{#1}}    % Titel in normaler Schrift im LV
\DeclareFieldFormat[article,incollection]{title}{#1}      % Titel in Journals und Sammelbänden nicht in "Hochkomma" im LV
%\DeclareFieldFormat{myjournaltitle}{#1} % Zeitschriftentitel kursiv (funktioniert nur ohne \emph{} Auszeichnung!???)
\DeclareFieldFormat{series}{\emph{#1}} % Reihentitel kursiv
\DeclareFieldFormat{issuetitle}{#1} % Ausgabetitel nicht kursiv
\DeclareFieldFormat{maintitle}{#1} % Sammelbandtitel nicht kursiv
\DeclareFieldFormat{booktitle}{#1} % Sammelbandtitel nicht kursiv


\setlength\bibitemsep{1.8mm}   % Abstand zwischen Einträgen im Literaturvereichnis (1.8mm) im LV
\setlength\bibhang{9mm}
\renewcommand{\bibnamedash}{--\hspace{7mm}}

% Kein In: bei Artikeln 
\renewbibmacro{in:}{   
\ifentrytype{article}{}{\printtext{\bibstring{in}\intitlepunct}}}

% Volume gefolgt von Nummer in Klammern  
\DeclareFieldFormat[article]{number}{\mkbibparens{#1}}
\renewbibmacro*{volume+number+eid}{%
	\printfield{volume}%
	\printfield{number}}     

%% Bei Artikeln die zweite Jahreszahl nicht in Klammer setzen
\renewbibmacro*{issue+date}{%
	\setunit{\addcomma\space}% NEW
	%  \printtext[parens]{% DELETED
	\iffieldundef{issue}
	{\usebibmacro{date}}
	{\printfield{issue}%
		\setunit*{\addspace}%
		%       \usebibmacro{date}}}% DELETED
		\usebibmacro{date}}% NEW
	\newunit}

%%%  Inbook, Incollection, Inproceedings
%  Editor nach dem In: und vor dem Titel
\renewbibmacro*{in:}{
	\ifentrytype{incollection}{%
		\DeclareNameAlias{editor}{first-last}
		\printtext{In: }
		\ifnameundef{editor}
		{}
		{\printnames{editor}%
			\addspace 
			\usebibmacro{editorstrg} % \mkbibparens{\usebibmacro{editorstrg}}, in diesem Falle keine Klammern nach Autor, ed. (2016)
			%\mkbibparens{\usebibmacro{editorstrg}}
			\setunit{\addcomma\addspace}% 
		}%
		\usebibmacro{maintitle+booktitle}
		\clearfield{maintitle}    %% Folgende Eintr�ge sollen nun nicht noch einmal ausgegeben werden 
		\clearfield{booktitle}
		\clearfield{volume}
		\clearfield{part}
		\clearname{editor}
	}
	{}%
}

%  ed./eds. immer in Klammern (auch nach Erstautor!)
\DeclareFieldFormat{editortype}{\mkbibparens{#1}}

% kein Komma vor ed. (z.B. wenn in 1. Zeile nach Erstautor)
\usepackage{xpatch}
\xpatchbibmacro{bbx:editor}
{\addcomma\space}{\addspace}{}{}    

%%%%%  Inbook, Incollection, Inproceedings, book
%% Ort und Jahr in Klammern, ohne Komma (Köln 2017)
\renewbibmacro*{publisher+location+date}{%
	%\printtext[parens]{%
	{%
		%\printlist{publisher}%   % Verlag wird nicht ausgegeben
		%\iflistundef{location}
		%{\setunit*{\addcomma\space}}
		%{\setunit*{\addcolon\space}}%
		\printlist{location}%
		\setunit*{\space}%
		\usebibmacro{date}%
	}\newunit%
}

%% Kein Punkt nach Titel bei book
%\xpatchbibdriver{book}
%{\newunit\newblock\usebibmacro{publisher+location+date}}{%
%	\setunit{\addspace}\newblock%
%	\usebibmacro{publisher+location+date}%
%}{}{}      
%
%% Kein Punkt nach Titel bei incollection 
%\xpatchbibdriver{incollection}
%{\newunit\newblock\usebibmacro{publisher+location+date}}{%
%	\setunit{\addspace}\newblock%
%	\usebibmacro{publisher+location+date}%
%}{}{}      
%
%% Kein Punkt nach Titel bei inbook 
%\xpatchbibdriver{inbook}
%{\newunit\newblock\usebibmacro{publisher+location+date}}{%
%	\setunit{\addspace}\newblock%
%	\usebibmacro{publisher+location+date}%
%}{}{}  








%\setstretch{1.3333} % 10pt has 12pt baselineskip
%\setlength\parskip{\z@ \@plus 1\p@}
\setlength\parindent{6mm}
\setlength\partopsep{6mm}

\usepackage{t1enc}
\usepackage[utf8]{inputenc}

\usepackage{amsmath}
\usepackage{amssymb}

\usepackage{enumitem}
\setitemize{leftmargin=*} % entfernt die Einrückung der item-Umgebung
\setenumerate{leftmargin=*} % entfernt die Einrückung der item-Umgebung

\usepackage[autostyle = true, german = guillemets]{csquotes}	% ermöglicht die Nutzung von franz. Anführungszeichen - mit \enquote{} beziehungsweise \enquote*{}

\usepackage{todonotes}				% lässt todo's im code zu
\usepackage[hidelinks]{hyperref}	% erstellt die farbigen Links im PDF und gibt die Möglichkeit URLs einzugeben




% #########################################################
% # ALTER HEAD                                            #
% #########################################################




\usepackage{mdwlist} % Listen ohne Zwischenabstand




% Abbildungen & Tabellen
% ======================

\makeatletter	% erzeugt in dem \ref auf subfigures eine Ausgabe : Fig.Subfig -- bspw. (Abb. 32.A)
\renewcommand\p@subfigure{\thefigure}
\renewcommand\thesubfigure{.\Alph{subfigure}}
\DeclareCaptionLabelFormat{mysublabelfmt}{\Alph{sub\@captype}}
\makeatother

\makeatletter	% erzeugt in dem \ref auf subfigures eine Ausgabe : Fig.Subfig -- bspw. (Abb. 32.A)
\renewcommand\p@subtable{\thetable}
\renewcommand\thesubtable{.\Alph{subtable}}
\DeclareCaptionLabelFormat{mysublabelfmt}{\Alph{sub\@captype}}
\makeatother

\usepackage{array}	% liefert vertikale Ausrichtung der Zelleninhalte von Tabellen mit m{}
\newcolumntype{P}[1]{>{\raggedright\arraybackslash}p{#1}}	% linksbündige Tabellen mit P
\newcolumntype{R}[1]{>{\raggedleft\let\newline\\\arraybackslash\hspace{0pt}}m{#1}}	% rechtsbündige Tabellen mit R

\usepackage{adjustbox}	% ermöglichst das Beschneiden von Grafiken

\usepackage{longtable}
\usepackage{supertabular}
\usepackage{arydshln}	% gestrichelte Linien mit \hdashline & \cdashline
\usepackage{multirow}	% http://ctan.org/pkg/multirow
\usepackage{multicol}
\usepackage{makecell}
\usepackage{booktabs}	% lässt \top-, \mid- & \bottomrule in Tab zu

\usepackage{chngcntr}	% setzt kontinuierliche Abbildungs- & Tabellennummerierung
\counterwithout{figure}{chapter}
\counterwithout{table}{chapter}







\renewcommand*{\figureformat}{\figurename~\thefigure}	% unterdrückt mögliche Sonderzeichen (Pkt.) hinter Abb.-Nr.
\renewcommand*{\tableformat}{\tablename~\thetable}		% vertikale Zellen in Tabelle

%\newcommand*\rot{\rotatebox{90}}						

\graphicspath{ {images/} }	% Bilderordner

\usepackage[figuresright]{rotating}	% Provides {sideways}{sidewaysfigure}{sidewaystable} environments; siehe https://en.wikibooks.org/wiki/LaTeX/Rotations
\usepackage{afterpage}	% longtable to the top of the next page

\usepackage{tikz}
\usetikzlibrary{shapes,arrows}
\usetikzlibrary{calc,
	arrows,decorations.pathmorphing,
	backgrounds,fit,positioning,shapes.symbols,chains}
\usetikzlibrary{decorations.pathreplacing}
\usepackage{smartdiagram}

% für Abbildungen die in Spaltenbreite gesetzt werden sollen
\newenvironment{Figure}
{\par\medskip\noindent\minipage{\linewidth}}
{\endminipage\par\medskip}

% Listen
% ======
% neue Umgebung definieren (http://tex.stackexchange.com/questions/6478/new-figure-environment)
\usepackage{float}
\floatstyle{plain}
\newfloat{ls}{thp}{lop}
\floatname{ls}{Liste}
\captionsetup[ls]{labelsep = quad}	% Abstand zwischen Nr und Text




% \input{header.tex}

% Text mit Kreis umschließen
% ==========================
\usetikzlibrary{arrows}
\usetikzlibrary{shapes}
\newcommand*\circled[1]{\tikz[baseline=(char.base)]{
            \node[shape = circle, draw, inner sep = 2pt] (char) {#1};}}

% =================================================================

%\usepackage{showframe}
\usepackage{blindtext}