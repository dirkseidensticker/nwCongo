% Manuskript
% **********
%\documentclass[12pt, a4paper, twoside, onecolumn, BCOR=5.00mm, numbers = noenddot, headings = optiontoheadandtoc, captions = nooneline]{scrbook}	% BCOR = Bindungskorrektur; noenddot entfertn Pkt nach Kap.-Nr., nooneline: alle Bildunterschiften sind immer linksbündlig
%\usepackage[top = 30mm, bottom = 30mm, left = 25mm, right = 25mm]{geometry}
%\usepackage{times}	% Schriftart auf Times ändern
% **********


% siehe Standop/Meyer 2002 Die Form der wissenschaftlichen Arbeit. 16. Aufl. (Wiebelsheim) 121f., 158f.
% **********
%\documentclass[12pt, a4paper, twoside, onecolumn, BCOR=10mm, numbers = noenddot, headings = optiontoheadandtoc, captions = nooneline]{scrbook}	% BCOR = Bindungskorrektur; noenddot entfertn Pkt nach Kap.-Nr., nooneline: alle Bildunterschiften sind immer linksbündlig
%\usepackage[width = 140mm, top = 40mm, bottom = 20mm]{geometry}
%\usepackage{times}	% Schriftart auf Times ändern
%\usepackage{fancyhdr}
%\pagestyle{fancy}
%\renewcommand{\chaptermark}[1]{\markboth{\thechapter\enspace#1}{}}
%\renewcommand{\sectionmark}[1]{\markright{\thesection\enspace#1}}
%\fancyhead{}
%\fancyhead[LE]{\thepage}
%\fancyhead[RE]{\nouppercase\leftmark}
%\fancyhead[RO]{\thepage}
%\fancyhead[LO]{\nouppercase\rightmark}
%\fancyfoot{}
%\renewcommand*\chapterpagestyle{empty}	% schaltet die Kapitelseiten auf empty
% **********

% AP-Stil
% *******
\documentclass[11pt, a4paper, twoside, twocolumn, numbers = noenddot, headings = optiontoheadandtoc]{scrbook}	% noenddot entfertn Pkt nach Kap.-Nr.
\usepackage[width = 167mm, top = 20mm, bottom = 34mm]{geometry}
\usepackage{fancyhdr}
\pagestyle{fancy}
\renewcommand{\chaptermark}[1]{\markboth{#1}{}}
\fancyhead{}
\fancyfoot{}
\fancyfoot[LE]{\thepage\hspace{6mm}\textit{Dirk Seidensticker}}
\fancyfoot[RO]{\textit{\nouppercase \leftmark}\hspace{6mm}\thepage}
%\fancyfoot[CE]{\textit{Arch\"aologie des nordwestlichen Kongobeckens}}
\renewcommand{\headrulewidth}{0pt}
% Seiteneinstellung für plain ändern, damit auf den ersten Seiten eines Kapitels auch die gleiche Fußnote kommt:
\fancypagestyle{plain}{%
	\fancyhead{}
	\fancyfoot{}
	\fancyfoot[LE]{\thepage\hspace{6mm}\textit{Dirk Seidensticker}}
	\fancyfoot[RO]{\textit{\nouppercase \leftmark}\hspace{6mm}\thepage}
}
%\usepackage[top = 22.5mm, bottom = 31.5mm, left = 25mm, right = 25mm]{geometry}
\setlength{\columnsep}{9mm}
\newcommand{\changefont}[3]{
\fontfamily{#1} \fontseries{#2} \fontshape{#3} \selectfont}
\changefont{ppl}{m}{n}	%Schriftart zu Palatino ändern
%\usepackage{palatinox}	% Palatino Linotype
% *******

\usepackage[utf8]{inputenc}
\usepackage[onehalfspacing]{setspace}	% Zeilenabstand auf 1,5
\usepackage[T1]{fontenc}
\usepackage{tipa}	% Nutzung phonetischer Zeichen
\setkomafont{disposition}{\normalfont\bfseries}

\usepackage{fontawesome}	% font awesome icons

\usepackage{amsmath}
\usepackage{amssymb}
\usepackage[ngerman]{babel}
\usepackage{enumitem}
\setitemize{leftmargin=*} % entfernt die Einrückung der item-Umgebung

\usepackage[autostyle = true, german = guillemets]{csquotes}	% ermöglicht die Nutzung von franz. Anführungszeichen - mit \enquote{} beziehungsweise \enquote*{}

\usepackage{todonotes}
\usepackage[hidelinks]{hyperref}	% erstellt die farbigen Links im PDF und gibt die Möglichkeit URLs einzugeben

\usepackage{mdwlist} % Listen ohne Zwischenabstand

\setkomafont{chapterentry}{\rmfamily \bfseries}

\renewcommand*{\chaptermarkformat}{}	% entfernt die Nummerierung der Kapitel aus der Kopfzeile
\renewcommand*{\sectionmarkformat}{}	% entfernt die Nummerierung der Unterkapitel aus der Kopfzeile

\renewcommand*\descriptionlabel[1]{\hspace\labelsep\normalfont #1}

% Fußnoten
% ========
\addtokomafont{footnote}{\footnotesize}	% größe der Fußnoten
\deffootnote{2em}{1em}{
	\makebox[2em][l]{\thefootnotemark}}	% Einzug der Fußnoten
\usepackage{chngcntr}
\counterwithout{footnote}{chapter}	% Fußnotennummerierung läuft über die Kapitel hinweg
\interfootnotelinepenalty=0	% =0 ermöglicht einfaches Umbrechen vs. =1000 verhindert den Umbruch von Fußnoten 

% Abbildungen & Tabellen
% ======================
\usepackage{graphicx}
%\usepackage[demo]{graphicx}	% zur schnellen Anzeige
\usepackage{morefloats}	% mehr float-Obj können verarbeitet werden
\usepackage[labelfont = bf]{caption}	% fette Abbildungsnummern
\usepackage{subcaption}
\captionsetup[subfigure]{labelfont=bf, labelsep = quad, labelformat = mysublabelfmt}	% keine Klammern um die Bezeichnung der Subfigure
\renewcommand{\thesubfigure}{\Alph{subfigure}}	% große Buchstaben als Bezeichner

\makeatletter	% erzeugt in dem \ref auf subfigures eine Ausgabe : Fig.Subfig -- bspw. (Abb. 32.A)
\renewcommand\p@subfigure{\thefigure}
\renewcommand\thesubfigure{.\Alph{subfigure}}
\DeclareCaptionLabelFormat{mysublabelfmt}{\Alph{sub\@captype}}
\makeatother

% subtable
\captionsetup[subtable]{labelfont=bf, labelsep = quad, labelformat = mysublabelfmt}	% keine Klammern um die Bezeichnung der Subfigure
\renewcommand{\thesubtable}{\Alph{subtable}}	% große Buchstaben als Bezeichner

\makeatletter	% erzeugt in dem \ref auf subfigures eine Ausgabe : Fig.Subfig -- bspw. (Abb. 32.A)
\renewcommand\p@subtable{\thetable}
\renewcommand\thesubtable{.\Alph{subtable}}
\DeclareCaptionLabelFormat{mysublabelfmt}{\Alph{sub\@captype}}
\makeatother




\usepackage{array}	% liefert vertikale Ausrichtung der Zelleninhalte von Tabellen mit m{}
\newcolumntype{P}[1]{>{\raggedright\arraybackslash}p{#1}}	% linksbündige Tabellen mit P
\newcolumntype{R}[1]{>{\raggedleft\let\newline\\\arraybackslash\hspace{0pt}}m{#1}}	% rechtsbündige Tabellen mit R

\usepackage{adjustbox}	% ermöglichst das Beschneiden von Grafiken

\usepackage{longtable}
\usepackage{supertabular}
\usepackage{arydshln}	% gestrichelte Linien mit \hdashline & \cdashline
\usepackage{multirow}	% http://ctan.org/pkg/multirow
\usepackage{multicol}
\usepackage{makecell}
\usepackage{booktabs}	% lässt \top-, \mid- & \bottomrule in Tab zu

\usepackage{chngcntr}	% setzt kontinuierliche Abbildungs- & Tabellennummerierung
\counterwithout{figure}{chapter}
\counterwithout{table}{chapter}

\usepackage{caption}	% entfernt den : hinter Abb./Tab.-Nr.
\captionsetup[figure]{labelsep = quad}	% Abstand zwischen Nr und Text
\captionsetup[table]{labelsep = quad}	% Abstand zwischen Nr und Text

\addto{\captionsngerman}{%
  \renewcommand*{\contentsname}{Inhalt}
  \renewcommand*{\listfigurename}{Abbildungen}
  \renewcommand*{\listtablename}{Tabellen}
  \renewcommand*{\figurename}{Abb.}
  \renewcommand*{\tablename}{Tab.}
}

\renewcommand*{\figureformat}{\figurename~\thefigure}	% unterdrückt mögliche Sonderzeichen (Pkt.) hinter Abb.-Nr.
\renewcommand*{\tableformat}{\tablename~\thetable}		% vertikale Zellen in Tabelle

%\newcommand*\rot{\rotatebox{90}}						

\graphicspath{ {images/} }	% Bilderordner

\usepackage[figuresright]{rotating}	% Provides {sideways}{sidewaysfigure}{sidewaystable} environments; siehe https://en.wikibooks.org/wiki/LaTeX/Rotations
\usepackage{afterpage}	% longtable to the top of the next page

\usepackage{tikz}
\usetikzlibrary{shapes,arrows}
\usetikzlibrary{calc,
	arrows,decorations.pathmorphing,
	backgrounds,fit,positioning,shapes.symbols,chains}
\usetikzlibrary{decorations.pathreplacing}
\usepackage{smartdiagram}

% Tafeln
% ======
% neue Umgebung definieren (http://tex.stackexchange.com/questions/6478/new-figure-environment)
\usepackage{float}
\floatstyle{plain}
\newfloat{pl}{thp}{lop}
\floatname{pl}{Tafel}
\captionsetup[pl]{labelsep = quad}	% Abstand zwischen Nr und Text

% Listen
% ======
% neue Umgebung definieren (http://tex.stackexchange.com/questions/6478/new-figure-environment)
\usepackage{float}
\floatstyle{plain}
\newfloat{ls}{thp}{lop}
\floatname{ls}{Liste}
\captionsetup[ls]{labelsep = quad}	% Abstand zwischen Nr und Text



% Literatur
% =========

% DAI-Paket
% *********
%\usepackage[style = archaeologie, backend = biber, kapitaelchen]{biblatex}	% DAI-Regeln
% *********

%\usepackage[style = authoryear-icomp, sorting = nyt, backend = biber, uniquename = false]{biblatex}	% maxcitenames = 2: nach 2 Autoren bereits u.a.; 
%\bibliography{Lit.bib}
%%\addbibresource{Lit.bib}
%\renewcommand*{\mkbibnamefamily}[1]{\textsc{#1}}	% Autorennamen in Kapitälchen
%%\renewcommand*{\mkbibnamelast}[1]{\textsc{#1}}	% bei älterem biblatex (<=3.3)
%\renewcommand*{\bibfont}{\footnotesize}	% kleiner Text im Literaturverzeichnis
%
%\DeclareFieldFormat{pages}{#1}				% entfernt 'S.' in Kurzzitaten & Literaturliste
%\DefineBibliographyStrings{german}{%
%   page = {{}{}},
%   pages = {{}{}},
%   % andothers = {{et\,al\adddot}},
%}

\usepackage[style = authoryear-icomp, 
 sorting = nyt, 
 backend = biber, 
 uniquename = false,
 uniquelist = false,
 language = auto, 
 mincitenames = 1,
 maxcitenames = 2, % nach 2 Autoren bereits u.a.
 maxbibnames = 100, 
 giveninits = true, % kürzt die Vornamen ab
 dashed = true, 
 ibidtracker = true,
 mergedate = false]{biblatex}
\bibliography{bib/bib.bib}
%\addbibresource{Lit.bib}
\renewcommand*{\mkbibnamefamily}[1]{\textsc{#1}}	% Autorennamen in Kapitälchen
%\renewcommand*{\mkbibnamelast}[1]{\textsc{#1}}	% bei älterem biblatex (<=3.3)
\renewcommand*{\bibfont}{\footnotesize}	% kleiner Text im Literaturverzeichnis

\renewcommand*{\postnotedelim}{\addcolon\space}	% Doppelpunkt anstatt Komma vor Seitenzahl
%\renewcommand*{\postnotedelim}{\ifciteibid{\addcolon\space}{\space}}

\renewcommand*{\finalnamedelim}{\addspace\&\space} % &-Zeichen zwischen Autoren

\DeclareFieldFormat{pages}{#1}	% entfernt 'S.' in Kurzzitaten & Literaturliste
\DefineBibliographyStrings{german}{%
	page = {{}{}},
	pages = {{}{}},
	% andothers = {{et\,al\adddot}},
}

% Bibliography

\AtEveryBibitem{
 \clearlist{publisher}
 %\clearfield{month}
 \clearfield{issn}
 \clearfield{isbn}
 \clearfield{url}
 \clearfield{doi}
}  

\renewcommand{\labelnamepunct}{\quad}         % Geviertleerzeichen zwischen Autor (2006) und Titel im LV
%\renewcommand{\bibnamedash}{--\hspace{7mm}}      % (Halbgeviert) als Ersatz für wiederkehrende Autoren oder Herausgeber in der Bibliografie verwendet wird
\renewcommand{\bibpagespunct}{\addcolon\addspace} % Seitenzahlen abtrennen mit : 
\DeclareFieldFormat{title}{\normalfont{#1}}    % Titel in normaler Schrift im LV
\DeclareFieldFormat[article,incollection]{title}{#1}      % Titel in Journals und Sammelbänden nicht in "Hochkomma" im LV
%\DeclareFieldFormat{myjournaltitle}{#1} % Zeitschriftentitel kursiv (funktioniert nur ohne \emph{} Auszeichnung!???)
\DeclareFieldFormat{series}{\emph{#1}} % Reihentitel kursiv
\DeclareFieldFormat{issuetitle}{#1} % Ausgabetitel nicht kursiv
\DeclareFieldFormat{maintitle}{#1} % Sammelbandtitel nicht kursiv
\DeclareFieldFormat{booktitle}{#1} % Sammelbandtitel nicht kursiv


\setlength\bibitemsep{1.8mm}   % Abstand zwischen Einträgen im Literaturvereichnis (1.8mm) im LV
\setlength\bibhang{9mm}
\renewcommand{\bibnamedash}{--\hspace{7mm}}

% Kein In: bei Artikeln 
\renewbibmacro{in:}{   
\ifentrytype{article}{}{\printtext{\bibstring{in}\intitlepunct}}}

% Volume gefolgt von Nummer in Klammern  
\DeclareFieldFormat[article]{number}{\mkbibparens{#1}}
\renewbibmacro*{volume+number+eid}{%
	\printfield{volume}%
	\printfield{number}}     

%% Bei Artikeln die zweite Jahreszahl nicht in Klammer setzen
\renewbibmacro*{issue+date}{%
	\setunit{\addcomma\space}% NEW
	%  \printtext[parens]{% DELETED
	\iffieldundef{issue}
	{\usebibmacro{date}}
	{\printfield{issue}%
		\setunit*{\addspace}%
		%       \usebibmacro{date}}}% DELETED
		\usebibmacro{date}}% NEW
	\newunit}

%%%  Inbook, Incollection, Inproceedings
%  Editor nach dem In: und vor dem Titel
\renewbibmacro*{in:}{
	\ifentrytype{incollection}{%
		\DeclareNameAlias{editor}{first-last}
		\printtext{In: }
		\ifnameundef{editor}
		{}
		{\printnames{editor}%
			\addspace 
			\usebibmacro{editorstrg} % \mkbibparens{\usebibmacro{editorstrg}}, in diesem Falle keine Klammern nach Autor, ed. (2016)
			%\mkbibparens{\usebibmacro{editorstrg}}
			\setunit{\addcomma\addspace}% 
		}%
		\usebibmacro{maintitle+booktitle}
		\clearfield{maintitle}    %% Folgende Eintr�ge sollen nun nicht noch einmal ausgegeben werden 
		\clearfield{booktitle}
		\clearfield{volume}
		\clearfield{part}
		\clearname{editor}
	}
	{}%
}

%  ed./eds. immer in Klammern (auch nach Erstautor!)
\DeclareFieldFormat{editortype}{\mkbibparens{#1}}

% kein Komma vor ed. (z.B. wenn in 1. Zeile nach Erstautor)
\usepackage{xpatch}
\xpatchbibmacro{bbx:editor}
{\addcomma\space}{\addspace}{}{}    

%%%%%  Inbook, Incollection, Inproceedings, book
%% Ort und Jahr in Klammern, ohne Komma (Köln 2017)
\renewbibmacro*{publisher+location+date}{%
	%\printtext[parens]{%
	{%
		%\printlist{publisher}%   % Verlag wird nicht ausgegeben
		%\iflistundef{location}
		%{\setunit*{\addcomma\space}}
		%{\setunit*{\addcolon\space}}%
		\printlist{location}%
		\setunit*{\space}%
		\usebibmacro{date}%
	}\newunit%
}

%% Kein Punkt nach Titel bei book
%\xpatchbibdriver{book}
%{\newunit\newblock\usebibmacro{publisher+location+date}}{%
%	\setunit{\addspace}\newblock%
%	\usebibmacro{publisher+location+date}%
%}{}{}      
%
%% Kein Punkt nach Titel bei incollection 
%\xpatchbibdriver{incollection}
%{\newunit\newblock\usebibmacro{publisher+location+date}}{%
%	\setunit{\addspace}\newblock%
%	\usebibmacro{publisher+location+date}%
%}{}{}      
%
%% Kein Punkt nach Titel bei inbook 
%\xpatchbibdriver{inbook}
%{\newunit\newblock\usebibmacro{publisher+location+date}}{%
%	\setunit{\addspace}\newblock%
%	\usebibmacro{publisher+location+date}%
%}{}{}  

\setlength{\columnsep}{8mm}	% Plastz zwischen den Spalten

% \input{header.tex}

% Text mit Kreis umschließen
% ==========================
\usetikzlibrary{arrows}
\usetikzlibrary{shapes}
\newcommand*\circled[1]{\tikz[baseline=(char.base)]{
            \node[shape = circle, draw, inner sep = 2pt] (char) {#1};}}

% =================================================================