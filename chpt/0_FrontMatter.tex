\urlstyle{same}	% URLs in der gleichen Schriftart wie der Text
\pagenumbering{gobble}
\title{Archäologische Untersuchungen\\zur eisenzeitlichen Besiedlungsgeschichte des nordwestlichen Kongobeckens}
\author{\parbox{.7\textwidth}{\normalsize\centering
 {Dissertation\\
 zur Erlangung des akademischen Grades\\
 Doktor der Philosophie\\
 in der Philosophischen Fakultät\\
 \vskip 5mm
 der Eberhard Karls Universität Tübingen
 {\LARGE 
 \vskip 25mm
 \textbf{Teil I}
% \textbf{Teil II}
 \vskip 5mm
 \textbf{Text}}}}}
% \textbf{Anhang}}}}}
% \vskip 5mm
\date{}%
\publishers{\normalsize{%
 vorgelegt von\\
 {\Large Dirk Seidensticker\\[.5ex]}%
 aus Hoyerswerda\\[\baselineskip]
 \the\year				% gibt nur die Jahreszahl aus
 %Stand: \today		% für das genaue Datum
 }}
\lowertitleback{%
 {%
 \begin{tabular}{ll} 
 Hauptberichterstatter: & Prof. Dr. Thomas Knopf\\ 
 Mitberichterstatter: & Prof. Dr. Manfred K. H. Eggert\\ 
 \end{tabular}
 \vskip 5mm
 }%Tag der mündlichen Prüfung: 20.12.2006}%
}

\maketitle
\cleardoublepage

\clearpage
\null\vfill
\begin{center}
\textit{Für Katharina}
\end{center}
\vfill
\clearpage

\cleardoublepage

\pagenumbering{roman}
% \chapter*{Abstract} Abstract goes here

% \chapter*{Dedication} To mum and dad

% \chapter*{Declaration} I declare that..

%\onecolumn % AP-Stil

\chapter*{Erklärung}

Ich erkläre hiermit, dass ich die zur Promotion eingereichte Arbeit mit dem Titel: \enquote{Archäologische Untersuchungen zur eisenzeitlichen Besiedlungsgeschichte des nordwestlichen Kongobeckens} selbständig verfasst, nur die angegebenen Quellen und Hilfsmittel benutzt und wörtlich oder inhaltlich übernommene Stellen als solche gekennzeichnet habe. Ich versichere an Eides statt, dass diese Angaben wahr sind und dass ich nichts verschwiegen habe. Mir ist bekannt, dass die falsche Abgabe einer Versicherung an Eides statt mit Freiheitsstrafe bis zu drei Jahren oder mit Geldstrafe bestraft wird.

\vspace{2.5em}
\noindent Tübingen, den 04. Oktober 2017

\vspace{1em}
\noindent Dirk Seidensticker



\cleardoublepage

\chapter*{Vorwort}

Die vorliegende Arbeit wäre ohne die vielfältige Unterstützung zahlreicher Personen und Institutionen nicht möglich gewesen. Ihnen allen gebührt mein aufrichtigster Dank.

Mein außerordentlicher Dank gilt Manfred K.~H. Eggert der mir nicht nur das Fundmaterial zur Bearbeitung überließ sondern mir durch seine intensive Betreuung auch die Finalisierung der Arbeit ermöglichte. Nicht minder dankbar bin ich Thomas Knopf für seine konstruktive Unterstützung und Betreuung in der Schlussphase. 

Von 2012 bis 2016 wurde die Arbeit durch  Hans-Peter Wotzka an der Universität zu Köln betreut. Für die wissenschaftliche Begleitung in dieser Zeit gebührt ihm mein Dank. Die vorliegende Arbeit versteht sich auch als Ergänzung der von Hans-Peter Wotzka im Jahr 1995 veröffentlichten, wegweisenden Untersuchung zum Fundmaterial des \textit{River Reconnaissance Project} aus dem Inneren Kongobecken. Zudem verdanke ich Hans-Peter Wotzka die Finanzierung von zwei Radiokohlenstoffdatierungen an Knochen eines Grabes aus Maluba am Lua sowie der Analyse von Schlacken durch Jane Humphris und Ole Nordland. 

Die in dieser Arbeit vorgelegten Untersuchungen zur Töpfereitechnologie, insbesondere die Analyse von Makrospuren an der Gefäßkeramik wäre nicht ohne die konstante Ermutigung und Unterstützung durch Anne Mayor (Genf) und Alexandre Livingstone Smith (Tervuren) möglich gewesen. Im Zuge einer Ende 2015 durchgeführte Reise nach Genf konnte ich erste Erfahrungen mit der entsprechenden Methodik sammeln. In Tervuren gewährte mit Alexandre Livingstone Smith Einblick in das damals in Bearbeitung befindliche Fundmaterial aus dem nordöstlichen Kongobecken. Für Hilfe und Unterstützung bei der Interpretation der Anschliffe danke ich Heiko Riemer (Köln).

Monika Doll und Angelika Wilk gilt mein Dank für die Bestimmung der Tierknochen und Michael Francken für die Unterstützung bei der erneuten Evaluierung des menschlichen Skelettmaterials. Neben Hilfe bei der Redaktion der Arbeit gilt 
Christina Vossler-Wolf (alle Tübingen) auch mein Dank für eine Expertise zu potentiell neuzeitlichen Importfunden. 
	
Für anregende Diskussionen und Einsatz bei der Korrektur der Arbeit danke ich Claudia Spohn-Drosihn und Julian Spohn sowie Christina Vossler-Wolf und Marion Etzel (alle Tübingen), Friederike Jesse (Köln), Melanie Augstein (Leipzig), Sebastian Kirschner (München) und Nicole Rupp (Frankfurt). Oliver Vogels (Köln) danke ich für eine Vorlage zur automatisierten Erstellung des Literaturverzeichnisses.

Abschließend möchte ich mich bei meiner Partnerin Katharina Jungnickel für ihre konstante Unterstützung bedanken. Sie hat mir geholfen, den Blick in der Schlussphase auf das Wesentliche zu lenken.

Eine phonetisch korrekte Wiedergabe afrikanischer Bezeichnungen und Namen erfolgte aus technischen Gründen nicht.

\vspace{2.5em}
\noindent Tübingen, den 04. Oktober 2017

\vspace{1em}
\noindent Dirk Seidensticker

\setcounter{secnumdepth}{3}		% Tiefe des Inhaltsverzeichnisses
\setcounter{tocdepth}{3}
\tableofcontents
\cleardoublepage

\pagenumbering{gobble}
