\pagestyle{empty}
\setcounter{secnumdepth}{3}		% Tiefe des Inhaltsverzeichnisses
\setcounter{tocdepth}{3}
\tableofcontents
\cleardoublepage

\thispagestyle{empty}
\section*{Vorwort}
\addcontentsline{toc}{chapter}{\hspace{1.5em}Vorwort}
\begin{multicols}{2}
\raggedcolumns
\noindent Die vorliegende Untersuchung stellt die überarbeitete Fassung meiner im Oktober 2017 an der Philosophischen Fakultät der Eberhard Karls Universität Tübingen eingereichten Promotionsschrift dar. Nach 2017 erschienene Literatur wurde nur in Einzelfällen eingearbeitet. Ohne die vielfältige Unterstützung zahlreicher Personen und Institutionen wäre diese Arbeit nicht möglich gewesen. Ihnen allen gebührt mein aufrichtiger Dank.

Besonderen Dank schulde ich Manfred K.~H. Eggert, der mir nicht nur das Fundmaterial zur Bearbeitung überließ, sondern mir durch seine intensive Betreuung auch die Fertigstellung der Arbeit ermöglichte. Nicht minder dankbar bin ich Thomas Knopf für seine konstruktive Unterstützung und Betreuung in der Schlussphase. Peter Breunig (Frankfurt a. M.) danke ich für sein Drittgutachten.

Von 2012 bis 2016 wurde die Arbeit durch Hans-Peter Wotzka, Forschungsstelle Afrika am Institut für Ur- und Frühgeschichte der Universität zu Köln, betreut. Für die wissenschaftliche Begleitung in dieser Zeit gebührt ihm mein Dank. Die vorliegende Arbeit versteht sich auch als Ergänzung zu der von ihm im Jahr 1995 veröffentlichten, wegweisenden Untersuchung zum Fundmaterial des \textit{River Reconnaissance Project} im Inneren Kongobecken. Zudem verdanke ich H.-P. Wotzka die Finanzierung von zwei Radiokohlenstoffdatierungen an Knochen eines Grabes in Maluba am Lua sowie die Analyse von Schlacken durch Jane Humphris und Ole Nordland. 

Die in dieser Arbeit vorgelegten Untersuchungen zur Töpfereitechnologie, insbesondere die Analyse von Makrospuren, wäre nicht ohne die konstante Ermutigung und Unterstützung durch Anne Mayor (Genf) und \mbox{Alexandre} \mbox{Livingstone} Smith (Tervuren) möglich gewesen. Bei einer 2015 durchgeführte Reise nach Genf konnte ich erste Erfahrungen mit der entsprechenden Methodik sammeln. In Tervuren gewährte mir \mbox{Alexandre} Livingstone Smith Einblick in das damals in Bearbeitung befindliche Fundmaterial aus dem nordöstlichen Kongobecken. Für Hilfe und Unterstützung bei der Interpretation der Anschliffe danke ich Heiko Riemer (Köln).

Im Zuge des von Manfred K. H. Eggert geleiteten Projekts zur Restaurierung der Eisenfunde des Gräberfeldes in Campo, Südkamerun, wurden in den Restaurierungswerkstätten des Römisch-Germanischen Zentralmuseums Mainz (RGZM) auch Eisenobjekte der Fundplätze Pikunda und Munda restauriert.

Monika Doll und Angelika Wilk gilt mein Dank für die Bestimmung der Tierknochen und Michael Francken für die Unterstützung bei der erneuten Evaluierung des menschlichen Skelettmaterials. Neben Hilfe bei der Redaktion der Arbeit gilt Christina Vossler-Wolf (alle Tübingen) auch mein Dank für eine Expertise zu potenziell neuzeitlichen Importfunden.

Für anregende Diskussionen und Einsatz bei der Korrektur der Arbeit danke ich Claudia Spohn-Drosihn und Julian Spohn sowie Christina Vossler-Wolf und Marion Etzel (alle Tübingen), Friederike Jesse (Köln), Melanie Augstein (Leipzig), Sebastian Kirschner (München) und Nicole Rupp (Frankfurt). Oliver Vogels (Köln) danke ich für eine Vorlage zur automatisierten Erstellung des Literaturverzeichnisses. Für eine Korrektur der englischsprachigen Zusammenfassung gilt mein Dank Christopher A. \mbox{Kiahtipes} (Tampa); Nicolas Nikis (Cambridge) danke ich für die Durchsicht der französischen Zusammenfassung.

Der finanziellen Unterstützung der Tübinger Universitätsstiftung ist es zu verdanken, dass diese Studie in der vorliegenden Form veröffentlicht werden kann. Für die gute Zusammenarbeit bei der Fertigstellung der Druckvorlage möchte ich mich beim Verlag Tübingen University Press, vor allem bei Sandra Binder bedanken.

Schließlich möchte ich mich ganz besonders bei meiner Partnerin Katharina Jungnickel für ihre konstante Unterstützung bedanken. Sie hat mir geholfen, den Blick in der Schlussphase auf das Wesentliche zu lenken. Ihr und meinen Eltern ist die vorliegende Arbeit gewidmet.

Eine phonetisch korrekte Wiedergabe afrikanischer Bezeichnungen und Namen erfolgte aus technischen Gründen nicht. Ebenfalls habe ich bewusst auf ein Verzeichnis der Bildunterschriften verzichtet. Alle bibliographischen Informationen wurden in die Bildunterschriften integriert.

\vspace{1em}
\noindent Gent, im September 2021 \hfill Dirk Seidensticker
\end{multicols}
\pagestyle{fancy}