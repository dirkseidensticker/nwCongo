\subsubsection{Pfeifen}\label{sec:Pfeifen}

Das Fundgut enthält neben der Gefäßkeramik auch 26 Tonpfeifen oder Bruchstücke von Pfeifen.\footnote{Siehe hierzu auch \parencites{Shaw.1960}{Philips.1983}{Cremer.2004}.} Lediglich zwei der Stücke stammen aus Grabungen beziehungsweise datierten Kontexten: ein Fragment eines zylindrischen Pfeifenkopfes mit horizontalen Rillen (Typ~2; Taf.~49.13) aus der rezenten Grube PIK~87/2 (Kat.-Nr.~9) in Pikunda (Fpl.~255) sowie ein ebenfalls unvollständiges Stück einer Pfeife mit einem Holm mit verdicktem Ende (Typ~1; Taf.~88.9) aus der subrezenten Grube MUN~87/1 (Kat.-Nr.~15) in Munda (Fpl.~304). Die übrigen 24 Stücke wurden bei Oberflächensurveys gefunden. Ebenfalls erwähnenswert sind insbesondere die Fundstellen Ebambe (Fpl.~297; Taf.~83.2,14) und Pandama am Ngoko (Fpl.~276; Taf.~65.8--9), an denen jeweils noch vier Pfeifen beziehungsweise Fragmente von Pfeifen gefunden wurden (Tab.~\ref{tab:Pfeifen_Fpl-Typen}). Das Gros der Tonpfeifen stammt von Fundstellen entlang des Likwala-aux-Herbes (15 Stücke). Von Fundstellen entlang des Sangha sind sechs, vom Ngoko vier Stücke belegt. Entlang des Ubangi wurden lediglich in Boyoka (Fpl.~196; Taf.~5.6) ein Fragment einer Tonpfeife entdeckt.

Eine formale Systematisierung der 26 vorliegenden Tonpfeifen erbrachte eine Differenzierung in vier Typen:
\setlength\LTleft{0pt}
\begin{longtable}{@{}lll@{}}
1 & Holm mit verdicktem Ende & (Taf.~88.9) \\
2 & zylindrischer Pfeifenkopf & (Taf.~49.13, 83.2) \\ 
3 & trichterförmiger Pfeifenkopf & (Taf.~65.8--9, 83.14) \\
4 & unspezifisch, rundlich & (Taf.~5.6, 39.3) \\
\end{longtable}
\addtocounter{table}{-1}

\begin{table*}[tb]
	%\begin{sidewaystable*}[p]
	\centering
	{\small \begin{sftabular}{@{}llllllllr@{}}
	\toprule
	{}            & {}                       & {}            & \multicolumn{5}{c}{\textbf{Typ}} &   \\
	\textbf{Flusslauf}    & \textbf{Fundort}                      & \textbf{Befund}        & \textbf{1}    & \textbf{2}    & \textbf{3}    & \textbf{4}    & \textbf{k.a.}     & \textbf{$\sum$} \\ \midrule
	\mbox{Ubangi}        & Boyoka (Fpl. 196)        & BYO 85/101    & -   & -   & -   & 1   & -    &  1 \\
	\mbox{Sangha}        & \mbox{Sangha} Fkm 40 (Fpl. 240) & SGH 40 87/040 & -   & -   & -   & -   & 1    &  1 \\
	{}            & Monjolomba (Fpl. 243)    & MJL 87/101    & -   & -   & -   & 2   & -    &  2 \\
	{}            & Pikunda (Fpl. 255)       & PIK 87/2      & -   & 1   & -   & -   & -    &  1 \\
	{}            & Molanda (Fpl. 258)       & MLD 87/101    & -   & -   & -   & 1   & -    &  1 \\
	{}            & Mandombe (Fpl. 259)      & MDB 87/101    & -   & -   & 1   & -   & -    &  1 \\
	\mbox{Ngoko}         & Pandama (Fpl. 276)       & PDM 87/101    & -   & -   & 3   & -   & 1    &  4 \\
	Likwala-Esobe & Bojenjo (Fpl. 292)       & BJJ 87/101    & -   & -   & -   & 2   & -    &  2 \\
	{}            & Ebambe (Fpl. 297)        & EBA 87/101    & -   & 1   & 2   & 1   & -    &  4 \\
	{}            & Mosenge (Fpl. 299)       & MSG 87/101    & -   & -   & 1   & -   & -    &  1 \\
	{}            & Botongo (Fpl. 302)       & BTG 87/101    & 1   & -   & -   & -   & 1    &  2 \\
	{}            & Jeke (Fpl. 303)          & JEK 87/101    & 1   & -   & -   & -   & -    &  1 \\
	{}            & Munda (Fpl. 304)         & MUN 87/1      & 1   & -   & -   & -   & -    &  1 \\
	{}            & {}                       & MUN 87/101    & 4   & -   & -   & -   & -    &  4 \\ \midrule
	{}            & {}                       & \multicolumn{1}{r}{\textbf{$\sum$}}            & 7   & 2   & 7   & 7   & 3    & 26 \\ \bottomrule
\end{sftabular}
}
	\caption{Pfeifen: Funde aus dem Arbeitsgebiet.}
	\label{tab:Pfeifen_Fpl-Typen}
	%\end{sidewaystable*}
\end{table*}

\noindent 23 der 26 Tonpfeifen aus dem Arbeitsgebiet ließen sich einem der vier Typen zuweisen. Die Typen 1 sowie 3 und 4 sind mit jeweils sieben Stücken gleich häufig vertreten, während Typ 2 nur zwei Individuen umfasst. Die von anderen Fundstellen wie Bisségué 1 in Gabun \parencite[688 Abb. 7-119]{Clist.20042005} bekannten, europäischen Tabakpfeifen\footnote{Aus europäischem Import stammende Tabakpfeifen zeichnen sich unter anderem dadurch aus, dass bei ihnen Pfeifenkopf, Rauchkammer und Kolben beziehungsweise Mundstück aus einem Stück gefertigt sind. Zudem sind sie deutlich feiner gearbeitet als die Tonpfeifen aus dem Arbeitsgebiet und weisen einen im Vergleich kleineren, runderen Pfeifenkopf auf (siehe \textsc{Clist} 2004/2005: 688 Abb. 7-119).} konnten im Inventar aus dem Arbeitsgebiet nicht beobachtet werden. Einheimische Formen, mit deutlich elaborierteren Ausprägungen des Pfeifenkopfes sind auch im von \textcite[18]{Coart.1907} vorgelegten kolonialzeitlichen, ethnografischen Fundgut vertreten. Zwei der abgebildeten Stücke zeigen die markanten, teilweise auch spitzen Ausziehungen am unteren Ende der Rauchkammer, wie sie vornehmlich bei den Pfeifen des Typs 1 beobachtet wurden (Taf.~88.9). Einige Pfeifen des Typs 3 wiesen an dieser Stelle eine Öse auf (Taf.~65.8--9). Zusätzlich zur formalen Differenzierung wurden auch quantitative Daten in Form von Messstrecken erhoben. Tonpfeifen vom Typ 1 wurden ausschließlich an Fundstellen am Oberlaufs des Likwala-aux-Herbes beobachtet (Tab.~\ref{tab:Pfeifen_Fpl-Typen}).

%\todo[inline]{
%Weitere Literatur
%\parencite{Shaw.1960}
%\parencite{Philips.1983}
%\parencite{Cremer.2004}	
%}
