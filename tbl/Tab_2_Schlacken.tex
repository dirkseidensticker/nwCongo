\begin{footnotesize}
%\begin{sftabular}{@{}m{.05\textwidth}m{.1\textwidth}m{.1\textwidth}m{.1\textwidth}m{.1\textwidth}m{.39\textwidth}@{}}
\begin{sftabular}{@{}ccccm{.2\textwidth}m{.46\textwidth}@{}}
\toprule
\textbf{Typ} & \textbf{Fließ"-str.} & \textbf{Kantig} & \textbf{Magn.} & \textbf{Farbe} & \textbf{Beschreibung} \\
\midrule
1a &  $\bullet$ &  & $\bullet$ & metallisch grau/rostig rot & wachstropfenförmig, blasig, sehr selten Abdrücke organischen Materials, auffallend leicht, viele Hohlräume an Bruchkanten sichtbar \\
1b &  $\bullet$ & & $\bullet$ & rötlich/violett & wie 1a, nur rötlich/violett \\
2a &  $\bullet$ &  &  & metallisch grau/rostig rot & wie 1a, nur nicht magnetisch \\
2b &  $\bullet$ &  &  & grünlich & wie 2a, nur grünlich \\
2c &  $\bullet$ &  &  & rötlich/violett & wie 2a, nur rötlich bis violett\\
3 & &  $\bullet$ & $\bullet$ & metallisch grau/rostig rot & kantig, meist massiv ohne Hohlräume, selten Abdrücke organischen Materials; teilweise auffallend schwer \\
4a &  &  $\bullet$ &  & metallisch grau/rostig rot & wie 3, nur nicht magnetisch, enthält teilweise auch Gesteinsbrocken \\
4b &  &  $\bullet$ &  & grünlich & wie 4a, nur grünlich \\
5 & &  &  & rostig rot & knollenförmig, massiv, rostig rot, auch im Bruch; meist spielwürfelgroß \\
6 &  &  &  & hellgrau & ähnlich Typ 2a, kantig, viele kleine Luftbläschen, sehr selten Abdrücke organischen Materials, auffallend leichter als Typ 2a.\\
\bottomrule
\end{sftabular}
\end{footnotesize}